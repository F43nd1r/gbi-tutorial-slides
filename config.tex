
% This is a configuration file with personal tutor information.
% It is therefore excluded from the git repository, so changes in this file will not conflict in git commits.

% Copy this template, rename to config.tex and add your information below.

\newcommand{\myname}{Lukas Morawietz}
\newcommand{\mymail}{lukas.morawietz@gmail.com} % Consider using your named student mail address to keep your u**** account private.
\newcommand{\mytutnumber}{31}

% Don't forget to update ILIAS url. WARNING: Underscores '_' and Ampersands '&' have to be escaped with backslashes '\'. Blame TeX, not me.
\newcommand{\myILIASurl}{https://ilias.studium.kit.edu/ilias.php?ref\_id=855240\&cmdClass=ilrepositorygui\&cmdNode=5r\&baseClass=ilrepositorygui}

% Uncommenting this will print Socrative info with here defined roomname whenever \Socrative is called.
% (Otherwise, \Socrative will remain silent.)
% \newcommand{\mysocrativeroom}{???}

%\def\ThassesTut{}
\def\DanielsTut{}

\newcommand{\aboutMeFrame}{
	\begin{frame}{Über mich}
		\myname \\
		Informatik, 9. Fachsemester (Bachelor)
		% Lebensgeschichte...
		% Stammbaum...
		% Aufarbeitung der eigenen Todesser-Vergangenheit...
	\end{frame}
}

\def\thisyear{2019}

% Update date of exam
\def\myKlausurtermin{18.~März~2020, 14:00–16:00~Uhr}

\def\mydate#1{
		  \ifnum#1=1\relax	  23. Oktober \thisyear \
	\else \ifnum#1=2\relax	  30. Oktober \thisyear \
	\else \ifnum#1=3\relax    06. November \thisyear \
	\else \ifnum#1=4\relax    13. November \thisyear \
	\else \ifnum#1=5\relax    20. November \thisyear \
	\else \ifnum#1=6\relax    27. November \thisyear \
	\else \ifnum#1=7\relax    04. Dezember \thisyear \
	\else \ifnum#1=8\relax    11. Dezember \thisyear \
	\else \ifnum#1=9\relax    18. Dezember \thisyear \
	\else \ifnum#1=10\relax   08. Januar \nextyear \
	\else \ifnum#1=11\relax   15. Januar \nextyear \
	\else \ifnum#1=12\relax   22. Januar \nextyear \
	\else \ifnum#1=13\relax   29. Januar \nextyear \
	\else \ifnum#1=14\relax   05. Februar \nextyear \
	\else \textbf{Datum undefiniert!} 
	\fi\fi\fi\fi\fi\fi\fi\fi\fi\fi\fi\fi\fi\fi
}

\def\mylasttimestext{Was letztes Mal geschah...}

\colorlet{beamerlightred}{red!40}
\colorlet{beamerlightgreen}{green!50}
\colorlet{beamerlightyellow}{yellow!50}
\colorlet{lightred}{red!30}
\colorlet{lightgreen}{green!40}
\colorlet{lightyellow}{yellow!50}
\colorlet{fullred}{red!60}
\colorlet{fullgreen}{green}

\definecolor{myalertcolor}{rgb}{1,0.33,0.24}
\setbeamercolor{alerted text}{fg=myalertcolor}

% Flag to toggle display of KIT Logo.
% If you want to conform to the official logo guidelines, 
% you are not allowed to use the logo and should disable it
% using the following flag. Just saying.
% (But it's too beautiful, so best leave this commented. :P)
%\newcommand{\noKITLogo}{}

% Toggle handout mode by including the following line before including PraeambelTut
% and removing the % at the start (but do NOT remove the % char here, otherwise handout mode will always be on!)
% Please keep handout mode off in all commits!

% \newcommand{\handout}{}