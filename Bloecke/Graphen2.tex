\section{Repräsentation von Graphen}
\begin{frame}{Darstellung von Graphen}
	\begin{block}{Auf Papier}
		\begin{itemize}
			\item Graphische Darstellung
			\item Mengendarstellung
			\item Textuelle Beschreibung
		\end{itemize}
	\end{block}

	\pause
	\begin{block}{Im Rechner}
		Systematisches abspeichern der Kanten notwendig.\\
		Knoten werden oftmals implizit verwendet.
		\begin{itemize}
			\item (Kantenliste)
			\item Adjazenzlisten
			\item Adjazenzmatrix
		\end{itemize}
	\end{block}
\end{frame}

\begin{frame}{Adjazenzlisten}
	\begin{Definition}
		In einer \emph{Adjazenzliste} werden zu einem Knoten $x$ alle Knoten eingetragen, die von $x$ aus direkt mit einer Kante verbunden sind.
	\end{Definition}

	\pause
	Für jeden Knoten existiert eine Liste, alle Listen werden meist in einem Feld gespeichert.
\end{frame}

\begin{frame}{Adjazenzlisten}
	\begin{Beispiel}
		\begin{columns}
			\column{0.4\linewidth}
			\begin{tikzpicture}[->,>=stealth,baseline=-5mm]
		        \matrix[matrix of math nodes,nodes={draw,circle,minimum size=5mm,inner sep=2pt},row sep=10mm,column sep=10mm,ampersand replacement=\&]
		        {
		          |(0)| 0 \& |(1)| 1 \& |(2)| 2 \\
		          \& |(3)| 3 \& \\
		        };
		        \draw  (0) -- (1);
		        \draw  (0) -- (3);
		        \draw  (2)  to [bend left] (3);
		        \draw  (2) -- (1);
		        \draw  (3) to [bend left] (2);
		        \draw  (0) to [bend left]  (2);
		        \path (2) edge [loop right] ();
		        \draw (1) -- (3);
	      	\end{tikzpicture}
			\column{0.4\linewidth}
				\pause
				Für die Adjazentenlisten gilt 
				\begin{table}[H]
					\centering \vspace*{1em}
					\begin{tabular}{c|c} 0 & 1,2,3 \\ \hline 1 & 3 \\ \hline 2 & 1,2,3 \\ \hline 3 & 2 
					\end{tabular}  
				\end{table}
		\end{columns}
	\end{Beispiel}
\end{frame}

\begin{frame}{Adjazenzmatrix}
	\begin{Definition}
		Die \textbf{Adjazenzmatrix} eines Graphen $(V, E)$ mit $n$ Knoten ist die Matrix $A\in \{0,1\}^{n\times n}$ mit $$ A_{ij} = \begin{cases} 0 & (i,j) \notin E \\ 1 & (i,j) \in E \end{cases} $$ 
	\end{Definition}

	\medskip
	\emph{Achtung}: Bei dieser Definition müssen Matrix- und Knotenindizes mit dem gleichen Wert starten ($0$ oder $1$)
\end{frame}

\begin{frame}{Adjazenzmatrix}
	\begin{Beispiel}
		\begin{columns}
			\column{0.4\linewidth}
			\begin{tikzpicture}[->,>=stealth,baseline=-5mm]
			\matrix[matrix of math nodes,nodes={draw,circle,minimum size=5mm,inner sep=2pt},row sep=10mm,column sep=10mm,ampersand replacement=\&]
			{
				|(0)| 0 \& |(1)| 1 \& |(2)| 2 \\
				\& |(3)| 3 \& \\
			};
			\draw  (0) -- (1);
			\draw  (0) -- (3);
			\draw (1) -- (3);
			\draw  (2)  to [bend left] (3);
			\draw  (2) -- (1);
			\draw  (3) to [bend left] (2);
			\draw  (0) to [bend left]  (2);
			\path (2) edge [loop right] ();
			\end{tikzpicture}
			\column{0.4\linewidth}
			\pause
			$$ A = \begin{pmatrix} 0 & 1 & 1 & 1 \\ 0 & 0 & 0 & 1 \\ 0 & 1 & 1 & 1 \\ 0 & 0 & 1 & 0 \end{pmatrix} $$
		\end{columns}
	\end{Beispiel}
\end{frame}

\begin{frame}{Adjazenzmatrix}
	\begin{block}{Besondere Eigenschaften der Adjazenzmatrix}
		\begin{itemize}
			\item Schlingen lassen sich an einer $1$ auf der Diagonalen erkennen (Wert von $A_{ii}$) 
			\item Bei ungerichteten Graphen ist $A$ immer symmetrisch (also $A_{ij} = A_{ji}$).
		\end{itemize}
	\end{block}
\end{frame}

\begin{frame}{Exkurs: Vergleich der Darstellungen}
	\begin{itemize}[<+->]
		\item Adjazenzliste: Speicherplatz abhängig von Kanten $(m)$\\
		Besser bei dünn besetzten Graphen. 
		\item Adjazenzmatrix: Immer gleich viel Speicherplatz $(n^2)$\\
		Besser bei dicht besetzten Graphen (kein Overhead für Listen nötig).
	\end{itemize}
\end{frame}


\begin{frame}{Exkurs: Vergleich der Darstellungen}
	\begin{itemize}[<+->]
		\item Adjazenzliste: Nachbarn ermitteln in $O(1)$\\
		Ermitteln ob $(i,j)$ adjazent sind in $O(n)$
		\item Adjazenzmatrix: Nachbarn ermitteln in $O(n)$\\
		Ermitteln ob $(i,j)$ adjazent sind in $O(1)$
	\end{itemize}

	\pause
	In der Praxis meistens (Varianten von) Adjazenzlisten verwendet.\\
	Denn: Die meisten Graphalgorithmen traversieren den Graphen, dafür sind Adjazenzlisten deutlich besser.
	
	\bigskip
	Viel mehr dazu in Algorithmen I
\end{frame}
