\section{Formale Sprachen}

\begin{frame}{Beispiel}
	Sei $A = \{a,b\}$ ein Alphabet. Mit $L$ wollen wir alle Wörter beschreiben, die genau ein $b$ enthalten. \\ \pause
	$$ L = \{w_1 b w_2 \mid w_1, w_2 \in \{a\}^\ast \}  \qquad  L = \{a\}^\ast \cdot \{b \} \cdot \{a\}^\ast$$
	
	Was ist $L^3$? Was enthält $L^i$? \pause
	Zum Beispiel ist $$aaababaaaabaa = aaaba \ baa \ aabaa \in L_3$$ \pause
	$L^i$ enthält alle Wörter, die genau $i$-mal ein $b$ enthalten! \\[1em]
	
	Was enthält $$L^i \setminus \{b\}^\ast$$ \pause
	Alle Wörter, die aus $i$ $b$'s bestehen, aber auch noch mindestens ein $a$ enthalten. \\
\end{frame}

\begin{frame}{Aufgabe}
	Welche Eigenschaft muss eine formale Sprache $L$ über einem Alphabet
	$A$ erfüllen, damit gilt: $$ L^0 \subseteq L^1 \subseteq L^2 \subseteq L^3 \subseteq ... $$
	
	\pause
	\begin{block}{Lösung}
		Das gilt, wenn $$ \varepsilon \in L $$
	\end{block}
	
\end{frame}

\subsubsection{A1}
\begin{frame}{Aufgabe (Klausur) \stars{4}}
		\begin{itemize}
			\item Wiederlegen Sie: Für alle formalen Sprachen $L_1 , L_2$ gilt: 
			$$L_1^\ast \cup L_2^\ast = (L_1 \cup L_2 )^\ast$$
			
			\item Zeigen Sie: Für alle formalen Sprachen $L$ gilt: 
				$$L^\ast \cdot L = L^+ $$ 
	\end{itemize}

	Tipp zu 2) (nicht in der Klausur gegeben): Hier handelt es sich um eine Mengengleichheit, also argumentieren wir mit \enquote{$\subseteq$} und \enquote{$\supseteq$}
\end{frame}

\begin{frame}{Lösung}
	\textit{Für alle formalen Sprachen $L_1 , L_2$ gilt: 
		$$L_1^\ast \cup L_2^\ast = (L_1 \cup L_2 )^\ast$$ } \\[2em] \pause
	Diese Aussage ist falsch: Sei $L_1 = \{a\}$ und $L_2 = \{b\}$. Dann liegt \code{ab} in $(L_1 \cup L_2 )^\ast = \{a, b\}^\ast$ aber nicht in $L_1^\ast \cup L_2^\ast = \{a\}^\ast \cup \{b\}^\ast$.
\end{frame}

\begin{frame}{Lösung}
	\textit{Für alle formalen Sprachen $L$ gilt: 
		$$L^\ast \cdot L = L^+ $$ } \\[1em] \pause
	Diese Aussage ist wahr! 
	\begin{block}{1. Schritt: $L^\ast \cdot L \subseteq L^+$:} \pause
	Wenn $w \in L^\ast \cdot L$ liegt, dann lässt es sich in Teilwörter auftrennen $$ w = w_1 \cdot w_2$$ mit $w_1 \in L^\ast$ und $w_2 \in L$. Für $w_1$ existiert ein $i \in \N_0$ mit $w_1 \in L^i$. Also $$w = w_1 w_2 \in L^i \cdot L = L^{i+1} \subset L^+$$
	\end{block}
\end{frame}

\begin{frame}{Lösung}
	\textit{Für alle formalen Sprachen $L$ gilt: 
		$$L^\ast \cdot L = L^+ $$ } \\[1em] 
	Diese Aussage ist wahr! 
	\begin{block}{2. Schritt: $L^\ast \cdot L \supseteq L^+$:} \pause
	Wähle nun $w \in L^+$. Dann existiert ein $i \in \N_+$ mit $w \in L^i$. Da $i > 0$ lässt es sich schreiben als $i = j + 1$ für ein $j \in \N_0$. Also ist $$w \in L^{j+1} = L^j \cdot L \subset L^\ast \cdot L$$
	\end{block}
\end{frame}

\subsubsection{A2}
\begin{frame}{Aufgabe (WS 2008) \stars{3}}
	Es sei $A = \{a, b\}$. Die Sprache $L \subset A^\ast$ sei definiert durch $$L = (\{a\}^\ast \cdot \{b\} \cdot \{a\}^\ast)^\ast$$
	Zeigen Sie, dass jedes Wort $w$ aus $\{a, b\}^\ast$, das mindestens einmal das Zeichen
	$b$ enthält, in $L$ liegt. (Hinweis: Führen Sie eine Induktion über die Anzahl der
	Vorkommen des Zeichens $b$ in $w$ durch.)
\end{frame}

\begin{frame}{Lösung}
	$$L = (\{a\}^\ast \cdot \{b\} \cdot \{a\}^\ast)^\ast$$
	Sei $k$ die Anzahl der Vorkommen von $b$ in einem Wort $w \in \{a, b\}^\ast$.
	\begin{block}{Induktionsanfang}  \pause
		Für $k = 1$: In diesem Fall lässt sich das Wort $w$ aufteilen in $$w = w_1 \cdot b \cdot w_2$$ wobei $w_1$ und $w_2$ keine $b$ enthalten und somit in $\{a\}^\ast$ liegen. Damit gilt $w \in \{a\}^\ast \cdot \{b\} \cdot \{a\}^\ast$ und somit auch $$w \in (\{a\}^\ast \cdot \{b\} \cdot \{a\}^\ast)^\ast = L$$
	\end{block}
\end{frame}

\begin{frame}{Lösung}
	\begin{block}{Induktionsannahme}  \pause
		Für ein festes $k \in \N$ gilt, dass alle Wörter über $\{a, b\}^\ast$, die genau $k$-mal das Zeichen $b$ enthalten, in $L$ liegen.
	\end{block} \pause
	\begin{block}{Induktionsschritt}  \pause
		Wir betrachten ein Wort $w$, das genau $k + 1$ mal das Zeichen $b$ enthält. Dann kann man $w$ zerlegen in $w$ = $w_1 \cdot w_2$, wobei $w_1$ genau einmal das Zeichen $b$ enthält und $w_2$ genau $k$-mal das Zeichen $b$. \pause Nach Induktionsanfang liegt $w_1$ in $\{a\}^\ast \{b\}\{a\}^\ast$. Nach Induktionsvoraussetzung liegt $w_2$ in $(\{a\}^\ast \{b\}\{a\}^\ast )^\ast$, was bedeutet, dass $w = w_1 \cdot w_2$ in $$(\{a\}^\ast \{b\}\{a\}^\ast )(\{a\}^\ast \{b\}\{a\}^\ast )^\ast \subseteq (\{a\}^\ast \{b\}\{a\}^\ast )^\ast = L$$ liegt und die Behauptung ist gezeigt.
	\end{block}
\end{frame}

%\subsubsection{A3}
%\begin{frame}
%	\frametitle{Noch mehr Aufgaben}
%	Begründen oder widerlegen Sie:
%	\begin{itemize}
%		\item Für alle formalen Sprachen $L$ gilt: 
%		$$(L_1^\ast \cdot L_2^\ast)^\ast = (L_1 \cdot L_2)^\ast$$ 
%		
%		\item Für alle formalen Sprachen $L_1 , L_2$ gilt: 
%		$$(L_1^\ast \cup L_2^\ast)^\ast = (L_1 \cup L_2 )^\ast$$
%	\end{itemize}
%\end{frame}
%
%\begin{frame}
%	\frametitle{Lösung}
%	\textit{Für alle formalen Sprachen $L_1 , L_2$ gilt: 
%		$$(L_1^\ast \cdot L_2^\ast )^\ast = (L_1 \cdot L_2 )^\ast$$ } \\[2em] \pause
%	Diese Aussage ist falsch: Sei $L_1 = \{a\}$ und $L_2 = \{b\}$. Dann liegt $$\mathbf{aa} = \mathbf{aa} \cdot \varepsilon$$ in $(L_1^\ast \cdot L_2^\ast ) = (L_1^\ast \cdot L_2^\ast )^1 \subset (L_1^\ast \cdot L_2^\ast )^\ast$, aber nicht in $(L_1 \cdot L_2 )^\ast = \{ab\}^\ast$.
%	
%\end{frame}
%
%\begin{frame}
%	\frametitle{Lösung}
%	\textit{Für alle formalen Sprachen $L_1 , L_2$ gilt: 
%		$$(L_1^\ast \cup L_2^\ast )^\ast = (L_1 \cup L_2 )^\ast$$ } \\[2em] \pause
%	Die Aussage ist korrekt: Sei $w$ ein Wort aus $(L_1^\ast \cup L_2^\ast )^\ast$. Dieses lässt sich in Teilwörter $w_1 , \cdots , w_k$ unterteilen, so dass für $1 \leq i \leq k$ gilt: 
%	$$w_i \in (L_1^\ast \cup L_2^\ast ) \implies w_i \in L_1^\ast \text{ oder } w_i \in L_2^\ast$$
%	Diese Teilwörter $w_i$ lassen sich wieder in Teilwörter $w_{i_1}, \dots w_{i_s}$ zerlegen, die entweder aus $L_1$ kommen, wenn $w_i \in L_1^\ast$ liegt, oder in $L_2$ liegen, wenn $w_i \in L_2^\ast$ liegt. Damit lässt sich $w$ in Teilwörter $w_{i_j}$ aus $L_1 \cup L_2$ unterteilen und es folgt $w \in (L_1 \cup L_2 )^\ast$. 
%\end{frame}
%
%\begin{frame}
%	\frametitle{Lösung}
%	\textit{Für alle formalen Sprachen $L_1 , L_2$ gilt: 
%		$$(L_1^\ast \cup L_2^\ast )^\ast = (L_1 \cup L_2 )^\ast$$ } \\[2em]
%	Sei umgekehrt ein Wort $w$ aus $(L_1 \cup L_2 )^\ast$. Dieses lässt sich dann in Teilwörter $w_1, \dots, w_k$ unterteilen, so dass für $1 \leq i \leq k$ gilt 
%	$$w_i \in L_1 \cup L_2 \implies w_i \in L_1 \subset L_1^\ast \text{ oder } w_i \in L_2 \subset L_2^\ast$$
%	Somit lässt sich $w$ in Teilwörter aus $L_1^\ast \cup L_2^\ast$ unterteilen, und es folgt $w \in (L_1^\ast \cup L_2^\ast )^\ast$.
%\end{frame}