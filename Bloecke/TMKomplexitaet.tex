\section{Komplexität}

\subsection{Unentscheidbare Probleme}
\begin{frame}{Unentscheidbare Probleme}
	Es existieren Probleme, die von keiner Turingmaschine entschieden werden können.\\ \pause
	\textbf{Erinnerung}: Entscheidbar heißt, dass es eine Turingmaschine gibt, die für jede Eingabe hält und entscheiden kann, ob das Wort in der Sprache liegt oder nicht. Statt \textbf{Sprachen} spricht man auch gerne von \textbf{Problemen}.\\ \pause
	
	\bigskip
	\textbf{Achtung}: Unentscheidbar meint wirklich \enquote{hält für manche Eingaben nie} und nicht \enquote{braucht $2^{2^{2^{n}}}$ Zeit}.
\end{frame}

\begin{frame}{Das Halteproblem}
	Im Folgenden: Turingmaschine $\equiv$ Java-Programm. {\small (Beide können genauso viel.)}\\ \pause
	\bigskip
	Als „Halteproblem“ bezeichnen wir salopp die Sprache aller haltenden Java-Programme.
	\begin{Definition} 
		\vspace{-1.5\baselineskip}
		\begin{align*}
		\text{Halteproblem } H &:= \set{w \in \set{\word 0, \word 1}^* \Mid w \text{ \textbf{beschreibt Turingmaschine}, die hält!}} \\
		&\:\approx \set{w \in \text{Java-Programme} \Mid w \text{ hält}} \\
		&\:\approx \set{w \in \text{C++-Programme} \Mid w \text{ hält}}
		\end{align*}
		Diese Definition ist ungenau, damit es nicht zu kompliziert wird. \textbf{Die offizielle Definition aus der VL ist allerdings wichtig!}
	\end{Definition}
\end{frame}

\mycomment{ % Viel zu kompliziert zum Vorziehen und sowieso keine Zeit dafür gehabt.
	% Falls irgendwann mal Zeit dafür sein sollte, dann müsste man das ganze eher visuell angehen oder wirklich die TGI-Version nehmen.
	\begin{frame}{Codierung von Turingmaschinen}
		In der VL: Codierung einer Turingmaschine als Wort über dem Alphabet \nolinebreak $\{0,1,[,]\} \qquad$ (Gödelisierung)
		\bigskip
		
		\begin{Satz}
			Es existiert eine universelle Turingmaschine $U$, die für zwei Eingaben $[\mathtt w_1][\mathtt w_2]$ 
			\begin{itemize}
				\item überprüft ob $\mathtt w_1$ eine Turingmaschine $T$ codiert
				\item falls ja, die Eingabe $\mathtt w_2$ auf dieser Turingmaschine simuliert
				\item Das Ergebnis davon berechnet (falls $T$ hält)
			\end{itemize}
		\end{Satz}
	\end{frame}
	
	\begin{frame}{Halteproblem}
		\begin{Satz}
			Es ist nicht möglich, eine Turingmaschine $H$ zu bauen, die für jede Turingmaschine $T$ und jede Eingabe $w$ entscheidet, ob $T$ bei der Eingabe von $w$ hält.
		\end{Satz}
	\end{frame}
	
	\begin{frame}{Beweis}
		Sei eine Tabelle $x_i, f_j$ gegeben, wobei die $x_i$ alle Codierungen einer Turingmaschine sind und die $f_j$ die berechneten Funktionen der Turingmaschine $T_j$ sind. Sei jetzt $H$ eine TM, die das Halteproblem löst und $G$ eine Maschine, die \begin{itemize}
			\item Wenn $H$ mitteilt, dass $T_{x_i} ( x_i )$ hält, dann geht $G$ in eine Endlosschleife.
			\item Wenn $H$ mitteilt, dass $T_{x_i} ( x_i )$ nicht hält, dann hält $G$
		\end{itemize}
		Jede mögliche Turingmaschine $T_{x_i}$ verhält sich also für die Eingabe $x_i$ genau anders wie $G$. Also ist $G$ eine Turingmaschine, die nicht in der Tabelle liegt, aber in der Tabelle sind alle TMs enthalten, da diese ja abzählbar sind. Widerspruch!
	\end{frame}
	
	\begin{frame}{Beweis 2}
		Sei wieder $H$ eine TM, die das Halteproblem löst und $G$ eine Maschine, die \begin{itemize}
			\item Wenn $H$ mitteilt, dass $T_{x_i} ( x_i )$ hält, dann geht $G$ in eine Endlosschleife.
			\item Wenn $H$ mitteilt, dass $T_{x_i} ( x_i )$ nicht hält, dann hält $G$
		\end{itemize}
		Also gilt: $$ G \text{ hält für Eingabe } w \iff T_w \text{ hält nicht für Eingabe } w $$
		Setzen wir nun für $w$ die Codierung von $G$ ein, so erhalten wir:
		$$ G \text{ hält für Eingabe } w \iff G \text{ hält nicht für Eingabe } w $$
		Widerspruch!
	\end{frame}
}

\begin{frame}{Das Halteproblem}
	\begin{Satz}
		Das Halteproblem ist unentscheidbar. \\
		Beweis: VL.
	\end{Satz}
	Das heißt, es kann \textbf{keine Turingmaschine} gebaut werden, die für jedes Java-Programm sagen kann, ob es hält oder nicht und dabei \textbf{selbst immer eine Antwort liefert}, also hält. \\ 
	\medskip \pause
	Es kann auch kein Java-Programm gebaut werden, welches das tut.
\end{frame}

\begin{headframe}
	\alert{\Huge Das HALTEPROBLEM ist unentscheidbar!}
\end{headframe}

\begin{headframe}
	\alert{\Huge Es gibt Probleme, die mit dem Rechner NICHT gelöst werden können!}
\end{headframe}

\begin{frame}{Fleißige Biber}
	\begin{Definition}
		Eine Bibermaschine ist eine Turingmaschine, die $n+1$ Zustände hat (wobei ein Anfangszustand und ein Haltezustand darunter sind) und die nur \word 1en produzieren kann.
	\end{Definition}
	\pause
	
	\begin{Definition}
		Als Busy-Beaver-Funktion $\bb(n)$ wird die maximale Anzahl an \word 1en bezeichnet, die ein Biber mit $n+1$ Zuständen auf dem Band hinterlassen kann. \\
		Ein Biber, der diese Zahl schafft, heißt auch \emph{fleißiger} Biber.
	\end{Definition} 
	
	\begin{Satz}
		Die Busy-Beaver-Funktion ist nicht berechenbar (es gibt keine Turingmaschine, die die Funktionswerte als Ausgabe liefert).
	\end{Satz}
\end{frame}

\begin{frame}[plain] \large \bf \centering
	Wir betrachten für die kommenden Folien nur Turingmaschinen, die bei jedem Eingabewort halten!
\end{frame}

\subsection{Berechnungskomplexität}
\begin{frame}{Zeitkomplexität}
	\textbf{Schaut euch das auch nochmal in der VL an!}
	\begin{Definition}
		Die Zeitkomplexität Time$(n)$ einer Turingmaschine ist die maximale Anzahl an Schritten, die eine Turingmaschine bei Eingabe eines Worts der Länge $n$ benötigen kann (worst-case).
	\end{Definition}
	\pause
	\textbf{Beispiel:} Überprüfung auf Palindrom:
	\begin{itemize}[<+->]
		\item Erstes Symbol \?> letztes Symbol ($n$) 
		\item Zurück zum ersten ($n$)
		\item Mit kürzerem Wort wiederholen ($n-2$)
	\end{itemize} \pause
	Also insgesamt $$\|Time|(n) \leq 2n + 1 + \|Time|(n-2)$$ 
	Daraus folgern wir $$\|Time|(n) \in O(n^2)$$
\end{frame}

\begin{frame}{Platzkomplexität}
	\begin{Definition}
		Die Platzkomplexität Space$(n)$ einer Turingmaschine ist die maximale Anzahl an Feldern, die eine Turingmaschine bei Eingabe eines Worts der Länge $n$ benötigen kann (worst-case). Benötigt wird ein Feld, wenn es vom Schreibkopf besucht wird oder von der Eingabe belegt wurde.
	\end{Definition}
	\pause
	\textbf{Beispiel:} Überprüfung auf Palindrom:
	\begin{itemize}[<+->]
		\item Erstes Symbol \?> letztes Symbol ($n + 1$) 
		\item Zurück zum ersten ($0$)
		\item Mit kürzerem Wort wiederholen ($0$)
	\end{itemize} \pause
	Also insgesamt $$\|Space|(n) = n+1 \in \Th{n}$$
\end{frame}

\begin{frame}{Platzkomplexität}
	Wie hängen Zeit- und Platzkomplexität zusammen?
	\begin{itemize}
		\item Wenn eine TM nur $n$ Schritte macht, kann sie auch nur $n$ Felder besuchen
		\item Aber anders herum: Auch mit $n$ Feldern können exponentiell viele Schritte durchgeführt werden.\\
			  Beispiel: Band mit $\word 0^n$. Solange inkrementieren (siehe Aufgabe), bis $\word 1^n$ auf dem Band steht. \pause Also $2^n - 1$ mal inkrementieren, somit mindestens $2^n - 1$ Schritte.
	\end{itemize}
\end{frame}

\begin{frame}
	\begin{Definition}
	\begin{itemize}
		\item \textbf{P} ist die Menge aller Entscheidungsprobleme, die von Turingmaschinen entschieden werden können, deren Zeitkomplexität polynomiell ist.
		\item \textbf{PSPACE} ist die Menge aller Entscheidungsprobleme, die von Turingmaschinen entschieden werden können, deren Raumkomplexität polynomiell ist.
	\end{itemize}
	\end{Definition}
	\pause
	Es ist \\ 
	\mbox{}\hphantom{Wir wissen nicht, ob auch \qquad }$\textbf{P} \subseteq \textbf{PSPACE}.$ \\
	\medskip\pause
	Die andere Richtung ist \textbf{ein großes offenes Problem}:\\
	Wir wissen nicht, ob auch \qquad $\textbf{P} \supseteq \PSPACE$. \\
	Wir haben zwar ein Beispiel für eine TM mit polynomiellen Platz und exponentieller Zeit gesehen, aber das \textbf{Problem} (alle $\word 0$ zu $\word 1$ umwandeln) hätte man deutlich effizienter lösen können.
\end{frame}

\subsection{Aufgabe}
\begin{frame}{Aufgabe} %(WS 2011)
	\only<1|handout:1>{	
		Die Turingmaschine $T$ mit Anfangszustand $S$ sei durch folgende Überführungsfunktion gegeben 
		\begin{table}[H]
		\centering
		\begin{tabular}{c|c|c|c|c}
		& $S$ & $S_a$ & $S_b$ & $R$ \\
		\hline
		$\word a$ & $(\word X,S_a,L)$ & $(\word a,S_a,L)$ & $(\word a,S_b,L)$ & $(\word a,R,R)$\\
		$\word b$ & $(\word X,S_b,L)$ & $(\word b,S_a,L)$ & $(\word b,S_b,L)$ & $(\word b,R,R)$\\
		$\word X$ & $(\word X,S,R)$ & $(\word X,S_a,L)$ & $(\word X,S_b,L)$ & $(\word X,S,R)$ \\
		$\square$ & --- & $(\word a,R,R)$ & $(\word b,R,R)$ & ---
		\end{tabular} 
		\end{table}
		\smallskip
		
		Tipp: Manchmal hilft es, die TM zu zeichnen.
	}

	\begin{itemize}
		\item Was steht bei der Eingabe des Wortes $w\in \{\word a,\word b\}^*$ am Ende der Berechnung auf dem Band? \\
			\only<2|handout:2>{\emph{Lösung}: Am Ende steht das Wort $R(w)\word X^{\vert w \vert}$ auf dem Band, wobei $R(w)$ das Spiegelbild von $w$ ist.}
		\item Welche Platzkomplexität hat $T$?\\
			\only<2|handout:2>{\emph{Lösung}: Eingabe der Länge $n$: Platzbedarf ist $2n+1$}
		\item Geben Sie eine einfache Funktion $f:\N_0 \functionto \N_0$ an, so dass die Zeitkomplexität von $T$ in $\Theta(f(n))$ liegt.\\
			\only<2|handout:2>{\emph{Lösung}: Eingabe der Länge $n$: Zeitbedarf in $\Theta(n^2)$ \\}
	\end{itemize}
\end{frame}



% TODO: Aufgaben!
