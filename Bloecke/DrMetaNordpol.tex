%% Zum Aufwärmen

\section{Eine Weihnachtsgeschichte}

\begin{frame}{Dr. Meta am Nordpol}
	Der ebenso geniale wie hochwissenschaftliche Superbösewicht Dr. Meta ist entsetzt.
	Seit Jahren sinken seine Bekanntheitswerte, mittlerweile kennt nur noch eine kleine Gruppe von Studenten an einer einzigen Universität seinen Namen. (Vielleicht hätte er seine Twitter-Kampagne aktiver verfolgen sollen – aber nein, die Übungsblätter in GBI zu infiltrieren war wichtiger!)\\
	\bigskip \pause
	Damit alle Welt von seinen Schurkentaten erfährt, verfolgt er einen ehrgeizigen Plan: Er möchte seine Memoiren an sämtliche Haushalte verteilen. Dies von Hand zu erledigen ist eines Superschurkens seiner Größe jedoch nicht würdig. Daher möchte er das effizienteste und flächendeckendste Logistiksystem der Welt verwenden:\\ 
	Den Weihnachtsmann! Statt Geschenken soll es dieses Jahr Angst und Schrecken unterm Tannenbaum geben.
\end{frame}

\begin{frame}{Dr. Meta am Nordpol}
	\only<1|handout:1>{
	Am Nordpol angekommen muss Dr. Meta leider feststellen, dass das Weihnachtsdorf nicht so leicht zu erreichen ist wie erhofft.\\
	Nach Tagen des Umherirrens gerät er an eine Hütte, von der zwei Wege fortführen. Vor der Hütte steht ein Schild:\\
	\bigskip
	}
	\only<1-3|handout:2>{
	%„Willkommen im Servicezentrum für unangemeldete Eindringlinge!\\
	„Gib acht, Eindringling. \\
	\smallskip
	}
	\only<2-3|handout:2>{
	Einer dieser beiden Wege führt direkt zum Weihnachtsdorf, der andere jedoch...“\\
	Nur der Wächterwichtel weiß, welcher Weg der richtige ist.\\
	\textbf{Streng abwechselnd sagt er an einem Tag die Wahrheit und am darauf folgenden Tag lügt er}.\\ 
	\smallskip
	Keiner außer ihm weiß, ob er heute die Wahrheit sagt oder aber heute lügt.\\
	}
	\visible<3|handout:2>{
	Ihr dürft dem Wichtel genau eine Frage stellen, danach müsst ihr euch für einen der Wege entscheiden.\\
	\bigskip
	Helft Dr. Meta, das Weihnachtsdorf zu erreichen, indem ihr die richtige Frage stellt.\\
	}
\end{frame}

\begin{frame}{Dr. Meta am Nordpol\textsuperscript{\thefootnote}}
	\stepcounter{footnote}\footnotetext{Rätselidee frei übernommen nach: Der Philosoph am Scheideweg} % Fuck you, LaTeX...
		Helft Dr. Meta, das Weihnachtsdorf zu erreichen, indem ihr die richtige Frage stellt. %Wie lautet die richtige Frage?
	\\[1em]
	\Impl „Was würdest du sagen, wenn ich dich morgen fragen würde, ob der linke Weg zum Weihnachtsdorf führt?“\\
	\bigskip
	Was antwortet der Wächterwichtel? \\
	\smallskip
	\begin{tabular}{|l|c|c|}
		\hline
		Richtiger Weg & heute lügt er & heute sagt er die Wahrheit \\
		\hhline{|=|=|=|}
		links & „Nein“ & „Nein“ \\
		\hline
		rechts & „Ja“ & „Ja“ \\
		\hline
	\end{tabular}
	
	%Und wenn der Philosoph jeden Morgen würfelt, ob er die Wahrheit sagt oder lügt?\\ 
	%\only<2|handout:2>{
	%Was würdest du sagen, wenn ich dich \textbf{heute} fragen würde, ob der linke Weg der richtige ist?\\ }
\end{frame}