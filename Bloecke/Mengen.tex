\section{Mengen}

\morescalingdelimiters

\begin{frame}{Mengen}
	\pause % -> HIER Sammeln, was Tutanden bereits wissen (an der Tafel)
	\begin{block}{Definition: Mengen} \vspace{-.5\baselineskip}
		\begin{itemize}
			\item \textbf{Menge} $M$: Ansammlung verschiedener Objekte
			\item $m \in M$: Objekt $m$ ist \textbf{Element} der Menge $M$
			\item Schreibweisen: \\
			$M = \{m_1, m_2, m_3 \}, \qquad M = \{m \mid m \geq 0\}$
			\item Reihenfolge der Aufzählung wurscht, Duplikate auch: \\
			$\{2,1,3,1,4\} = \{1,2,3,4\}$
			\item \textbf{Leere Menge} $\emptyset$: enthält keine Elemente, \quad also $\{\} = \emptyset$
		\end{itemize}
	\end{block}
	\pause
	
	\begin{block}{Beispiel}
		Wichtige Mengen sind \\ 
		\centered{$\Z, \Q, \R, \C, \qquad \N_+ = \{1, 2, 3, \dots\}, \quad \N_0 = \{0, 1, 2, 3, \dots\}$} 
		Es gilt:\\ 
		\centered{$-5 \in \Z\qquad -5 \notin \N_0$}
	\end{block}
	
\end{frame}

\begin{frame}{Anwendung}
	Wir haben ein „Universum“ an Dingen:\\ 
	$U = \{ $ Gurken, Obst, Fußball, Hockey, Äpfel, Maracujas, Tomaten, Bananen, Karotten, Wurzelpetersilie, Sport,  Orangen, Gemüse, Basketball, Ersti-Weitwurf $ \} $  \\[0.5em]
	
	\pause
	Ordnen wir diese in eine Teilmenge $C$ für Kategorien und jeweils eine Teilmenge $E_c$ für die Elemente einer Kategorie $c \in C$. \\ {\small (\textbf{Überhaupt} nicht konstruiert!)} \\[0.5em]
	
	\pause
	$C = \{$ Obst, Gemüse, Sport $\}$ \\[0.3em]
	$E_{\text{Obst}} = \{$ Äpfel, Orangen, Maracujas, Bananen $\}$ \\
	$E_{\text{Gemüse}} = \{$ Tomaten, Karotten, Gurken, Wurzelpetersilie $\}$ \\
	$E_{\text{Sport}} = \{$ Fußball, Basketball, Hockey, Ersti-Weitwurf $\}$ \\
	
	\vspace{.5\baselineskip}
	\Large \alert{
			\textbf{Achtung}: 
			$$\underbrace{\text{Obst}}_{\mathllap{\text{ein Bezeichner, keine Menge}}}
			 \neq 
			\underbrace{E_{\text{Obst}}}_{\mathrlap{\text{eine Menge}}}$$
	}
	
\end{frame}

\begin{frame}{Kardinalität, Teilmengen, Gleichheit}
	
	\begin{block}{Definitionen}  \vspace{-.4\baselineskip}
		\begin{itemize}
			\item \textbf{Kardinalität} $\setsize{M}$: \\
			Anzahl der Elemente einer endlichen Menge $M$ \\
			\pause
			\item $N \subseteq M$: \\
			Menge $N$ ist \textbf{Teilmenge} von $M$, also jedes Element aus $N$ auch in $M$ \\
			Es gilt: \qqquad $ N \subseteq M \iff \forall n \in N : n \in M$
			\pause
			\item $N = M$: \\
			Mengen $N$ und $M$ sind \textbf{gleich}, enthalten also die gleichen Elemente \\
			Es gilt: \qqquad $ N = M \iff N \subseteq M \ \text{und} \ M \subseteq N$
		\end{itemize}
	\end{block} 
\end{frame}

\begin{frame}{Kardinalität, Teilmengen, Gleichheit}
	
	\begin{block}{Beispiel}
		
		\begin{align*}
		\setsize{\{1,2,3,2,1\}} &= \only<2->{3} \\
		\setsize{ \emptyset } &= \only<3->{0} \\
		\{1,2\} \only<1-3|handout:0>{ &\mathrel{?\ }  } \only<4->{&\subsetneq}  \{1,2,3\} \\
		\{1,2\} \only<1-4|handout:0>{ &\mathrel{?\ } } \only<5->{&\nsubseteq} \{\text{Hund}, \text{Katze}, \text{Maus}\} \\
		\{1,2,3\} \only<1-5|handout:0>{ &\mathrel{?\ } } \only<6->{&\supseteq} \{1,2\}
		\end{align*}
	\end{block} 
	

\end{frame}

\begin{frame}{Schnitt und Vereinigung}
	
	\begin{block}{Definition}
		Seien $M$ und $N$ zwei Mengen. Dann heißen
		$$M \cap N := \left\{x \mid x \in M \text{ und } x \in N \right\} \qquad M \cup N := \left\{x \mid x \in M \text{ oder } x \in N \right\} $$
		\textbf{Durchschnitt} bzw. \textbf{Vereinigung} von $M$ und $N$.\\[1em] 
		\pause
		$M$ und $N$ heißen \textbf{disjunkt}, wenn sie keine gemeinsamen Elemente haben, also $M \cap N = \emptyset$ (der Schnitt ist leer).
	\end{block}
	
	\pause
	\begin{block}{Beispiel}
		\centered{$ \{1,2\} \cup \{2,3\} = \{1,2,3\} \qquad \{1,2\} \cap \{2,3\} = \{2\} $}
	\end{block}
	
	\pause
	
	\begin{block}{Lemma}
		Es gilt: \\ \centered{$\setsize{M \cup N} \ = \ \setsize{M} + \setsize{N} - \setsize{M \cap N}$}
	\end{block}
	
	
\end{frame}

\begin{frame}{Mengendifferenz}
	\begin{block}{Definition}
		Seien $A$ und $B$ zwei beliebige Mengen, so heißt $$ A\setminus B := \left\{ x \mid x\in A \text{ und } x\notin  B  \right\} $$ die \textbf{Differenz}(-menge) von $A$ und $B$.
	\end{block}
	
	\pause
	
	\begin{block}{Lemma}
		Es gilt: $$ N \subseteq M \iff N \setminus M = \emptyset $$
	\end{block} 
	
	\pause
	
	\begin{block}{Weitere Beispiele}
		\vspace{-2\baselineskip} % WtF LaTeX??
		\begin{align*}
		A \cup \emptyset \uncover<4->{ &= A }  \\
		A \cap \emptyset \uncover<5->{ &= \emptyset }\\
		\N_+ \cup \{0\} \uncover<6->{ &= \N_0} \\
		\N_0 \setminus \{0\} \uncover<7->{ &= \N_+} 
		\end{align*}
	\end{block}
	
\end{frame}

\begin{frame}[t]{Mengengleichheit: Beispiel}
	\begin{itemize}
		\item<1-> Sei $ A $ und $M$ beliebige Mengen. Zeigt, dass gilt 
		\begin{align*}
		A &=  \underbrace{ \left(A \setminus M \right)}_{T_1} \cup  \underbrace{ \left(A \cap M \right)}_{T_2}  
		\end{align*}		 
		\only<2-5>{
			\item<2-5> \textbf{Richtung}: $ A \subseteq T_1 \cup T_2 $ 
			\item<3-5> Wähle $x\in A$ und unterscheide:
			\item<4-5> Fall 1: Ist $x\in M$, so gilt $x\in A $ und $x\in M$ und damit $x\in A\cap M  = T_2 $
			\item<5-5> Fall 2 : Ist $x\notin M$, so gilt $x\in T_1$, da $T_1 = \left\{x \mid x \in A \text{ und } x\notin M \right\} $  }
		\only<6->{\item<6-> \textbf{Richtung}: $T_1 \cup T_2 \subseteq A $ 
			\item<7-> Wähle $x\in T_1 \cup T_2$. Dies bedeutet $x\in T_1 $ oder $x\in T_2$. 
			\item<8-> Fall 1: $x\in T_1$ (angenommen $T_1 \neq \emptyset$). Aus Definition folgt $x\in A$.
			\item<9-> Fall 2: $x\in T_2$ (angenommen $T_2 \neq \emptyset$). Somit $x\in A$ und $x\in M$. $\qed$}
	\end{itemize}
\end{frame}

\begin{frame}{Eine Menge Mengen...}
	\begin{block}{Aufgabe}
		Wir haben Mengen $A = \{1, 2\}, B = \{3\}, C = \{1, 3\}; A,B,C \subseteq M = \{1, 2, 3\}$.\\
		Dann ist:
		\begin{align*}
		A \cup B &= \visible<2->{ \{1, 2, 3\} }  \\
		A \cap C &= \only<3->{ \{1\} }\\
		A \setminus C &= \only<4->{ \{2\} }\\
		B \setminus A &= \only<5->{ \{3\} }\\
		A \cup (B \setminus C) &= \only<6->{ \{1, 2\} }\\
		C &= \{1, 3\} \\
		(A \setminus C) \cup B &= \only<7->{ \{2, 3\} }\\
		A \cap B &= \only<8->{ \emptyset }
		\end{align*}
	\end{block}
\end{frame}



\section{Potenzmengen}

\begin{frame}{... in einer Menge! (Potenzmengen)}
	\begin{block}{Definition}
		Die \textbf{Potenzmenge} $2^M$ oder auch $\Pot (M)$ ist die Menge aller möglicher Teilmengen von $M$. 
		\begin{align*}
			2^M = \left\{ N \mid N \subseteq M \right\}
		\end{align*}
	\end{block}
	\pause
	
	\begin{block}{Beispiel}
		Sei $M = \left\{ 1,2,0 \right\} $. \\ \pause
		
		Dann gilt  
		\begin{align*}   
		2^M &= \left\{ \emptyset, \{ 0 \}, \{ 1 \}, \{ 2 \}, \{ 0,1 \} , \{ 0,2 \}, \{ 1,2 \}, \{ 0,1,2 \} \right\}
		\end{align*}
		
		\textbf{Beachte}: Es gilt immer $M \in 2^M \ \text{und} \ \emptyset \in 2^M$.
	\end{block}
	
\end{frame}

\begin{frame}{Potenzmengen}
	\begin{block}{Aufgabe}
		Wie viele Elemente enthält $2^M$? \\[0.5em]
		
		\pause
		$2^{\setsize{M}}$
	\end{block}
	
	\pause
	\begin{block}{Noch ne Aufgabe}
		Gebt eine Abbildung $\phi \colon 2^{M} \functionto 2^{M}$ so an,
		dass für jedes $L \in 2^{M}$ und für jedes $w \in M$ gilt:
		\begin{equation*}
			w \in L \text{ genau dann, wenn } w \notin \phi(L).
		\end{equation*}
		
		\pause
		\begin{align*}
			\phi \colon 2^{M} &\functionto 2^{M},\\
			L &\mapsto M \setminus L.
		\end{align*}
	\end{block}
\end{frame}

\section{Paare}

\begin{frame}{Paare}
	\begin{block}{Definition}
		Seien $A$ und $B$ zwei Mengen und $a \in A$, $b \in B$.\\
		$$(a, b)$$ heißt \textbf{Paar} mit der ersten Komponente $a$ und der zweiten Komponente $b$.\\[1em]
		\pause
		Paare sind \textbf{keine Mengen, sondern was \alert{völlig anderes}!} \\
		In Paaren sind Duplikate \textbf{möglich} und die Reihenfolge der Elemente ist \textbf{wichtig}.\\
	\end{block}

	\pause
	\begin{block}{Beispiel}
		$$ (KI, T) = (KI, T) \qquad (KI, T) \neq (T, KI) \qquad (1, 1) \neq \underbrace{(1)}_{\mathclap{\text{überhaupt gar kein Paar!}}} $$
	\end{block}
\end{frame}

\begin{frame}
	\centering
	\Huge
	\alert{Runde und geschweifte Klammern 
		   $$\Big(\;\Big) \textbf{\quad vs. \quad} \Big\{\;\Big\}$$ \\
		   \textbf{NICHT VERWECHSELN!}}
\end{frame}

\begin{frame}{Paare}
	Das Konzept der Paare lässt sich auf das Konzept der Mengen zurückführen.
	
	\begin{block}{Aufgabe}
		Gegeben sei die Menge $M = \{m_1, m_2\}$.\\
		Wie kann man die Paare $(m_1, m_2)$ und $(m_2, m_1)$ eindeutig darstellen, allerdings nur unter Verwendung von Mengen und $m_1, m_2$?
	\end{block}
	\pause
	
	\begin{block}{Lösung}
		Wir definieren: \\ 
		$(m_1, m_2) := \left\lbrace m_1, m_2, \{m_1\}\right\rbrace$ und $(m_2, m_1) := \left\lbrace m_1, m_2, \{m_2\}\right\rbrace$
	\end{block}
\end{frame}