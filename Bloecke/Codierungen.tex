\section{Codierungen}

\begin{frame}{Übersetzungen}
	\textbf{Übersetzung:} „Bedeutungserhaltende“ Abbildung \\[0.5em] \pause
	\textbf{Codierung:} Injektive Übersetzung \\ \pause
	\begin{itemize}
		\item Injektiv reicht, weil wir jedem $f(w)$ sein erzeugendes $w$ zuordnen können
		\implitem Definiere Bedeutung von $f(w) := \text{Bedeutung von } w$
	\end{itemize}
	
	\pause
	\begin{itemize}
		\item \textbf{Problem}: \emph{Beliebige} Codierungen zu speichern sehr aufwendig
		\item $\abs{\text{Definitionsbereich}} = \infty$ \impl sogar unmöglich!
		\implitem Bringen wir etwas Struktur ins Spiel!
	\end{itemize}
\end{frame}

\begin{frame}{Homomorphismen}
	Ein \textbf{Homomorphismus} ist eine \textit{strukturerhaltende} Abbildung \\ \pause
	\begin{threealign}
		\Phi : A &\functionto& B \\
		\text{ mit } \forall a \in A, b \in B: \quad  \Phi(a \sim b) &=& \Phi(a) \bowtie \Phi(b)
	\end{threealign}
	\medskip \\
	Dabei stehen $\sim$ und $\bowtie$ hier für feste beliebige Operationen, wie z.~B. $+, -, \·, /, \oplus, \barwedge, ...$ \\
	(Muss es auf $A$ bzw. $B$ natürlich geben!)
\end{frame}

\begin{frame}{Homomorphismen}
	In GBI: Homomorphismen auf Wörtern mit „$\·$“ als Operation.\\
	\impl $A, B$ Alphabete, dann ist $h: A^* \functionto B^*$ ein Homomorphismus, wenn
	$$ \forall x, y \in A^* : \quad h(x \cdot y) = h(x) \cdot h(y). $$
	
	\pause
	\impl Brauchen nur Funktionswerte für einzelne Zeichen $a \in A$, dann kennen wir schon das ganze $h$! \\
	\impl Brauchen uns nur eine Tabelle für die Zeichen zu merken \\
	\impl $A$ endlich \impl Tabelle endlich \impl gut speicherbar
	
	\pause
	\begin{Beispiel}
		Sei $h$ ein Homomorphismus, für den gilt: $h(\word a) = \word 2, h(\word b) = \word 3$. \\
		Dann gilt $h(\word{aba}) = h(\word a) \cdot h(\word b) \cdot h(\word a) = \word{232}. $ \\[0.5em]
		$\fTrans_{16,2}$ ist fast ein Homomorphismus (führende Nullen machen Probleme)\\
		$\fTrans_{3,2}$ ist kein Homomorphismus (siehe dazu ÜB WS15/16)
	\end{Beispiel}
\end{frame}

\begin{frame}{Homomorphismen bauen}
	Haben: Abbildung der einzelnen Zeichen \\
	Wollen: Abbildung ganzer Wörter 
	\begin{Definition}
		Sei $f: A \functionto B^*,$ \pause definiere $f^{**}:A^* \functionto B^*$ als
		\begin{align*}
		f^{**}(\eps) &= \eps  \\
		\forall w\in A^*, x\in A: \quad  f^{**}(w \· x) &= f^{**}(w) \· f(x)       
		\end{align*}
	\end{Definition}

	$f^{**}$ heißt der durch $f$ \textbf{induzierte} Homomorphismus.
\end{frame}

\begin{frame}{$\eps$-Freiheit}
	Klar ist: Für jeden Hom. $h$ ist $h(\eps) = \eps$.
	
	\pause
	\begin{Definition}
		Ein Homomorphismus heißt $\eps$-frei, wenn 	$$ \forall x\in A : h(x) \neq \eps. $$
	\end{Definition}

	
	Was ist das Problem mit Homomorphismen, die nicht $\eps$-frei sind? \\ \pause
	\impl Es geht Information verloren.\\
	
	\begin{Beispiel}
		Sei $h$ ein Hom. mit $h(\word c) = \eps, \; h(\word b) = \word 2$. \\
		Haben unbekanntes $w \in \{\word b, \word c\}^*$ und wissen: $h(w) = \word 2$. \\
		\smallskip
		Wie kommen wir von $h(w)$ wieder zu $w$ zurück? \\ \pause
		\impl Gar nicht! (Wissen nicht, wieviel \word c in $w$ drin sind.)
	\end{Beispiel}
\end{frame}

\begin{frame}{Präfixfreiheit}
	\begin{Definition}
		Ein Homomorphismus heißt \textbf{präfixfrei}, wenn für
		\emph{keine} zwei verschiedenen Symbole $x_1,x_2\in A$ gilt: $h(x_1)$
		ist ein Präfix von $h(x_2)$.
	\end{Definition}

	\bigskip
	Was ist das Problem mit Homomorphismen, die nicht präfixfrei sind? \\ \pause
	\impl Es geht Information verloren.\\
	
	\begin{Beispiel}
		Sei $h$ ein Hom. mit $h(\word a) = \word 2, \; h(\word b) = \word 3, \; h(\word c) = \word{23}$. \\ 
		Haben unbek. $w \in \{\word a, \word b, \word c\}^*$ und wissen: $h(w) = \word{23}$. \\
		\smallskip
		Wie kommen wir von $h(w)$ wieder zu $w$ zurück?\\ \pause
		\impl Gar nicht! $w = \word c$ oder $w = \word{ab}$, wir wissen es nicht!
	\end{Beispiel}
	
\end{frame}

\begin{frame}{Zurück zu Codierungen}
	\begin{block}{Beobachtung}
		Präfixfreie Homomorphismen sind $\eps$-frei.
	\end{block}

	\pause
	\begin{block}{Lemma}
		Präfixfreie Homomorphismen sind Codierungen (also injektiv).
	\end{block}

	\pause
	\begin{block}{Beobachtung}
		Präfixfreie Codes kann man \enquote{einfach} decodieren:
		\[
		u(w) = 
		\begin{cases}
		\eps, & \text{ falls } w=\eps\\
		x\·u(w'), & \text{ falls } w=h(x) \· w' \text{ für ein } x\in A \\
		\bot,  & \text{ sonst }\\
		\end{cases}
		\]
	\end{block}
	($\bot$: „undefiniert“, „bottom“.)
	
\end{frame}