\section{Formale Sprachen}

\begin{frame}
	\frametitle{Rückblick}
	Sei $\Sigma = \{A, B, ..., Z, a, b, ..., z\}$ ein Alphabet.\\
	\pause
	Dann enthält $\Sigma^*$ alle Wörter, die man mit Zeichen aus $\Sigma$ bilden kann. Aber nicht jedes dieser Wörter ist auch sinnvoll.\\[1em]
	
	\enquote{egnarts si efiL} ist kein sinnvolles Wort. \pause Oder? \\[1em]
	\pause
	Wie wir sehen, hängt es immer vom Kontext ab, welche Wörter wir als (syntaktisch) korrekt betrachten.\\
	
\end{frame}

\begin{frame}
	\frametitle{Formale Sprachen}
		\begin{Definition}
			Sei $\Sigma$ ein Alphabet.\\
			Eine formale Sprache $L$ ist eine Teilmenge von $\Sigma^*$.
		\end{Definition}
		\pause
		\vspace{10pt}
		Durch die formale Sprache geben wir an, welche der möglichen Wörter wir als  \emph{syntaktisch korrekt} ansehen.\\
		\pause
		Formale Sprachen werden häufig nicht direkt, sondern über Bildungsvorschriften angegeben.
		
		\pause
		\begin{Beispiel}
			$\Sigma = \{0, 1\}$ \\
			$L = \{ \omega \in \Sigma^* \mid \omega \text{ endet auf } 10 \}  = \{10, 010, 110, 0010, ...\}$
		\end{Beispiel}
\end{frame}

\begin{frame}
	\frametitle{Formale Sprachen}
	
	\begin{Beispiel}
		Formale Sprache $L$ aller Wörter über $A=\{\literal{a},\literal{b}\}$, in denen nirgends das Teilwort $\literal{ab}$ vorkommt.
		\begin{itemize}
			\pause
			\item $L=\{\literal{a},\literal{b}\}^*
			\setminus \{w_1 \cdot \literal{ab} \cdot w_2 \mid w_1,w_2\in
			\{\literal{a},\literal{b}\}^*\}$
			
			\pause
			\item Erst ein beliebiges Wort (evtl. $\varepsilon$) nur aus $\literal{b}$,\\
			danach ein beliebiges Wort (evtl. $\varepsilon$) nur aus $\literal{a}$.
			
			\pause
			\item $L=\{w_1w_2 \mid w_1\in 
			\{\literal{b}\}^*  \text{ und }  w_2\in \{\literal{a}\}^* \}$
		\end{itemize}
	\end{Beispiel}

	\begin{block}{Bemerkungen}
		\begin{itemize}
			\pause
			\item Die Beschreibung einer formalen Sprache ist nicht eindeutig (siehe oben).
			\pause
			\item Immer auf den Unterschied achten: $\{\literal{abc} \} \neq \literal{abc} $
		\end{itemize}
	\end{block}
\end{frame}


\begin{frame}
	\frametitle{Beispiele aus dem Leben}
	\begin{itemize}
		\item Sprache der korrekten IP4-Adressen \pause
		\begin{align*}
		192.168.178.1,\\
		76.147.112.6, ...
		\end{align*} 
		\pause Aber nicht: $000.999.123.666$ \pause
		\item Formale Sprache der Schlüsselwörter in Java $$L = \{ class, int , if, \dots \}$$ \pause
		\item Formale Sprache der legalen Zahlen vom Typ \textbf{int}: Mit $A = \{0...9\}$ 
		\pause $$A \cdot A^* = \{-19, 12849, 1001, 42, ...\}$$ 
		\pause Und minus? Also besser $ \{ -, \varepsilon \} \cdot A \cdot A^*$
	\end{itemize}
\end{frame}


\begin{frame}
	\frametitle{Produkt}
		\begin{Definition}
			Seien $L_1$ und $L_2$ zwei formale Sprachen. Dann bezeichnet
				$$L_1 \cdot L_2 = \{w_1 w_2 \mid w_1 \in L_1 \text{ und } w_2 \in L_2 \}$$
				das \textbf{Produkt} der Sprachen $L_1$ und $L_2$.
		\end{Definition}
		\pause
		In $L_1 \cdot L_2$ sind also alle Wörter enthalten, deren erster Teil aus $L_1$ und deren zweiter Teil aus $L_2$ ist.
	
\end{frame}

\begin{frame}
	\frametitle{Produkt}
	\begin{Beispiele}
		$\{a, b\} \cdot \{c, d\} = \{ac, ad, bc, bd\}$\\[0.3em]
		\pause
		$ L = \{w_1w_2 \mid w_1\in \{\literal{b}\}^* ,  w_2\in \{\literal{a}\}^* \} 
		= \{\literal{b}\}^* \cdot \{\literal{a}\}^* $\\[1em]
		\pause
		Für alle formalen Sprachen $L$ gilt:\\
		$$ L \cdot \{\varepsilon\} = L  \qquad L \cdot \emptyset = \emptyset$$
	\end{Beispiele}
\end{frame}

\begin{frame}
	\frametitle{Potenzen}
	\begin{Definition}
	Damit kann man induktiv die Potenz formaler Sprachen definieren:
	$$L^0 = \{\varepsilon \}$$
	$$L^{i+1} = L^i \cdot L$$
	\end{Definition} \pause
	$L^i$ enthält also alle Kombinationen von $i$ (nicht unbedingt verschiedenen) Wörtern aus $L$.
\end{frame}

\begin{frame}
	\frametitle{Konkatenationsabschluss}
	\begin{Definition}
		Der Konkatenationsabschluss einer formalen Sprache $L$ ist $$L^\ast = \bigcup \limits_{i=0}^\infty L^i$$ 
		\pause
		Der $\varepsilon$-freie Konkatenationsabschluss ist $$L^+ = \bigcup \limits_{i=1}^\infty L^i$$
	\end{Definition} \pause
	Achtung: Der $\varepsilon$-freie Konkatenationsabschluss muss nicht $\varepsilon$-frei sein! \pause
	
	$$ \{\}^* = \{\varepsilon\} $$
\end{frame}


\begin{frame}
	\frametitle{Aufgabe \stars{3}}
	Es sei $A = \{a, b\}$. Beschreiben Sie die folgenden formalen Sprachen mit den Symbolen $\word\{, \word\}, \word a, \word b, \eps, 
	\mword{\cup}, 
	\mword{{}^*}, 
	\word, , \word), \word( \text{ und } 
	\word{{}^+} 
	$:
	\begin{itemize}
		\item die Menge aller Wörter über $A$, die das Teilwort \word{ab} enthalten.
		\item die Menge aller Wörter über $A$, deren vorletztes Zeichen ein \word b ist.
		\item die Menge aller Wörter über $A$, in denen nirgends zwei \word bs unmittelbar hintereinander vorkommen.
	\end{itemize}
\end{frame}

\begin{frame}
	\frametitle{Lösung}
	\begin{itemize}
		\item \textit{die Menge aller Wörter über $A$, die das Teilwort \code{ab} enthalten.}  \pause
			$$\{a,b\}^\ast \cdot \{ab\} \cdot \{a,b\}^\ast$$ \pause
		\item \textit{die Menge aller Wörter über $A$, deren vorletztes Zeichen ein b ist.}  \pause
			$$\{a,b\}^\ast \cdot \{b\} \cdot \{a,b\}$$ \pause
		\item \textit{die Menge aller Wörter über $A$, in denen nirgends zwei \word bs unmittelbar hintereinander vorkommen.}  \pause
			$$\{a, ba\}^\ast \cdot \{b, \varepsilon \}$$
	\end{itemize}
\end{frame}

