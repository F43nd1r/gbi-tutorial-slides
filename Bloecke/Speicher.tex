\section{Funktionen von Funktionen}

\begin{frame}{Funktionen}
	\begin{Definition}
		Seien A und B Mengen. Dann ist $$B^A := \set{f \Mid f \from A \functionto B }$$ die Menge aller Abbildungen von $A$ nach $B$.
	\end{Definition}

	Abbildungen kann man sich auch als Tabellen vorstellen: \\
	Wir wählen für jedes $a \in A$ ein $b \in B$.
	
	\begin{block}{Beobachtung}
		Für endliche Mengen A und B gilt:
		\delimitershortfall=0pt
		$$\setsize{B^A} = \setsize B^{\setsize A}$$
	\end{block}
\end{frame}

\newcommand{\filter}{filter_{\only<2-3|handout:1>{even}\only<4|handout:2>{odd}\only<5|handout:3>{Uglg}}}

\begin{frame}{Problem: Ziffern filtern}
	Wir wollen aus einer Ziffernfolge (Wort aus $Z^*_{10}$) Ziffern herausfiltern, die eine bestimmte Eigenschaft haben:\\
	\visible<2->{\impl\ }\only<2-3|handout:1>{Gerade}\only<4|handout:2>{Ungerade}\only<5|handout:3>{Löst die Ungleichung $x^2 - 3 < 20$}
	
	\visible<3-> {
		\begin{threealign}
		\filter \from Z^*_{10} &\functionto& Z^*_{10} \\
		\eps &\mapsto& \eps \\
		z \cdot v &\mapsto& \begin{cases}
		z \cdot \filter(v) &\text{falls } \only<2-3|handout:1>{z \in \{\word 2,\word 4,\word 6,\word 8,\word 0\}} \only<4|handout:2>{z \in \{\word 1,\word 3,\word 5,\word 7,\word 9\}} \only<5|handout:3>{\fnum_{10}(z)^2 - 3 < 20 }\\
		\filter(v) &\text{sonst}
		\end{cases}\\
		& \multicolumn{2}{l}{\text{wobei } z \in Z_{10} \text{ und } v \in Z_{10}^*.}
		\end{threealign}
	}
\end{frame}

\begin{frame}{Problem: Ziffern filtern}
	\begin{block}{Aufgabe}
		Schreibt die Funktionen zum Filtern aller Ziffern...
		\begin{itemize}
			\item $> 5$
			\item $< 5$
			\item $<2$ oder $>7$
			\item ... die perfekt sind
			\item ... die sich ausmalen lassen
		\end{itemize}
	\end{block}

	\pause
	\impl Blödsinn!\\
	Alle Funktionen unterscheiden sich nur \textbf{an einer Stelle}: \pause \\
	Der \textbf{Bedingung}, die überprüft wird.
\end{frame}

\begin{frame}{Filter}
	Definiere stattdessen eine \textbf{universelle} Filter-Funktion:
	
	\begin{threealign}
		filter_p : Z^*_{10} &\functionto& Z^*_{10} \\
		\eps &\mapsto& \eps \\
		z \cdot v &\mapsto& \begin{cases}
		z \cdot filter_p(v) &\text{falls } p(z) = \W\\
		filter_p(v) &\text{sonst}
		\end{cases}\\
		&\multicolumn{2}{l}{\text{wobei } z \in Z_{10} \text{ und } v \in Z_{10}^*}
	\end{threealign}
	\medskip
	und zwar für alle \quad $p: Z_{10} \functionto \BB$.
\end{frame}

\begin{frame}{Filter}
	\begin{threealign}
		filter_p : Z^*_{10} &\functionto& Z^*_{10} \\
		\eps &\mapsto& \eps \\
		z \cdot v &\mapsto& \begin{cases}
		z \cdot filter_p(v) &\text{falls } p(z) = \W\\
		filter_p(v) &\text{sonst}
		\end{cases}\\
		&\multicolumn{2}{l}{\text{wobei } z \in Z_{10} \text{ und } v \in Z_{10}^*}
	\end{threealign}
	
	Definiere beispielsweise nun:
	$$ even:  Z_{10} \functionto \BB, \; z \mapsto \begin{cases}
	\textbf{w} &\text{falls } z \in \{\word 2,\word  4,\word  6,\word  8,\word  0\}\\
	\textbf{f} &\text{sonst }
	\end{cases}$$
	
	\pause
	Dann ist $filter_{even}(\word{123456}) = \word{246}$.
	
	\smallskip
	\impl Brauchen bloß $p$-Funktionen definieren anstatt jedes Mal ganze Filter-Funktion!
	
\end{frame}

% TOO MUCH FOR THE FIRST TIME, I GUESS...
\mycomment{
	\begin{frame}{Charakteristische Funktion}
		Sei $M$ eine abzählbar unendliche Menge und $L \subseteq M$.
		
		\begin{Definition}
			Die \textbf{Charakteristische Funktion} einer (Teil-)Menge $L$ ist die Funktion 
			\begin{threealign}
			C_L : M &\functionto& \{0, 1\}\\
			x &\mapsto& \begin{cases}
			1 &x \in L \\
			0 &x \notin L
			\end{cases}.
			\end{threealign}
		\end{Definition}
		
		Also ist $C_L \in \{0, 1\}^M$.
		\smallskip
		
		Klar ist außerdem: Es gibt eine Bijektion $C$ zwischen Mengen und Charakteristischen Funktionen: $$C : 2^M \to \{0, 1\}^M, L \mapsto C_L$$
	\end{frame}
	
	\begin{frame}{Charakteristische Funktion}
		Jetzt können wir auf diesen Funktionen äquivalente Operationen wie auf Mengen definieren...
		
		Vereinigung: $V\colon \{0,1\}^M \times \{0,1\}^M\to \{0,1\}^M$\\[1em]
		
		\only<2|handout:1>{
			Beispielbild für $L_1=\{a,c,d\}$ und $L_2=\{b,c\}$
			
			\begin{tabular}{*{4}{>{$}c<{$}}}
				& L_1 & L_2 & L_1\cup L_2 \\
				x & f_1(x) & f_2(x) & V(f_1,f_2) \\
				a & 1 & 0 & 1 \\
				b & 0 & 1 & 1 \\
				c & 1 & 1 & 1 \\
				d & 1 & 0 & 1 \\
				e & 0 & 0 & 0 \\
			\end{tabular}
		}
		
		\only<3-|handout:2> {
			Wie definiert man $V(f_1,f_2)$? Zum Beispiel so:
			\begin{align*}
			V \colon \{0,1\}^M \times  \{0,1\}^M &\to \{0,1\}^M \\
			(f_1,f_2) &\mapsto (x \mapsto \max(f_1(x),f_2(x))) \\
			\end{align*}
		} \only<4|handout:2> {
		Oder so: $V(f_1,f_2) (x) = \max(f_1(x),f_2(x))$
	}
	\end{frame}
}

\newcommand{\yellow}[1]{\fcolorbox{white}{yellow!60}{\ensuremath{#1}}}
\begin{frame}{Funktionen auf LSD: Filter}
	\impl Idee: Mache $p$ zu einem Argument von $filter$!
	\medskip
	\begin{threealign}
		filter \from \yellow{\BB^{Z_{10}} \times } Z^*_{10} &\functionto& Z^*_{10} \\
		(\yellow{p}, \eps) &\mapsto& \eps \\
		(\yellow{p}, z \cdot v) &\mapsto& \begin{cases}
		z \cdot filter(\yellow{p, }v) &\text{falls } p(z) = \W\\
		filter(\yellow{p, }v) &\text{sonst}
		\end{cases}\\
		&\multicolumn{2}{l}{\text{wobei } z \in Z_{10} \text{ und } v \in Z_{10}^*.}
	\end{threealign} \\
	\pause
	
	\impl $filter(even, \word{123456}) = \word{246}$. \\
	\impl $filter(odd, \word{123456}) = \word{135}$.
	\medskip
	
	\impl $filter$ ist jetzt eine Funktion, die \textbf{eine Funktion als Argument} nimmt.
\end{frame}

\newcommand{\fconst}{\text{const}}
\begin{frame}{Funktionen auf LSD: const}
	Schnell: Gebt sieben verschiedene konstante Funktionen an. \\
	\pause
	\impl Zu aufwendig? \visible<6->{\impl Nicht mit \fconst: $\set{\fconst(1),\fconst(2),\fconst(3),...}$}
	\pause
	
	\begin{threealign}
		\fconst \from \R &\functionto& \R^\R \\
		\fconst(a) &:=& \underbrace{\big( x \mapsto a \big)}_{\text{eine Funktion!}}
	\end{threealign} \\
	\pause
	\impl Definiere $f := \fconst(42)$. \quad Dann ist $f(5) = f(1000) = f(-2.3) = 42$. \\
	\smallskip
	\pause
	Oder auch: $\left(\fconst(42)\right)(5) = 42$. \\
	\medskip
	\pause[7]
	
	\impl $\fconst$ ist eine Funktion, die \textbf{eine Funktion als Ergebnis} zurückliefert.
\end{frame}

\begin{frame}
	Wir haben gesehen:
	\begin{itemize}
		\item Eine Abbildung, die eine Funktion auf einen Wert abbildet
		\item Eine Abbildung, die einen Wert auf eine Funktion abbildet
	\end{itemize}
	Gleich werden wir das kombinieren! \\
	\medskip
	(Hinweis: Abbildung $=$ Funktion, gell? \smiley)
\end{frame}

\section{Speicher}

\begin{frame}{Bit vs. Byte}
	\begin{Definition}
		Ein \textbf{Bit} ist ein Zeichen des Alphabets $\{\word 0, \word 1\}$. \\
		Ein Wort aus 8 \emph{Bits} wird \textbf{Byte} genannt. 
	\end{Definition}
	\pause	
	
	\begin{Definition}
		Ein \textbf{Speicher} $$m \from \Adr \functionto \Val$$ bildet Adressen ($\Adr$) auf Werte ($\Val$) ab.\\
		Wir schreiben $\Mem$ für $\Val^\Adr$. \\
		Also: $m \in \Val^{\Adr}$ \quad bzw. \quad $m \in \Mem$.
	\end{Definition}
	\impl Betrachten abstrakte Funktionsweise anstatt konkrete Realisierung 
	%Hier interessiert uns nicht die konkrete Realisierung der Speicherung, sondern nur die abstrakte Funktionsweise.
	\pause
	
	\begin{block}{Hinweis}
		Häufig: Adressen und Werte als Binärzahlen. \\
		%Im Umgang mit Speichern benutzen wir hier nur Zahlen im Binärsystem. \\
		% Zu detailliert:                     und was ist mit negatiuven Zahlen?
		%$\Adr = \set{0, ..., 2^k-1}, \Val = {0,...,2^l-1} \quad \text{für } k,l \in \N_+$.
	\end{block}
\end{frame}

\begin{frame}{Speicher}

	Operationen: \\
	\begin{itemize}
		\item $\memread(m, adr)$, um eine Speicherzelle zu lesen 
		\pause
		\item $\memwrite(m, adr, val)$, um in eine Speicherzelle zu „schreiben“.  
	\end{itemize}
	
	\pause
	\begin{threealign}
		\memread \from \Val^{\Adr} \times \Adr &\functionto& \Val \\
		(m,a) &\mapsto& m(a)  \medskip  
		\visible<4->{ \\
			\memwrite \from \Val^{\Adr} \times \Adr \times \Val &\functionto& \Val^{\Adr} \\
			(m,a',v') &\mapsto& m' , \quad \text{wobei} 
		} 
		\visible<5->{ \\ 
				m'(a) &=&
				\begin{cases}
				v' & \text{ falls } a=a' \\
				m(a) & \text{ falls } a\not=a' \\
				\end{cases}	
		}
	\end{threealign}	
	\pause[6]
	$\memwrite$ gibt die neue Speichertabelle $m'$ zurück!
\end{frame}

\begin{frame}{Speicher}
	\begin{block}{Beispiel}
		
	$\memread(m, \word{01}) = \only<2->{\word{00000111}}$\\
	\visible<2-|handout:2>{$m' := \memwrite(m, \word{01}, \word{11111100})$\\}
	\visible<3-|handout:2>{$\memread\big(\underbrace{\memwrite(m, \word{01}, \word{11111100})}_{m'}, \word{01}\big) = \word{11111100}$\\}
	\bigskip
	
	\begin{table}
		\begin{tabular}{r}
			\\ \\ \\ \\  \\
			\visible<4-|handout:2>{Realworld-Analogie: }
		\end{tabular}
		\begin{tabular}{|c|c|}
			\hline
			\multicolumn{2}{|c|}{Speicher $m$} \\
			\hline
			\word{00} & \word{00101000} \\
			\word{01} & \word{00000111} \\
			\word{10} & \word{10010110} \\
			\word{11} & \word{00100101} \\ 
			\hline 
			\multicolumn{2}{c}{\visible<4-|handout:2>{Zustand bei $t=0$}}
		\end{tabular} 
		\visible<2-|handout:2>{
			\begin{tabular}{|c|c|}
				\hline
				\multicolumn{2}{|c|}{Speicher $m'$} \\
				\hline
				\word{00} & \word{00101000} \\
				\underline{\word{01}} & \underline{\word{11111100}} \\
				\word{10} & \word{10010110} \\
				\word{11} & \word{00100101} \\
				\hline
				\multicolumn{2}{c}{\visible<4-|handout:2>{Zustand bei $t=1$}}
			\end{tabular} 
		}
	\end{table}

	\end{block}
\end{frame}

\mycomment{
	\only<handout:0>{\begin{frame}{Aufgaben}
		
	\end{frame}}
}