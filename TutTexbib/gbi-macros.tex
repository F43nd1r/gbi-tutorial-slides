% % \bigtimes abgeschrieben von http://tex.stackexchange.com/questions/14386/importing-a-single-symbol-from-a-different-font
% \DeclareFontFamily{U}{mathx}{\hyphenchar\font45}
% \DeclareFontShape{U}{mathx}{m}{n}{
%       <5> <6> <7> <8> <9> <10> gen * mathx
%       <10.95> mathx10 <12> <14.4> <17.28> <20.74> <24.88> mathx12
%       }{}
% \DeclareSymbolFont{mathx}{U}{mathx}{m}{n}
% \DeclareMathSymbol{\bigtimes}{\mathop}{mathx}{161}

\RequirePackage{xcolor}

\def\9{\square}
%\def\9{\blank}

% f"ur Aussagenlogik
\colorlet{alcolor}{blue}
\RequirePackage{tikz}
\usetikzlibrary{arrows.meta}
\newcommand{\alimpl}{\mathrel{\tikz[x={(0.1ex,0ex)},y={(0ex,0.1ex)},>={Classical TikZ Rightarrow[]}]{\draw[alcolor,->,line width=0.7pt,line cap=round] (0,0) -- (15,0);\path (0,-6);}}}
\newcommand{\aleqv}{\mathrel{\tikz[x={(0.1ex,0ex)},y={(0ex,0.1ex)},>={Classical TikZ Rightarrow[]}]{\draw[alcolor,<->,line width=0.7pt,line cap=round] (0,0) -- (18,0);\path (0,-6);}}}
\newcommand{\aland}{\mathbin{\raisebox{-0.6pt}{\rotatebox{90}{\texttt{\color{alcolor}\char62}}}}}
\newcommand{\alor}{\mathbin{\raisebox{-0.8pt}{\rotatebox{90}{\texttt{\color{alcolor}\char60}}}}}
%\newcommand{\ali}[1]{_{\mathtt{\color{alcolor}#1}}}
\newcommand{\alv}[1]{\mathtt{\color{alcolor}#1}}
\newcommand{\alnot}{\mathop{\tikz[x={(0.1ex,0ex)},y={(0ex,0.1ex)}]{\draw[alcolor,line width=0.7pt,line cap=round,line join=round] (0,0) -- (10,0) -- (10,-4);\path (0,-8) ;}}}
\newcommand{\alP}{\alv{P}} %ali{#1}}
%\newcommand{\alka}{\negthinspace\hbox{\texttt{\color{alcolor}(}}}
\newcommand{\alka}{\negthinspace\text{\texttt{\color{alcolor}(}}}
%\newcommand{\alkz}{\texttt{\color{alcolor})}}\negthinspace}
\newcommand{\alkz}{\text{\texttt{\color{alcolor})}}\negthinspace}
\newcommand{\AAL}{A_{AL}}
\newcommand{\LAL}{\hbox{\textit{For}}_{AL}}
\newcommand{\AxAL}{\hbox{\textit{Ax}}_{AL}}
\newcommand{\AxEq}{\hbox{\textit{Ax}}_{Eq}}
\newcommand{\AxPL}{\hbox{\textit{Ax}}_{PL}}
\newcommand{\AALV}{\hbox{\textit{Var}}_{AL}}
\newcommand{\MP}{\hbox{\textit{MP}}}
\newcommand{\GEN}{\hbox{\textit{GEN}}}
\newcommand{\W}{\ensuremath{\hbox{\textbf{w}}}\xspace}
\newcommand{\F}{\ensuremath{\hbox{\textbf{f}}}\xspace}
\newcommand{\WF}{\ensuremath{\{\W,\F\}}\xspace}
\newcommand{\val}{\hbox{\textit{val}}}
\newcommand{\valDIb}{\val_{D,I,\beta}}

\newcommand*{\from}{\colon}

% die nachfolgenden Sachen angepasst an cmtt
\newlength{\ttquantwd}
\setlength{\ttquantwd}{1ex}
\newlength{\ttquantht}
\setlength{\ttquantht}{6.75pt}
\def\plall{%
  \tikz[line width=0.67pt,line cap=round,line join=round,baseline=(B),alcolor] {
    \draw (-0.5\ttquantwd,\ttquantht) -- node[coordinate,pos=0.4] (lll){} (-0.25pt,-0.0pt) -- (0.25pt,-0.0pt) -- node[coordinate,pos=0.6] (rrr){} (0.5\ttquantwd,\ttquantht);
    \draw (lll) -- (rrr);
    \coordinate (B) at (0,-0.35pt);
  }%
}
\def\plexist{%
  \tikz[line width=0.67pt,line cap=round,line join=round,baseline=(B),alcolor] {
    \draw (-0.9\ttquantwd,\ttquantht) -- (0,\ttquantht) -- node[coordinate,pos=0.5] (mmm){} (0,0) --  (-0.9\ttquantwd,0);
    \draw (mmm) -- ++(-0.75\ttquantwd,0);
    \coordinate (B) at (0,-0.35pt);
  }\ensuremath{\,}%
}
\let\plexists=\plexist
\newcommand{\NT}[1]{\ensuremath{\langle\mathrm{#1} \rangle}}

\newcommand{\CPL}{\text{\itshape Const}_{PL}}
\newcommand{\FPL}{\text{\itshape Fun}_{PL}}
\newcommand{\RPL}{\text{\itshape Rel}_{PL}}
\newcommand{\VPL}{\text{\itshape Var}_{PL}}
\newcommand{\ATer}{A_{\text{\itshape Ter}}}
\newcommand{\ARel}{A_{\text{\itshape Rel}}}
\newcommand{\AFor}{A_{\text{\itshape For}}}
\newcommand{\LTer}{L_{\text{\itshape Ter}}}
\newcommand{\LRel}{L_{\text{\itshape Rel}}}
\newcommand{\LFor}{L_{\text{\itshape For}}}
\newcommand{\NTer}{N_{\text{\itshape Ter}}}
\newcommand{\NRel}{N_{\text{\itshape Rel}}}
\newcommand{\NFor}{N_{\text{\itshape For}}}
\newcommand{\PTer}{P_{\text{\itshape Ter}}}
\newcommand{\PRel}{P_{\text{\itshape Rel}}}
\newcommand{\PFor}{P_{\text{\itshape For}}}

\newcommand{\plka}{\alka}
\newcommand{\plkz}{\alkz}
%\newcommand{\plka}{\plfoo{(}}
%\newcommand{\plkz}{\plfoo{)}}
\newcommand{\plcomma}{\hbox{\texttt{\color{alcolor},}}}
\newcommand{\pleq}{{\color{alcolor}\,\dot=\,}}

\newcommand{\plfoo}[1]{\mathtt{\color{alcolor}#1}}
\newcommand{\plc}{\plfoo{c}}
\newcommand{\pld}{\plfoo{d}}
\newcommand{\plf}{\plfoo{f}}
\newcommand{\plg}{\plfoo{g}}
\newcommand{\plh}{\plfoo{h}}
\newcommand{\plx}{\plfoo{x}}
\newcommand{\ply}{\plfoo{y}}
\newcommand{\plz}{\plfoo{z}}
\newcommand{\plR}{\plfoo{R}}
\newcommand{\plS}{\plfoo{S}}

\newcommand{\bv}{\mathrm{bv}}
\newcommand{\fv}{\mathrm{fv}}

%\newcommand{\AxAL}{\hbox{\textit{Ax}}_{AL}}
%\newcommand{\AALV}{\hbox{\textit{Var}}_{AL}}

\renewcommand{\#}[1]{\literal{#1}}
\newcommand{\A}{\mathcal{A}}
\newcommand{\Adr}{\text{Adr}}
\newcommand{\ar}{\mathrm{ar}}
\newcommand{\ascii}[1]{\literal{\char#1}}
%\newcommand{\assert}[1]{\text{/\!\!/\ } #1}
\newcommand{\assert}[1]{\colorbox{black!7!white}{\ensuremath{\{\;#1\;\}}}}
\newcommand{\Assert}[1]{$\langle$\textit{#1}$\rangle$}
\newcommand{\B}{\mathcal{B}}
\newcommand{\bfmod}{\mathbin{\kw{ mod }}}
\newcommand{\bb}{{\text{bb}}}
\def\bottom{\hbox{\small$\pmb{\bot}$}}
\newcommand{\card}[1]{|#1|}
%\newcommand{\cod}{\mathop{\text{cod}}}  % ist in thwmathabbrevs
\newcommand{\Conf}{\mathcal{C}}
\newcommand{\define}[1]{\emph{#1}}
%\renewcommand{\dh}{d.\,h.\@\xspace}
%\newcommand{\Dh}{D.\,h.\@\xspace}
%\newcommand{\engl}[1]{engl.\xspace\emph{#1}}
\newcommand{\eps}{\varepsilon}
%\newcommand{\evtl}{evtl.\@\xspace}
\newcommand{\fbin}{\text{bin}}
\newcommand{\finv}{\text{inv}}
\newcommand{\fnum}{\text{num}}
\newcommand{\fNum}{{\text{Num}}}
\newcommand{\frepr}{\text{repr}}
\newcommand{\fRepr}{\text{Repr}}
\newcommand{\fZkpl}{\text{Zkpl}}
\newcommand{\fLen}{\text{Len}}
\newcommand{\fsem}{\text{sem}}
\providecommand{\fspace}{\mathord{\text{space}}}
\providecommand{\fSpace}{\mathord{\text{Space}}}
\providecommand{\ftime}{\mathord{\text{time}}}
\providecommand{\fTime}{\mathord{\text{Time}}}
\newcommand{\fTrans}{\text{Trans}}
\newcommand{\fVal}{\text{Val}}
\def\G{\mathbb{Z}}
\newcommand{\HT}[1]{\normalfont\textsc{HT-#1}}
\newcommand{\htr}[3]{\{#1\}\;#2\; \{#3\}}
\newcommand{\Id}{\text{I}}
%\newcommand{\ie}{i.\,e.\@\xspace}
\newcommand{\instr}[2]{\texttt{#1}\ \textit{#2}}
\newcommand{\Instr}[2]{\texttt{#1}\ \textrm{#2}}
\newcommand{\instrr}[3]{\texttt{#1}\ \textit{#2}\texttt{(#3)}}
\newcommand{\Instrr}[3]{\texttt{#1}\ \textrm{#2}\texttt{(#3)}}
\newcommand{\io}{\!\mid\!}
\usepackage{KITcolors}
\newcommand{\literal}[1]{\hbox{\textcolor{blue!95!white}{\textup{\texttt{\scalebox{1.11}{#1}}}}}}
%\newcommand{\literal}[1]{\hbox{\textcolor{KITblue!80!black}{\textup{\texttt{#1}}}}}
\def\kasten#1{\leavevmode\literal{\setlength{\fboxsep}{1pt}\fbox{\vrule  width 0pt height 1.5ex depth 0.5ex #1}}}
\newcommand{\kw}[1]{\ensuremath{\mathbf{#1}}}
\newcommand{\lang}[1]{\ensuremath{\langle#1\rangle}}
%\newcommand{\maw}{m.\,a.\,w.\@\xspace}
%\newcommand{\MaW}{M.\,a.\,w.\@\xspace}
\newcommand{\mdefine}[2][FOOBAR]{\define{#2}\def\foobar{FOOBAR}\def\optarg{#1}\ifx\foobar\optarg\def\optarg{#2}\fi\graffito{\optarg}}
\newcommand{\meins}{\rotatebox[origin=c]{180}{1}}
\newcommand{\Mem}{\text{Mem}}
\newcommand{\memread}{\text{memread}}
\newcommand{\memwrite}{\text{memwrite}}
\providecommand{\meta}[1]{\ensuremath{\langle}\textit{#1}\ensuremath{\rangle}}
\newcommand{\N}{\mathbb{N}}
\newcommand{\NP}{\mathbf{NP}}
\newcommand{\Nadd}{N_{\text{add}}}
\newcommand{\Nmult}{N_{\text{mult}}}
\newcommand{\Oh}[1]{O\left(#1\right)}
\newcommand{\Om}[1]{\Omega\left(#1\right)}
\newcommand{\personname}[1]{\textsc{#1}}
\newcommand{\regname}[1]{\texttt{#1}}
\newcommand{\mima}{\textsc{Mima}\xspace}
\newcommand{\mimax}{\textsc{Mima-X}\xspace}

\def\Pclass{\text{\bfseries P}}
\def\PSPACE{\text{\bfseries PSPACE}}

\newcommand{\SPush}{\text{push}}
\newcommand{\SPop}{\text{pop}}
\newcommand{\SPeek}{\text{peek}}
\newcommand{\STop}{\text{top}}
\newcommand{\STos}{\text{\itshape tos}}
\newcommand{\SBos}{\text{\itshape bos}}

\newcommand{\R}{\mathbb{R}}
\newcommand{\Rnullplus}{\R_0^{+}}
\newcommand{\Rplus}{\R_{+}}
\newcommand{\resp}{resp.\@\xspace}
\newcommand{\Sem}{\text{Sem}}
\newcommand{\sgn}{\mathop{\text{sgn}}}
\newcommand{\sqbox}{\mathop{\raisebox{-6.2pt}{\hbox{\hbox to 0pt{$^{^{\sqcap}}$\hss}$^{^{\sqcup}}$}}}}
\newcommand{\sqleq}{\sqsubseteq}
\newcommand{\sqgeq}{\sqsupseteq}
\newcommand{\Th}[1]{\Theta\left(#1\right)}
%\newcommand{\usw}{usw.\@\xspace}
\newcommand{\V}[1]{\hbox{\textit{#1}}}
\newcommand{\x}{\times}
\newcommand{\ZK}{\mathbb{K}}
\newcommand{\Z}{\mathbb{Z}}
%\newcommand{\zB}{z.\,B.\@\xspace}
%\newcommand{\ZB}{Z.\,B.\@\xspace}
% \newcommand{\bb}{{\text{bb}}}
% \def\##1{\hbox{\textcolor{darkblue}{\texttt{#1}}}}
% \def\A{\mathcal{A}}
% \newcommand{\0}{\#0}
% \newcommand{\1}{\#1}
% \newcommand{\Obj}{\text{Obj}}
% \newcommand{\start}{\mathop{\text{start}}}
% \newcommand{\compactlist}{\addtolength{\itemsep}{-\parskip}}
% \newcommand{\fval}{\text{val}}
% \newcommand{\lang}[1]{\ensuremath{\langle#1\rangle}}
% \newcommand{\io}{\!\mid\!}
% \def\sqbox{\mathop{\raisebox{-6.2pt}{\hbox{\hbox to 0pt{$^{^{\sqcap}}$\hss}$^{^{\sqcup}}$}}}}
% \def\sqleq{\sqsubseteq}
% \def\sqgeq{\sqsupseteq}
\def\Td{T_{\overline{d}}}
% \newcommand{\csym}[1]{\ensuremath{\#{c}_{\#{\hbox{\scriptsize #1}}}}}
% \newcommand{\F}{\ensuremath{\mathcal{F}}}
% \newcommand{\fsym}[2]{\ensuremath{\#{f}^{\#{\hbox{\scriptsize #1}}}_{\#{\hbox{\scriptsize #2}}}}}
% \newcommand{\rsym}[2]{\ensuremath{\#{R}^{\#{\hbox{\scriptsize #1}}}_{\#{\hbox{\scriptsize #2}}}}}
% \newcommand{\xsym}[1]{\ensuremath{\#{x}_{\#{\hbox{\scriptsize #1}}}}}
% \newcommand{\I}{\mathcal{I}}
% ********************************************************************
