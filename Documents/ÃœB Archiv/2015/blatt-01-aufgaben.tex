\documentclass[12pt]{article}

\input{preamble-aufgaben}
% =======================================================

% =======================================================
%\newcounter{blattnr}
%\setcounter{blattnr}{1}
\newcommand{\ausgabetermin}{28.~Oktober 2015}
\newcommand{\abgabetermin}{6.~Novber 2015}
\newcommand{\punkteblatt}{13} % Blatt 1
\newcommand{\punktetotal}{13} % Blatt 1
\newcommand{\punkteblattphysik}{13} % Blatt 1
\newcommand{\punktetotalphysik}{13} %
% =======================================================

\begin{document}

% -----------------------------------------------------------------------------
\def\ssm{\smallsetminus}
\begin{aufgabe}[3]
  Es sei $M$
  eine Menge und es seien $A\subseteq M$
  und $B\subseteq M$. Beweisen Sie:
  \[
    M \smallsetminus (A\cup B) = (M\smallsetminus A) \cap (M\smallsetminus B)
  \]
\end{aufgabe}

\begin{loesung}
  \begin{description}
  \item[$\subseteq$:]
    Es sei $x\in M \ssm (A\cup B)$.
    Dann ist $x\in M$,
    und $x\notin A\cup B$.
    Also ist $x \in M$,
    und $x\notin A$ und $x\notin B$. Somit ist
    \begin{itemize}
    \item $x\in M$ und $x\notin A$ und
    \item $x\in M$ und $x\notin B$.
    \end{itemize}
    Damit ist $x\in M\ssm A$ und $x\in M\ssm B$.
    Folglich ist $x\in (M\ssm A)\cap(M\ssm B)$.
  \item[$\supseteq$:]
    Es sei $x \in (M\smallsetminus A) \cap (M\smallsetminus B)$.
    Dann ist $x \in M\smallsetminus A$ und $x \in M\smallsetminus B$.
    Also ist $x \in M$ und $x \notin A$, und $x \in M$ und $x \notin B$.
    Somit ist $x \in M$, und $x \notin A$ und $x \notin B$.
    Damit ist $x \in M$ und $x \notin A \cup B$.
    Folglich ist $x \in M \ssm (A\cup B)$.
  \end{description}

  \begin{korrektur}
    falls zwei Inklusionen gezeigt: je 1.5 Punkte \\
    falls mit lauter "`gdw."' argumentiert: geeignete Abzüge bei Fehlern
  \end{korrektur}
\end{loesung}

% -----------------------------------------------------------------------------
\begin{aufgabe}[1 + 1 + 1 + 1 + 2 = 6]
  Es sei $f\colon A \to B$
  eine Abbildung. Zu $f$ definieren wir die Abbildung
  \[
    f^{-1}\colon 2^B \to 2^A, M \mapsto \{ a\in A \mid f(a) \in M \}
  \]
  Für jedes $M\subseteq B$
  nennt man $f^{-1}(M)$ das \define{Urbild} von $M$ (unter $f$).

  \begin{enumerate}
  \item Welche Bedingung muss $f$
    erfüllen, damit $f^{-1}$ injektiv ist?
  \item Welche Bedingung muss $f$
    erfüllen, damit $f^{-1}$ surjektiv ist?
  \item\label{it:d} Es sei $M\subseteq B$.
    Welche Mengenbeziehung besteht zwischen $M$ und $f(f^{-1}(M))$?
  \item Es sei $M\subseteq A$.
    Welche Mengenbeziehung besteht zwischen $M$ und $f^{-1}(f(M))$?
  \item Beweisen Sie Ihre Behauptung in Teilaufgabe \ref{it:d}.
  \end{enumerate}
\end{aufgabe}

\begin{loesung}
  \begin{enumerate}
  \item $f$ muss surjektiv sein.
  \item $f$ muss injektiv sein.
  \item $f(f^{-1}(M)) \subseteq M$. Anders ausgedrückt: $M \supseteq f(f^{-1}(M))$
  \item $M \subseteq f^{-1}(f(M))$. Anders ausgedrückt: $f^{-1}(f(M)) \supseteq M$
  \item Es sei $b\in f(f^{-1}(M))$.
    \begin{itemize}
    \item Dann gibt es ein $a\in f^{-1}(M)$
      mit $f(a) =b$.
    \item $a\in f^{-1}(M)$
      bedeutet gerade $f(a)\in M$.
    \item Wegen $b = f(a)$, folgt $b\in M$.
    \end{itemize}
  \end{enumerate}
  \begin{korrektur}
    \begin{itemize}
    \item a) bis d) je 1 Punkt für richtige Antwort;\\
      kann man sich Fälle vorstellen für 0.5 Punkte?
    \item bei e):
    \end{itemize}
  \end{korrektur}
\end{loesung}
% -----------------------------------------------------------------------------

\begin{aufgabe}[0.5 + 1.5 + 2 = 4]
  \begin{enumerate}
  \item Nichtnegative ganze Zahlen $x_i$,
    $i\in\N_0$, seien wie folgt definiert:
    %
    \[\begin{aligned}[t]
      x_0 &= 4 \;,\\
      \text{für jedes } n\in\N_0\colon x_{n+1} &= x_n + 2n + 5 \;.
    \end{aligned}
    \]%$\\[0.5\baselineskip]

    Geben Sie die Zahlenwerte von $x_1$, $x_2$, $x_3$ und $x_4$ an.
  \item Geben Sie für jedes $n\in\N_0$
    einen arithmetischen Ausdruck $E_n$,
    in dem kein $x_i$ vorkommt, so an, dass gilt: $x_n = E_n$.
  \item Geben Sie die induktive Definition für ganze Zahlen $y_i$,
    $i\in\N_0$, so an, dass für jedes $n\in\N_0$ gilt:
    \[
      y_n =
      \begin{cases}
        n, & \text{ falls $n$ gerade ist,} \\
        -n, & \text{ falls $n$ ungerade ist.}
      \end{cases}
    \]
    \emph{Hinweis:} In der Definition von $y_{n+1}$
    müssen Sie $y_n$
    sinnvoll benutzen. "`Scheinbenutzungen"' wie $\cdots y_n-y_n \cdots$ sind nicht ausreichend.
  \end{enumerate}
\end{aufgabe}

\begin{loesung}
  \begin{enumerate}
  \item $x_1=9$, $x_2=16$, $x_3=25$, $x_4=36$
  \item $E_n=(n+2)^2$
  \item zum Beispiel:
    \begin{align*}
      y_0 &= 0\\
      \text{für jedes } n\in\N_0\colon y_{n+1} &= -y_n + (-1)^{n+1}
    \end{align*}
    oder
    \begin{align*}
      y_0 &= 0\\
      \text{für jedes } n\in\N_0\colon y_{n+1} &= y_n + (-1)^{n+1}(2n+1)
    \end{align*}
  \end{enumerate}

  \begin{korrektur}
    \begin{enumerate}
    \item bei einem Fehler noch 0.5 Punkte, sonst 0 Punkte
    \item was machen wir mit $E_n=n^2+4n+4$?
    \item statt $(-1)^{n+1}$
      kann man \zB auch $(n\bmod 2) - ((n+1) \bmod 2)$
      schreiben\\
      bitte gründlich prüfen, ob studentische Lösungen
      korrekt sind
      und $y_n$ nichttrivial verwendet wird \\
      0.5 Punkte auf Anfang und 1.5 auf richtige Rekursion
    \end{enumerate}
  \end{korrektur}
\end{loesung}
% -----------------------------------------------------------------------------

\noindent
\emph{Allgemeiner Hinweis:} In dieser Vorlesung kommen an einigen
Stellen griechische Buchstaben vor. In anderen Vorlesungen wird das
auch passieren. Hier ist die Liste der Kleinbuchstaben (manchmal gibt
es verschiedene Schreibweisen):

$\alpha$,
$\beta$,
$\gamma$,
$\delta$,
$\varepsilon$
(oder $\epsilon$),
$\zeta$,
$\eta$,
$\theta$
(oder $\vartheta$),
$\iota, \kappa$,
$\lambda$,
$\mu$,
$\nu$,
$\xi$,
$o$,
$\pi$,
$\rho$,
$\sigma$,
$\tau$, $\upsilon,$ $\phi$, $\chi$, $\psi$, $\omega$

\noindent
Machen Sie sich mit der Schreibweise und den Namen der Zeichen vertraut!
% -----------------------------------------------------------------------------

\end{document}
%%%
%%% Local Variables:
%%% fill-column: 70
%%% mode: latex
%%% TeX-command-default: "XPDFLaTeX"
%%% TeX-master: "korrektur.tex"
%%% End:
