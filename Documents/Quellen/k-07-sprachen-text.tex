%=======================================================================
\Tut\chapter{Formale Sprachen}
\label{k:sprachen}

Den Begriff der formalen Sprache haben wir schon in Kapitel~\ref{k:woerter}
eingeführt.
%
Eine formale Sprache über einem Alphabet $A$
ist eine Teilmenge $L\subseteq A^*$. 
%
In diesem kurzen Kapitel definieren wir noch einige nützliche Operationen auf
formalen Sprachen.

%-----------------------------------------------------------------------
\Tut\section{Operationen auf formalen Sprachen}

%-----------------------------------------------------------------------
\Tut\subsection{Produkt oder Konkatenation formaler Sprachen}
\label{subsec:produkt-sprachen}

Wir haben schon definiert, was die Konkatenation zweier Wörter ist. 
%
Das erweitern wir nun auf eine übliche Art und Weise auf Mengen von Wörtern:
Für zwei formale Sprachen $L_1$ und $L_2$ heißt
\[
L_1\cdot L_2=\{ w_1 w_2 \mid w_1\in L_1\text{ und } w_2\in L_2 \}
\]
das \mdefine[Produkt von Sprachen]{Produkt der Sprachen $L_1$ und
  $L_2$.}\index{Sprache!Produkt}\index{Produkt!formaler Sprachen}
%
\begin{tutorium}
  \noindent\textbf{Produkt von Sprachen}
  \begin{itemize}
  \item Def: $L_1\cdot L_2 = \{w_1w_2 \mid w_1\in L_1 \text{ und } w_2\in L_2\}$
  \item Beispiel: noch mal: formale Sprache $L$ aller Wörter über
    $A=\{\literal{a},\literal{b}\}$, in denen nirgends das Teilwort
    $\literal{ab}$ vorkommt.\\
    kann man jetzt so schreiben: $L=\{\literal{b}\}^*\{\literal{a}\}^*$
  \item Beispiel: Menge aller Wörter über $A$ außer dem leeren: $A \cdot A^*$,
    denn jedes nichtleere Wort hat ein erstes Symbol und dahinter kommt ein
    beliebiges (evtl auch leeres) Wort
  \item formale Sprache $L_I$ der legalen Zahlen vom Typ \textbf{int}:
    \begin{itemize}
    \item Versuch:
      $A=\{\literal{0},\ldots,\literal{9}\}$ \\
      $L_I= A \cdot A^*$
    \item Was fehlt? jedenfalls das Minuszeichen; \\
      besser: $\{\eps,\literal{-}\}\cdot A \cdot A^*$
    \item Was ist mit Präfix $\literal{0x}$? gibts den? ich weiß es nicht 
    \end{itemize}
  \item formale Sprache $L_V$ der legalen Variablennamen in Java:
    \begin{itemize}
    \item Versuch:
      $A=\{\literal{\_},\literal{a},\ldots,\literal{z},\literal{A},\ldots,\literal{Z}\}$,
      $B=A \cup \{\literal{0},\ldots,\literal{9}\}$ \\
      $L_V= A \cdot B^*$.
    \item es fehlen die Umlaute, \dots
    \item Was ist noch falsch? \zB Schlüsselwörter (\textbf{if},
      \dots) sind als Variablennamen verboten. \\
      also eher sowas wie $L_V= (A \cdot B^*)
      \smallsetminus\{\literal{if}, \literal{class}, \dots\}$ \\
      Mitteilung: da könnte man jetzt alle endlich vielen Schlüsselwörter
      aufzählen, aber wenn nur endlich viele Wörter verboten sind, geht es im
      Prinzip ohne Mengendifferenz
    \end{itemize}
  \item $L_1=\{\literal{a}^n \mid n\in\N_0\}$ und  $L_2=\{\literal{b}^n \mid n\in\N_0\}$ \\
    \textbf{Achtung:} $L_1L_2=\{\literal{a}^k\literal{b}^m \mid
    k\in\N_0\text{ und } m\in\N_0\}$  die \textbf{Exponenten können verschieden
      sein!} \\
    hier steht nichts anderes als $\{\literal{a}\}^*\{\literal{b}\}^*$
  \end{itemize}
\end{tutorium}
%
\begin{lemma}
  \label{lem:L-eps}
  Für jede formale Sprache $L$ ist 
  \[
  L\cdot\{\eps\} = L = \{\eps\}\cdot L \;.
  \]
\end{lemma}

\begin{beweis}
  Einfaches Nachrechnen:
  \begin{align*}
    L\cdot\{\eps\} &=  \{ w_1 w_2 \mid w_1\in L \text{ und }  w_2\in \{\eps\} \} \\
    &= \{ w_1 w_2 \mid w_1\in L \text{ und }  w_2 =\eps \} \\
    &= \{ w_1\eps  \mid w_1\in L \} \\
    &= \{ w_1 \mid w_1\in L \} \\
    &= L
  \end{align*}
  Analog zeigt man $L = \{\eps\}\cdot L$.
\end{beweis}
%
\def\Lemail{L_{\text{\itshape email}}}
\def\Lheader{L_{\text{\itshape header}}}
\def\Lbody{L_{\text{\itshape body}}}
\def\Lleer{L_{\text{\itshape leer}}}
In Abschnitt~\ref{subsec:woerter-aufbau-emails} haben wir ein erstes Mal über
den Aufbau von E-Mails gesprochen. 
%
Produkte formaler Sprachen könnte man nun benutzen, um folgende Festlegung zu
treffen:
\begin{itemize}
\item Die formale Sprache $\Lemail$ der syntaktisch korrekten E-Mails ist
  \[
  \Lemail = \Lheader \cdot \Lleer \cdot \Lbody
  \]
\item Dabei sei
  \begin{itemize}
  \item $\Lheader$ die formale Sprache der syntaktisch korrekten E-Mail-Köpfe,
  \item $\Lleer$ die formale Sprache, die nur die Leerzeile enthält, also
    $\Lleer=\{\,\kasten{CR}\,\kasten{LF}\,\}$ und
  \item $\Lbody$ die formale Sprache der syntaktisch korrekten E-Mail-Rümpfe.
  \end{itemize}
\end{itemize}
% 
$\Lheader$ und $\Lbody$ muss man dann natürlich auch noch definieren.
%
Eine Möglichkeit, das bequem zu machen, werden wir in einem späteren Kapitel
kennenlernen.

Wie bei Wörtern will man \mdefine{Potenzen $L^k$}\index{Potenzen!einer
  formalen Sprache}\index{formale Sprache!Potenzen} definieren. 
%
Der "`Trick"' besteht darin, für den Fall $k=0$
etwas Sinnvolles zu finden --- Lemma~\ref{lem:L-eps} gibt einen Hinweis. 
%
Die Definition geht wieder induktiv:
\begin{align*}
  L^0 &= \{\eps\} \\
  \text{für jedes } k\in\N_0:\; L^{k+1} &= L \cdot L^k
\end{align*}
%
Wie auch schon bei der Konkatenation einzelner Wörter kann man auch hier
wieder nachrechnen, dass \zB gilt:
\begin{align*}
  L^1 &= L \\
  L^2 &= L\cdot L \\
  L^3 &= L\cdot L\cdot L \\
\end{align*}
%
\begin{tutorium}
  \noindent\textbf{Potenzen von $L$}

  \begin{itemize}
  \item Def: $L^0=\{\eps\}$ und $L^{i+1}=L^i\cdot L$
  \item Beispiel: $L=\{\literal{a}\}^*\{\literal{b}\}^*$\\
    dann enthält \zB 
    \begin{itemize}
    \item $L^0=\{\eps\}$
    \item $L^1=L=\{\eps, \literal{aabbbbb}, \literal{aaaaab}, \literal{aaaa},
      \literal{bbbbbbbb}, \dots \}$
    \item $L^2=\{\eps, \literal{aabbbbaaaaab}, \literal{aaaaabab}, \literal{aaaaa},
      \literal{bbbbbbb}, \dots \}$
    \item \usw
    \end{itemize}
  \end{itemize}
\end{tutorium}
Genau genommen hätten wir in der dritten Zeile $L\cdot (L\cdot L)$ schreiben
müssen. Aber Sie dürfen glauben (oder nachrechnen), dass sich die
Assoziativität vom Produkt von Wörtern auf das Produkt von Sprachen überträgt.

Als einfaches Beispiel betrachte man $L=\{\literal{aa}, \literal{b}\}$. Dann
ist
\begin{align*}
  L^0 &= \{\eps\} \\
  L^1 &= \{\literal{aa}, \literal{b}\} \\
  L^2 &= \{\literal{aa}, \literal{b}\}\cdot \{\literal{aa}, \literal{b}\} 
       = \{\literal{aa}\cdot\literal{aa}, \literal{aa}\cdot\literal{b}, \literal{b}\cdot\literal{aa}, \literal{b}\cdot\literal{b}\} \\
      &= \{\literal{aaaa}, \literal{aab}, \literal{baa}, \literal{bb}\} 
  \\
  L^3 &= \begin{aligned}[t]
         \{&\literal{aa}\cdot\literal{aa}\cdot\literal{aa}, \; 
           \literal{aa}\cdot\literal{aa}\cdot\literal{b}, \; 
           \literal{aa}\cdot\literal{b}\cdot\literal{aa}, \; 
           \literal{aa}\cdot\literal{b}\cdot\literal{b}, \; \\
           &\literal{b}\cdot\literal{aa}\cdot\literal{aa}, \; 
           \literal{b}\cdot\literal{aa}\cdot\literal{b}, \; 
           \literal{b}\cdot\literal{b}\cdot\literal{aa}, \; 
           \literal{b}\cdot\literal{b}\cdot\literal{b}\} 
         \end{aligned} \\
         &= \{\literal{aaaaaa}, 
           \literal{aaaab}, 
           \literal{aabaa}, 
           \literal{aabb},
           \literal{baaaa},
           \literal{baab},
           \literal{bbaa}, 
           \literal{bbb}\} \\
\end{align*}
%
In diesem Beispiel ist $L$ endlich. 
%
Man beachte aber, dass die Potenzen auch definiert sind, wenn $L$
unendlich ist. 
%
Betrachten wir etwa den Fall
\[
L = \{ \literal{a}^n\literal{b}^n \mid n\in\N_+\} \;,
\]
es ist also (angedeutet)
\[
L = \{ \literal{ab}, \literal{aabb}, \literal{aaabbb},\literal{aaaabbbb},
\dots \} \;.
\]
Welche Wörter sind in $L^2$?  
%
Die Definion besagt, dass man alle Produkte $w_1w_2$
von Wörtern $w_1\in L$ und $w_2\in L$ bilden muss.
%
Man erhält also (erst mal ungenau hingeschrieben)
\begin{equation*}
  L^2 = \begin{aligned}[t]
    & \{ \literal{ab}\cdot\literal{ab}, \literal{ab}\cdot\literal{aabb}, \literal{ab}\cdot\literal{aaabbb}, \dots \} \\
    \cup\  & \{ \literal{aabb}\cdot\literal{ab}, \literal{aabb}\cdot\literal{aabb}, \literal{aabb}\cdot\literal{aaabbb}, \dots \} \\
    \cup\  & \{ \literal{aaabbb}\cdot\literal{ab}, \literal{aaabbb}\cdot\literal{aabb}, \literal{aaabbb}\cdot\literal{aaabbb}, \dots \} \\
    \vdots \\
  \end{aligned}
\end{equation*}
%
Mit anderen Worten ist
\[
L^2 = \{ \literal{a}^{n_1}\literal{b}^{n_1}\literal{a}^{n_2}\literal{b}^{n_2} \mid n_1\in\N_+  \text{ und }  n_2\in\N_+ \} \;.
\]
Man beachte, dass bei die Exponenten $n_1$ "`vorne"' und $n_2$ "`hinten"'
verschieden sein dürfen.

Für ein Alphabet $A$ und für $i\in\N_0$ hatten wir auch die Potenzen $A^i$
definiert.  
%
Und man kann jedes Alphabet ja auch als eine formale Sprache auf"|fassen, die
genau alle Wörter der Länge $1$ über $A$ enthält.
%
Machen Sie sich klar, dass die beiden Definitionen für Potenzen konsistent
sind, \dh $A^i$ 
ergibt immer die gleiche formale Sprache, egal, welche Definition man zu
Grunde legt.
      
%-----------------------------------------------------------------------
\Tut\subsection{Konkatenationsabschluss einer formalen Sprache}

Bei Alphabeten hatten wir neben den $A^i$ auch noch $A^*$ definiert und darauf
hingewiesen, dass für ein Alphabet $A$ gilt:
\[
A^* =\bigcup_{i\in\N_0} A^i \;.
\]
Das nehmen wir zum Anlass nun den \mdefine{Konkatenationsabschluss $L^*$ von
  $L$}\index{Konkatenationsabschluss} und den \mdefine[$\eps$-freier
Konkatenationsabschluss $L^+$ von $L$]{$\eps$-freien Konkatenationsabschluss
  $L^+$ von $L$}%
\index{Konkatenationsabschluss!$\eps$-freier}%
\index{e-freier@{$\eps$-freier Konkatenationsabschluss}} definieren:
%
\begin{equation*}
  L^+ = \bigcup_{i\in\N_+} L^i \text{\qquad und\qquad}
  L^* = \bigcup_{i\in\N_0} L^i
\end{equation*}
% 
Wie man sieht, ist $L^*=L^0\cup L^+$. In $L^*$ sind also alle Wörter, die sich
als Produkt einer beliebigen Zahl (einschließlich $0$) von Wörtern schreiben
lassen, die alle Element von $L$ sind.
%
\begin{tutorium}
  \noindent\textbf{Kokatenationsabschluss}
  \begin{itemize}
  \item Def: 
    \begin{equation*}
      L^+ = \bigcup_{i\in\N_+} L^i \text{\qquad und\qquad}
      L^* = \bigcup_{i\in\N_0} L^i
    \end{equation*}
  \item Beispiel: $L=\{\literal{a}\}^*\{\literal{b}\}^*$
    \begin{itemize}
    \item Man mache sich klar: $L^* = \{\literal{a},\literal{b}\}^*$, also
      alles
    \item da geht \zB so: (die Studenten möglichst selber drauf kommen lassen)
      zerhacke beliebiges aber festes $w\in \{\literal{a},\literal{b}\}^*$ an
      allen Stellen, wo Teilwort $\literal{ba}$ vorkommt, zwischen dem
      $\literal{b}$ und dem $\literal{a}$.
      Die entstehenden Teilwörter sind aus $L$.
    \end{itemize}
  \item Man beweise:  $ L^*\cdot L = L^+$
    \begin{itemize}
    \item Wie beweist man, dass zwei Mengen gleich sind?
    \item Zum Beispiel, indem man zeigt, dass $\subseteq$ und
      $\supseteq$ gelten.
    \item Also:
      \begin{itemize}
      \item $\subseteq$: \\
        Wenn $w\in L^*\cdot L$, dann $w=w'w''$ mit
        $w'\in L^*$ und $w''\in L$. \\
        Also existiert ein $i\in\N_0$ mit $w'\in L^i$.\\
        Also $w=w'w''\in L^i\cdot L = L^{i+1}$. \\
        Da $i+1\in\N_+$, ist $L^{i+1}\subseteq L^+$, also $w\in L^+$.
      \item $\supseteq$: Wenn $w\in L^+$, dann existiert ein $i\in\N_+$
        mit $w\in L^i$. \\
        Da $i\in \N_+$ ist $i=j+1$ für ein $j\in\N_0$, \\
        also ist für ein $j\in\N_0$: $w\in L^{j+1}= L^j\cdot L$. \\
        also $w=w'w''$ mit $w'\in L^j$ und $w''\in L$ . \\
        Wegen $L^j\subseteq L^*$ ist $w=w'w''\in L^*\cdot L$.
      \end{itemize}
    \end{itemize}
  \end{itemize}
\end{tutorium}

Als Beispiel betrachten wieder $L = \{ \literal{a}^n\literal{b}^n \mid
n\in\N_+\}$. 
%
Weiter vorne hatten wir schon gesehen:
%
\[
L^2 = \{ \literal{a}^{n_1}\literal{b}^{n_1}\literal{a}^{n_2}\literal{b}^{n_2}
\mid n_1\in\N_+  \text{ und }  n_2\in\N_+ \} \;.
\]
%
Analog ist 
%
\[
L^3 = \{ \literal{a}^{n_1}\literal{b}^{n_1}\literal{a}^{n_2}\literal{b}^{n_2}
\literal{a}^{n_3}\literal{b}^{n_3} \mid n_1\in\N_+  \text{ und }  n_2\in\N_+  \text{ und } 
n_3\in\N_+ \} \;.
\]
%
Wenn wir uns erlauben, Pünktchen zu schreiben, dann ist allgemein
%
\[
L^i = \{
\literal{a}^{n_1}\literal{b}^{n_1}\cdots \literal{a}^{n_i}\literal{b}^{n_i}
\mid n_1\ldots, n_i\in\N_+ \} \;.
\]
Und für $L^+$ könnte man vielleicht notieren:
\[
L^+ = \{
\literal{a}^{n_1}\literal{b}^{n_1}\cdots \literal{a}^{n_i}\literal{b}^{n_i}
\mid i\in\N_+  \text{ und }   n_1\ldots, n_i\in\N_+ \} \;.
\]
Aber man merkt (hoffentlich!) an dieser Stelle doch, dass uns $^+$ und $^*$
die Möglichkeit geben, etwas erstens präzise und zweitens auch noch kürzer zu
notieren, als wir es sonst könnten.

Zum Abschluss wollen wir noch darauf hinweisen, dass die Bezeichnung
$\eps$-freier Konkatenationsabschluss für $L^+$ leider (etwas?) irreführend
ist. 
%
Wie steht es um das leere Wort bei $L^+$ und $L^*$? Klar sollte
inzwischen sein, dass für \emph{jede} formale Sprache $L$ gilt: $\eps\in
L^*$. 
%
Das ist so, weil ja $\eps\in L^0\subseteq L^*$ ist.
%
Nun läuft zwar in der Definition von $L^+$
die Vereinigung der $L^i$
nur ab $i=1$.
%
Es kann aber natürlich sein, dass $\eps\in L$ ist.
%
In diesem Fall ist dann aber offensichtlich $\eps\in L=L^1\subseteq L^+$.
Also kann $L^+$ sehr wohl das leere Wort enthalten.

Außerdem sei erwähnt, dass die Definition von $L^*$ zur Folge hat, dass gilt:
\[
\{\}^* = \{\eps\}
\]

%-----------------------------------------------------------------------
\section{Zusammenfassung}

In diesem Kapitel wurden \emph{Produkt} und der \emph{Konkatenationsabschluss}
\emph{formaler Sprachen} eingeführt.

Wir haben gesehen, dass man damit jedenfalls manche formalen Sprachen kurz und
verständlich beschreiben kann. 
%
Dass diese Notationsmöglichkeiten auch in der Praxis Verwendung finden, wird
später noch deutlich werden.
%
Manchmal reicht das, was wir bisher an Notationsmöglichkeiten haben, aber noch
nicht.
%
Deshalb werden wir in späteren Kapiteln auch mächtigere Hilfsmittel
kennenlernen.
 
\cleardoublepage

%-----------------------------------------------------------------------
%%%
%%% Local Variables:
%%% fill-column: 78
%%% mode: latex
%%% TeX-master: "../k-07-sprachen/skript.tex"
%%% TeX-command-default: "XPDFLaTeX"
%%% End:
