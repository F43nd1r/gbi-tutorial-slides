%beamer

% Comment/uncomment this line to toggle handout mode
%\newcommand{\handout}{}

\input{../framework/PraeambelTut.tex}

\morescalingdelimiters

\begin{document}
\starttut{4}



\framePrevEpisode

\begin{frame}{Rückblick}
	\begin{itemize}
		\item \textbf{Aussagen} sind Sätze, die wahr oder falsch sind
		\item Wir können Aussagen mit \textbf{Konnektiven} zusammenbauen: \\
		$\bund, \boder, \bnot, \bimp$
		\item \textbf{Aussagevariablen} helfen dabei, konkrete Inhalte zu ignorieren 
		\item \textbf{Interpretationen} liefern Wahrheitswerte zu Variablen
		\item $val_I(\*)$ liefert Wahrheitswert für ganze Formel (rekursiv)
	\end{itemize}
\end{frame}

\begin{frame}{Äquivalenz \& Co.}
	\begin{itemize}
		\item $P \leftrightarrow Q \quad\widehat{=}$ Wahr, wenn $P$ und $Q$ äquivalent sind. \\($P, Q$ sind aussagenlogische Variablen oder Formeln)
		\item $P \equiv Q \quad\widehat{=}$ Wir wissen: $P$ und $Q$ sind äquivalent für alle Interpretationen. \\($P, Q$ sind aussagenlogische Formeln)
		\item $P \Leftrightarrow Q \quad\widehat{=}$ $P$ gilt genau dann wenn $Q$ gilt \\($P, Q$ sind Aussagen, verwendet in Beweisen)
		\item $P = Q \quad\widehat{=}$ $P$ ist gleich $Q$ \\($P, Q$ sind Werte, z.B. Zahlen oder Ausdrücke, z.B. $f(x), a + b$)
		\item $P \widehat{=} Q \quad\widehat{=}$ $P$ entspricht $Q$ (informell)
	\end{itemize}
\end{frame}

\begin{frame}[t]{Wahr oder Falsch?}
 	% Socrative: https://b.socrative.com/teacher/#import-quiz/31489145
 	\Socrative
 	
 	\TrueQuestion{Dieser Satz ist eine Aussage.}
 	\FalseQuestionE{$\alA \boder \alB = \alB \boder \alA$}{Das sind syntaktisch verschiedene AL-Formeln!}
 	\TrueQuestion{$\alA \boder \alB \equiv \alB \boder \alA$}
 	% \TrueQuestion{Der AL-Kalkül ist vollständig und korrekt.}  % Wurde noch gar nicht erklärt...!?
 	%\TrueQuestion{Es gibt unendlich viele Axiome im Aussagenkalkül.}
 	%\FalseQuestion{$\bleftBr \word G \bimp \word H \brightBr \; \vdash \;  \bleftBr \word H \bimp \word G \brightBr$}
 	\FalseQuestion{Induktion kann man nur auf Zahlen anwenden.}
\end{frame}

\input{../Bloecke/FormaleSprachen.tex}

% TODO: Im letzten Jahr hat die Zeit nicht gereicht,
% daher diesen Inhalt hier eher streichen.
%\input{Sprache_gesucht_beamer.tex}

\input{../Bloecke/FormaleSprachenP2.tex}

%\input{../Bloecke/Darstellung.tex}

\begin{frame}	
	\begin{block}{Was ihr nun wissen solltet}
		\begin{itemize}
			\item Was eine formale Sprache ist und warum das Konzept wichtig ist
			\item Wie man einfache formale Sprachen formal angeben kann
			\item Einfache Operationen auf formalen Sprachen
			\item Wie man formale Sprachen angeben kann
			\item Wie man Beweise mit formalen Sprachen führt
		\end{itemize}
	\end{block}
	
	\begin{block}{Was nächstes Mal kommt}
		\begin{itemize}
			\item Wie man von Zahlendarstellungen zu Zahlen kommt...
			\item[] ... und wieder zurück
			\item Nicht immer so positiv: Negative Zahlen
			%\item Komprimierung: Huffmann-Codierungen
		\end{itemize}
	\end{block}
\end{frame}

% TODO 
\thassedaniel{
	\xkcdframevert{953}{Danke für eure Aufmerksamkeit! \smiley}{2.5}
}{
	\xkcdframe{1516}{Win by Induction}{2}
}

\end{document}
