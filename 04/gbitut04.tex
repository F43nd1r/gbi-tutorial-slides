%beamer

% Comment/uncomment this line to toggle handout mode
%\newcommand{\handout}{}

%% Beamer-Klasse im korrekten Modus
\ifdefined \handout
\documentclass[handout]{beamer} % Handout mode
\else
\documentclass{beamer}
\fi

%% UTF-8-Encoding
\usepackage[utf8]{inputenc}

% % \bigtimes abgeschrieben von http://tex.stackexchange.com/questions/14386/importing-a-single-symbol-from-a-different-font
% \DeclareFontFamily{U}{mathx}{\hyphenchar\font45}
% \DeclareFontShape{U}{mathx}{m}{n}{
%       <5> <6> <7> <8> <9> <10> gen * mathx
%       <10.95> mathx10 <12> <14.4> <17.28> <20.74> <24.88> mathx12
%       }{}
% \DeclareSymbolFont{mathx}{U}{mathx}{m}{n}
% \DeclareMathSymbol{\bigtimes}{\mathop}{mathx}{161}

\RequirePackage{xcolor}

\def\9{\square}
%\def\9{\blank}

% f"ur Aussagenlogik
\colorlet{alcolor}{blue}
\RequirePackage{tikz}
\usetikzlibrary{arrows.meta}
\newcommand{\alimpl}{\mathrel{\tikz[x={(0.1ex,0ex)},y={(0ex,0.1ex)},>={Classical TikZ Rightarrow[]}]{\draw[alcolor,->,line width=0.7pt,line cap=round] (0,0) -- (15,0);\path (0,-6);}}}
\newcommand{\aleqv}{\mathrel{\tikz[x={(0.1ex,0ex)},y={(0ex,0.1ex)},>={Classical TikZ Rightarrow[]}]{\draw[alcolor,<->,line width=0.7pt,line cap=round] (0,0) -- (18,0);\path (0,-6);}}}
\newcommand{\aland}{\mathbin{\raisebox{-0.6pt}{\rotatebox{90}{\texttt{\color{alcolor}\char62}}}}}
\newcommand{\alor}{\mathbin{\raisebox{-0.8pt}{\rotatebox{90}{\texttt{\color{alcolor}\char60}}}}}
%\newcommand{\ali}[1]{_{\mathtt{\color{alcolor}#1}}}
\newcommand{\alv}[1]{\mathtt{\color{alcolor}#1}}
\newcommand{\alnot}{\mathop{\tikz[x={(0.1ex,0ex)},y={(0ex,0.1ex)}]{\draw[alcolor,line width=0.7pt,line cap=round,line join=round] (0,0) -- (10,0) -- (10,-4);\path (0,-8) ;}}}
\newcommand{\alP}{\alv{P}} %ali{#1}}
%\newcommand{\alka}{\negthinspace\hbox{\texttt{\color{alcolor}(}}}
\newcommand{\alka}{\negthinspace\text{\texttt{\color{alcolor}(}}}
%\newcommand{\alkz}{\texttt{\color{alcolor})}}\negthinspace}
\newcommand{\alkz}{\text{\texttt{\color{alcolor})}}\negthinspace}
\newcommand{\AAL}{A_{AL}}
\newcommand{\LAL}{\hbox{\textit{For}}_{AL}}
\newcommand{\AxAL}{\hbox{\textit{Ax}}_{AL}}
\newcommand{\AxEq}{\hbox{\textit{Ax}}_{Eq}}
\newcommand{\AxPL}{\hbox{\textit{Ax}}_{PL}}
\newcommand{\AALV}{\hbox{\textit{Var}}_{AL}}
\newcommand{\MP}{\hbox{\textit{MP}}}
\newcommand{\GEN}{\hbox{\textit{GEN}}}
\newcommand{\W}{\ensuremath{\hbox{\textbf{w}}}\xspace}
\newcommand{\F}{\ensuremath{\hbox{\textbf{f}}}\xspace}
\newcommand{\WF}{\ensuremath{\{\W,\F\}}\xspace}
\newcommand{\val}{\hbox{\textit{val}}}
\newcommand{\valDIb}{\val_{D,I,\beta}}

\newcommand*{\from}{\colon}

% die nachfolgenden Sachen angepasst an cmtt
\newlength{\ttquantwd}
\setlength{\ttquantwd}{1ex}
\newlength{\ttquantht}
\setlength{\ttquantht}{6.75pt}
\def\plall{%
  \tikz[line width=0.67pt,line cap=round,line join=round,baseline=(B),alcolor] {
    \draw (-0.5\ttquantwd,\ttquantht) -- node[coordinate,pos=0.4] (lll){} (-0.25pt,-0.0pt) -- (0.25pt,-0.0pt) -- node[coordinate,pos=0.6] (rrr){} (0.5\ttquantwd,\ttquantht);
    \draw (lll) -- (rrr);
    \coordinate (B) at (0,-0.35pt);
  }%
}
\def\plexist{%
  \tikz[line width=0.67pt,line cap=round,line join=round,baseline=(B),alcolor] {
    \draw (-0.9\ttquantwd,\ttquantht) -- (0,\ttquantht) -- node[coordinate,pos=0.5] (mmm){} (0,0) --  (-0.9\ttquantwd,0);
    \draw (mmm) -- ++(-0.75\ttquantwd,0);
    \coordinate (B) at (0,-0.35pt);
  }\ensuremath{\,}%
}
\let\plexists=\plexist
\newcommand{\NT}[1]{\ensuremath{\langle\mathrm{#1} \rangle}}

\newcommand{\CPL}{\text{\itshape Const}_{PL}}
\newcommand{\FPL}{\text{\itshape Fun}_{PL}}
\newcommand{\RPL}{\text{\itshape Rel}_{PL}}
\newcommand{\VPL}{\text{\itshape Var}_{PL}}
\newcommand{\ATer}{A_{\text{\itshape Ter}}}
\newcommand{\ARel}{A_{\text{\itshape Rel}}}
\newcommand{\AFor}{A_{\text{\itshape For}}}
\newcommand{\LTer}{L_{\text{\itshape Ter}}}
\newcommand{\LRel}{L_{\text{\itshape Rel}}}
\newcommand{\LFor}{L_{\text{\itshape For}}}
\newcommand{\NTer}{N_{\text{\itshape Ter}}}
\newcommand{\NRel}{N_{\text{\itshape Rel}}}
\newcommand{\NFor}{N_{\text{\itshape For}}}
\newcommand{\PTer}{P_{\text{\itshape Ter}}}
\newcommand{\PRel}{P_{\text{\itshape Rel}}}
\newcommand{\PFor}{P_{\text{\itshape For}}}

\newcommand{\plka}{\alka}
\newcommand{\plkz}{\alkz}
%\newcommand{\plka}{\plfoo{(}}
%\newcommand{\plkz}{\plfoo{)}}
\newcommand{\plcomma}{\hbox{\texttt{\color{alcolor},}}}
\newcommand{\pleq}{{\color{alcolor}\,\dot=\,}}

% MODIFIED (DJ)
% previously: \newcommand{\plfoo}[1]{\mathtt{\color{alcolor}#1}}
\newcommand{\plfoo}[1]{\texttt{\color{alcolor}#1}}

\newcommand{\plc}{\plfoo{c}}
\newcommand{\pld}{\plfoo{d}}
\newcommand{\plf}{\plfoo{f}}
\newcommand{\plg}{\plfoo{g}}
\newcommand{\plh}{\plfoo{h}}
\newcommand{\plx}{\plfoo{x}}
\newcommand{\ply}{\plfoo{y}}
\newcommand{\plz}{\plfoo{z}}
\newcommand{\plR}{\plfoo{R}}
\newcommand{\plS}{\plfoo{S}}

\newcommand{\bv}{\mathrm{bv}}
\newcommand{\fv}{\mathrm{fv}}

%\newcommand{\AxAL}{\hbox{\textit{Ax}}_{AL}}
%\newcommand{\AALV}{\hbox{\textit{Var}}_{AL}}

%\renewcommand{\#}[1]{\literal{#1}}
\newcommand{\A}{\mathcal{A}}
\newcommand{\Adr}{\text{Adr}}
\newcommand{\ar}{\mathrm{ar}}
\newcommand{\ascii}[1]{\literal{\char#1}}
%\newcommand{\assert}[1]{\text{/\!\!/\ } #1}
\newcommand{\assert}[1]{\colorbox{black!7!white}{\ensuremath{\{\;#1\;\}}}}
\newcommand{\Assert}[1]{$\langle$\textit{#1}$\rangle$}
\newcommand{\B}{\mathcal{B}}
\newcommand{\bfmod}{\mathbin{\kw{ mod }}}
\newcommand{\bb}{{\text{bb}}}
\def\bottom{\hbox{\small$\pmb{\bot}$}}
\newcommand{\card}[1]{|#1|}
%\newcommand{\cod}{\mathop{\text{cod}}}  % ist in thwmathabbrevs
\newcommand{\Conf}{\mathcal{C}}
\newcommand{\define}[1]{\emph{#1}}
%\renewcommand{\dh}{d.\,h.\@\xspace}
%\newcommand{\Dh}{D.\,h.\@\xspace}
%\newcommand{\engl}[1]{engl.\xspace\emph{#1}}
\newcommand{\eps}{\varepsilon}
%\newcommand{\evtl}{evtl.\@\xspace}
\newcommand{\fbin}{\text{bin}}
\newcommand{\finv}{\text{inv}}
\newcommand{\fnum}{\text{num}}
\newcommand{\fNum}{{\text{Num}}}
\newcommand{\frepr}{\text{repr}}
\newcommand{\fRepr}{\text{Repr}}
\newcommand{\fZkpl}{\text{Zkpl}}
\newcommand{\fLen}{\text{Len}}
\newcommand{\fsem}{\text{sem}}
\providecommand{\fspace}{\mathord{\text{space}}}
\providecommand{\fSpace}{\mathord{\text{Space}}}
\providecommand{\ftime}{\mathord{\text{time}}}
\providecommand{\fTime}{\mathord{\text{Time}}}
\newcommand{\fTrans}{\text{Trans}}
\newcommand{\fVal}{\text{Val}}

% MODIFIED (DJ)
\newcommand{\Val}{\text{Val}}

%\def\G{\mathbb{Z}}
\newcommand{\HT}[1]{\normalfont\textsc{HT-#1}}
\newcommand{\htr}[3]{\{#1\}\;#2\; \{#3\}}
\newcommand{\Id}{\text{I}}
%\newcommand{\ie}{i.\,e.\@\xspace}
\newcommand{\instr}[2]{\texttt{#1}\ \textit{#2}}
\newcommand{\Instr}[2]{\texttt{#1}\ \textrm{#2}}
\newcommand{\instrr}[3]{\texttt{#1}\ \textit{#2}\texttt{(#3)}}
\newcommand{\Instrr}[3]{\texttt{#1}\ \textrm{#2}\texttt{(#3)}}

% MODIFIED (DJ)
% previously:  \newcommand{\io}{\!\mid\!}
\newcommand{\io}{\ensuremath{\!\mid\!}}

\usepackage{KITcolors}
\newcommand{\literal}[1]{\hbox{\textcolor{blue!95!white}{\textup{\texttt{\scalebox{1.11}{#1}}}}}}
%\newcommand{\literal}[1]{\hbox{\textcolor{KITblue!80!black}{\textup{\texttt{#1}}}}}
\def\kasten#1{\leavevmode\literal{\setlength{\fboxsep}{1pt}\fbox{\vrule  width 0pt height 1.5ex depth 0.5ex #1}}}
\newcommand{\kw}[1]{\ensuremath{\mathbf{#1}}}
\newcommand{\lang}[1]{\ensuremath{\langle#1\rangle}}
%\newcommand{\maw}{m.\,a.\,w.\@\xspace}
%\newcommand{\MaW}{M.\,a.\,w.\@\xspace}
\newcommand{\mdefine}[2][FOOBAR]{\define{#2}\def\foobar{FOOBAR}\def\optarg{#1}\ifx\foobar\optarg\def\optarg{#2}\fi\graffito{\optarg}}
\newcommand{\meins}{\rotatebox[origin=c]{180}{1}}
\newcommand{\Mem}{\text{Mem}}
\newcommand{\memread}{\text{memread}}
\newcommand{\memwrite}{\text{memwrite}}
\providecommand{\meta}[1]{\ensuremath{\langle}\textit{#1}\ensuremath{\rangle}}
%\newcommand{\N}{\mathbb{N}}
\newcommand{\NP}{\mathbf{NP}}
\newcommand{\Nadd}{N_{\text{add}}}
\newcommand{\Nmult}{N_{\text{mult}}}
% MODIFIED (DJ): added \!, mathcal{O}
\newcommand{\Oh}[1]{\mathcal{O}\!\left(#1\right)}
\newcommand{\Om}[1]{\Omega\!\left(#1\right)}
\newcommand{\personname}[1]{\textsc{#1}}
\newcommand{\regname}[1]{\texttt{#1}}
\newcommand{\mima}{\textsc{Mima}\xspace}
\newcommand{\mimax}{\textsc{Mima-X}\xspace}

\def\Pclass{\text{\bfseries P}}
\def\PSPACE{\text{\bfseries PSPACE}}

\newcommand{\SPush}{\text{push}}
\newcommand{\SPop}{\text{pop}}
\newcommand{\SPeek}{\text{peek}}
\newcommand{\STop}{\text{top}}
\newcommand{\STos}{\text{\itshape tos}}
\newcommand{\SBos}{\text{\itshape bos}}

%\newcommand{\R}{\mathbb{R}}
\newcommand{\Rnullplus}{\R_0^{+}}
\newcommand{\Rplus}{\R_{+}}
\newcommand{\resp}{resp.\@\xspace}
\newcommand{\Sem}{\text{Sem}}
\newcommand{\sgn}{\mathop{\text{sgn}}}
\newcommand{\sqbox}{\mathop{\raisebox{-6.2pt}{\hbox{\hbox to 0pt{$^{^{\sqcap}}$\hss}$^{^{\sqcup}}$}}}}
\newcommand{\sqleq}{\sqsubseteq}
\newcommand{\sqgeq}{\sqsupseteq}
% MODIFIED (DJ): added \!
\newcommand{\Th}[1]{\Theta\!\left(#1\right)}
%\newcommand{\usw}{usw.\@\xspace}
\newcommand{\V}[1]{\hbox{\textit{#1}}}
\newcommand{\x}{\times}
\newcommand{\ZK}{\mathbb{K}}
%\newcommand{\Z}{\mathbb{Z}}
\newcommand{\zB}{z.\,B.\@\xspace}
\newcommand{\ZB}{Z.\,B.\@\xspace}
% \newcommand{\bb}{{\text{bb}}}
% \def\##1{\hbox{\textcolor{darkblue}{\texttt{#1}}}}
% \def\A{\mathcal{A}}
% \newcommand{\0}{\#0}
% \newcommand{\1}{\#1}
% \newcommand{\Obj}{\text{Obj}}
% \newcommand{\start}{\mathop{\text{start}}}
% \newcommand{\compactlist}{\addtolength{\itemsep}{-\parskip}}
% \newcommand{\fval}{\text{val}}
% \newcommand{\lang}[1]{\ensuremath{\langle#1\rangle}}
% \newcommand{\io}{\!\mid\!}
% \def\sqbox{\mathop{\raisebox{-6.2pt}{\hbox{\hbox to 0pt{$^{^{\sqcap}}$\hss}$^{^{\sqcup}}$}}}}
% \def\sqleq{\sqsubseteq}
% \def\sqgeq{\sqsupseteq}
\def\Td{T_{\overline{d}}}
% \newcommand{\csym}[1]{\ensuremath{\#{c}_{\#{\hbox{\scriptsize #1}}}}}
% \newcommand{\F}{\ensuremath{\mathcal{F}}}
% \newcommand{\fsym}[2]{\ensuremath{\#{f}^{\#{\hbox{\scriptsize #1}}}_{\#{\hbox{\scriptsize #2}}}}}
% \newcommand{\rsym}[2]{\ensuremath{\#{R}^{\#{\hbox{\scriptsize #1}}}_{\#{\hbox{\scriptsize #2}}}}}
% \newcommand{\xsym}[1]{\ensuremath{\#{x}_{\#{\hbox{\scriptsize #1}}}}}
% \newcommand{\I}{\mathcal{I}}
% ********************************************************************

\usepackage[blue]{../framework/thwregex}
\usepackage{environ}
\usepackage{bm}
\usepackage{calc}
\usepackage{varwidth}
\usepackage{wasysym}
\usepackage{mathtools}

%%%%%%%%%%%%%%%%%%%%%%%%%%%%%%%%%%%% Copied from Style_Tut.tex


% Das ist der KIT-Stil
%\usepackage{../TutTexbib/beamerthemekit}
\usepackage[deutsch,titlepage0]{../framework/KIT/beamerthemeKITmod}
\TitleImage[width=\titleimagewd]{../figures/titlepage.jpg}
%\usetheme[deutsch,titlepage0]{KIT}

% Include PDFs
\usepackage{pdfpages}

% Libertine font (Original GBI font)
\usepackage{libertine}
%\renewcommand*\familydefault{\sfdefault}  %% Only if the base font of the document is to be sans serif

% Nicer math symbols
\usepackage{eulervm}
%\usepackage{mathpazo}
\renewcommand\ttdefault{cmtt} % Computer Modern typewriter font, see lecture slides.

\usepackage{csquotes}

%%%%%%

%% Schönere Schriften
\usepackage[TS1,T1]{fontenc}

%% Bibliothek für Graphiken
\usepackage{graphicx}

%% der wird sowieso in jeder Datei gesetzt
%%\graphicspath{{../figures/}}

%% Anzeigetiefe für Inhaltsverzeichnis: 1 Stufe
\setcounter{tocdepth}{1}

%% Hyperlinks
\usepackage{hyperref}
% I don't know why, but this works and only includes sections and NOT subsections in the pdf-bookmarks.
\hypersetup{bookmarksdepth=subsection} 

%\usepackage{lmodern}
\usepackage{colortbl}
\usepackage[absolute,overlay]{textpos}
\usepackage{listings}
\usepackage{forloop}
%\usepackage{algorithmic} % PseudoCode package 

\usepackage{tikz}
\usetikzlibrary{matrix}
\usetikzlibrary{arrows.meta}
\usetikzlibrary{automata}
\usetikzlibrary{tikzmark}

% Needed for gbi-macros
\usepackage{xspace}

%%%%%%

%% Verbatim
\usepackage{moreverb}

%%%%%%%%%%%%%%%%%%%%%%%%%%%%%%%%%%%% Copy end

%% Tabellen
\usepackage{array}
\usepackage{multicol}

%% Bibliotheken für viele mathematische Symbole
\usepackage{amsmath, amsfonts, amssymb}

%% Deutsche Silbentrennung und Beschriftungen
\usepackage[ngerman]{babel}

\usepackage{kbordermatrix}

% kbordermatrix settings
\renewcommand{\kbldelim}{(} % Left delimiter
\renewcommand{\kbrdelim}{)} % Right delimiter


% This is a configuration file with personal tutor information.
% It is therefore excluded from the git repository, so changes in this file will not conflict in git commits.

% Copy this template, rename to config.tex and add your information below.

\newcommand{\myname}{Lukas Morawietz}
\newcommand{\mymail}{lukas.morawietz@gmail.com} % Consider using your named student mail address to keep your u**** account private.
\newcommand{\mytutnumber}{31}

% Don't forget to update ILIAS url. WARNING: Underscores '_' and Ampersands '&' have to be escaped with backslashes '\'. Blame TeX, not me.
\newcommand{\myILIASurl}{https://ilias.studium.kit.edu/ilias.php?ref\_id=855240\&cmdClass=ilrepositorygui\&cmdNode=5r\&baseClass=ilrepositorygui}

% Uncommenting this will print Socrative info with here defined roomname whenever \Socrative is called.
% (Otherwise, \Socrative will remain silent.)
% \newcommand{\mysocrativeroom}{???}

%\def\ThassesTut{}
\def\DanielsTut{}

\newcommand{\aboutMeFrame}{
	\begin{frame}{Über mich}
		\myname \\
		Informatik, 9. Fachsemester (Bachelor)
		% Lebensgeschichte...
		% Stammbaum...
		% Aufarbeitung der eigenen Todesser-Vergangenheit...
	\end{frame}
}

\def\thisyear{2019}

% Update date of exam
\def\myKlausurtermin{18.~März~2020, 14:00–16:00~Uhr}

\def\mydate#1{
		  \ifnum#1=1\relax	  23. Oktober \thisyear \
	\else \ifnum#1=2\relax	  30. Oktober \thisyear \
	\else \ifnum#1=3\relax    06. November \thisyear \
	\else \ifnum#1=4\relax    13. November \thisyear \
	\else \ifnum#1=5\relax    20. November \thisyear \
	\else \ifnum#1=6\relax    27. November \thisyear \
	\else \ifnum#1=7\relax    04. Dezember \thisyear \
	\else \ifnum#1=8\relax    11. Dezember \thisyear \
	\else \ifnum#1=9\relax    18. Dezember \thisyear \
	\else \ifnum#1=10\relax   08. Januar \nextyear \
	\else \ifnum#1=11\relax   15. Januar \nextyear \
	\else \ifnum#1=12\relax   22. Januar \nextyear \
	\else \ifnum#1=13\relax   29. Januar \nextyear \
	\else \ifnum#1=14\relax   05. Februar \nextyear \
	\else \textbf{Datum undefiniert!} 
	\fi\fi\fi\fi\fi\fi\fi\fi\fi\fi\fi\fi\fi\fi
}

\def\mylasttimestext{Was letztes Mal geschah...}

\colorlet{beamerlightred}{red!40}
\colorlet{beamerlightgreen}{green!50}
\colorlet{beamerlightyellow}{yellow!50}
\colorlet{lightred}{red!30}
\colorlet{lightgreen}{green!40}
\colorlet{lightyellow}{yellow!50}
\colorlet{fullred}{red!60}
\colorlet{fullgreen}{green}

\definecolor{myalertcolor}{rgb}{1,0.33,0.24}
\setbeamercolor{alerted text}{fg=myalertcolor}

% Flag to toggle display of KIT Logo.
% If you want to conform to the official logo guidelines, 
% you are not allowed to use the logo and should disable it
% using the following flag. Just saying.
% (But it's too beautiful, so best leave this commented. :P)
%\newcommand{\noKITLogo}{}

% Toggle handout mode by including the following line before including PraeambelTut
% and removing the % at the start (but do NOT remove the % char here, otherwise handout mode will always be on!)
% Please keep handout mode off in all commits!

% \newcommand{\handout}{}



% define custom \handout command flag if handout mode is toggled  #DirtyAsHellButWell...
\only<beamer:0>{\def\handout{}} %beamer:0 == handout mode

\newcommand{\R}{\mathbb{R}}
\newcommand{\N}{\mathbb{N}}
\newcommand{\Z}{\mathbb{Z}}
\newcommand{\Q}{\mathbb{Q}}
\newcommand{\BB}{\mathbb{B}}
\newcommand{\C}{\mathbb{C}}
\newcommand{\K}{\mathbb{K}}
\newcommand{\G}{\mathbb{G}}
\newcommand{\nullel}{\mathcal{O}}
\newcommand{\einsel}{\mathds{1}}
\newcommand{\Pot}{\mathcal{P}}
\renewcommand{\O}{\text{O}}

\def\word#1{\hbox{\textcolor{blue}{\texttt{#1}}}}
\let\literal\word
\def\mword#1{\hbox{\textcolor{blue}{$\mathtt{#1}$}}}  % math word
\def\sp{\scalebox{1}[.5]{\textvisiblespace}}
\def\wordsp{\word{\sp}}

%\newcommand{\literal}[1]{\textcolor{blue}{\texttt{#1}}}
\newcommand{\realTilde}{\textasciitilde \ }
\newcommand{\setsize}[1]{\ensuremath{\left\lvert #1 \right\rvert}}
\newcommand{\size}[1]{\setsize{#1}}  % Shame on you, TeXStudio...
\newcommand{\set}[1]{\left\{#1\right\}}
\newcommand{\tuple}[1]{\left(#1\right)}
\newcommand{\normalvar}[1]{\text{$#1$}}

% Modified by DJ
\let\oldemptyset\emptyset
\let\emptyset\varnothing % proper emptyset

\newcommand{\boder}{\ensuremath{\mathbin{\textcolor{blue}{\vee}}}\xspace}
\newcommand{\bund}{\ensuremath{\mathbin{\textcolor{blue}{\wedge}}}\xspace}
\newcommand{\bimp}{\ensuremath{\mathrel{\textcolor{blue}{\to}}}\xspace}
\newcommand{\bgdw}{\ensuremath{\mathrel{\textcolor{blue}{\leftrightarrow}}}\xspace}
\newcommand{\bnot}{\ensuremath{\textcolor{blue}{\neg}}\xspace}
\newcommand{\bone}{\ensuremath{\textcolor{blue}{1}}\text{}}
\newcommand{\bzero}{\ensuremath{\textcolor{blue}{0}}\text{}}
\newcommand{\bleftBr}{\ensuremath{\textcolor{blue}{\texttt{(}}}\text{}}
\newcommand{\brightBr}{\ensuremath{\textcolor{blue}{\texttt{)}}}\text{}}

% Fix of \b... commands:

\renewcommand{\boder}{\alor}
\renewcommand{\bund}{\aland}
\renewcommand{\bimp}{\alimpl}
\renewcommand{\bgdw}{\aleqv}
\renewcommand{\bnot}{\alnot}
\renewcommand{\bleftBr}{\alka}
\renewcommand{\brightBr}{\alkz}
\newcommand{\alA}{\word A}
\newcommand{\alB}{\word B}
\newcommand{\alC}{\word C}

\newcommand{\plB}{\plfoo{B}}
\newcommand{\plE}{\plfoo{E}}

\newcommand{\summe}[2]{\sum\limits_{#1}^{#2}}
\newcommand{\limes}[1]{\lim\limits_{#1}}

%\newcommand{\numpp}{\advance \value{weeknum} by -2 \theweeknum \advance \value{weeknum} by 2}
%\newcommand{\nump}{\advance \value{weeknum} by -1 \theweeknum \advance \value{weeknum} by 1}

\newcommand{\mycomment}[1]{}
\newcommand{\Comment}[1]{}

%% DISCLAIMER START 
% It is INSANELY IMPORTANT NOT TO DO THIS OUTSIDE BEAMER CLASS! IN ARTCILE DOCUMENTS, THIS IS VERY LIKELY TO BUG AROUND!
\makeatletter%
\@ifclassloaded{beamer}%
{
	% TODO 
	% no time...
	% redefine section to ignore multiple \section calls with the same title
}%
{
	\errmessage{ERROR: section command redefinition outside of beamer class document! Please contact the author of this code.}
}%
\makeatother%
%% DISCLAIMER END

\newcounter{abc}
\newenvironment{alist}{
  \begin{list}{(\alph{abc})}{
      \usecounter{abc}\setlength{\leftmargin}{8mm}\setlength{\labelsep}{2mm}
    }
}{\end{list}}


\newcommand{\stdarraystretch}{1.20}
\renewcommand{\arraystretch}{\stdarraystretch}  % for proper row spacing in tables

\newcommand{\morescalingdelimiters}{   % for proper \left( \right) typography
	\delimitershortfall=-1pt  
	\delimiterfactor=1
}

\newcommand{\centered}[1]{\vspace{-\baselineskip}\begin{center}#1\end{center}\vspace{-\baselineskip}}

% for \implitem and \item[bla] stuff to look right:
\setbeamercolor*{itemize item}{fg=black}
\setbeamercolor*{itemize subitem}{fg=black}
\setbeamercolor*{itemize subsubitem}{fg=black}

\setbeamercolor*{description item}{fg=black}
\setbeamercolor*{description subitem}{fg=black}
\setbeamercolor*{description subsubitem}{fg=black}

\renewcommand{\qedsymbol}{\textcolor{black}{\openbox}}

\renewcommand{\mod}{\mathop{\textbf{mod}}}
\renewcommand{\div}{\mathop{\textbf{div}}}

\newcommand{\ceil}[1]{\left\lceil#1\right\rceil}
\newcommand{\floor}[1]{\left\lfloor#1\right\rfloor}
\newcommand{\abs}[1]{\left\lvert #1 \right\rvert}
\newcommand{\Matrix}[1]{\begin{pmatrix} #1 \end{pmatrix}}
\newcommand{\braced}[1]{\left\lbrace #1 \right\rbrace}

% "something" placeholder. Useful for repairing spacing of operator sections, like `\sth = 42`.
\def\sth{\vphantom{.}}

\def\fract#1/#2 {\frac{#1}{#2}} % ! Trailing space is crucial!
\def\dfract#1/#2 {\dfrac{#1}{#2}} % ! Trailing space is crucial!

\newcommand{\Mid}{\;\middle|\;}

\let\after\circ



\def\·{\cdot}
\def\*{\cdot}
\def\?>{\ensuremath{\rightsquigarrow}}  % Fuck you, Latex
\def\~~>{\ensuremath{\rightsquigarrow}}  

\newcommand{\tight}[1]{{\renewcommand{\arraystretch}{0.76} #1}}
\newcommand{\stackedtight}[1]{\renewcommand{\arraystretch}{0.76} \begin{matrix} #1 \end{matrix} }
\newcommand{\stacked}[1]{\begin{matrix} #1 \end{matrix} }
\newcommand{\casesl}[1]{\delimitershortfall=0pt  \left\lbrace\hspace{-.3\baselineskip}\begin{array}{ll} #1 \end{array}\right.}
\newcommand{\casesr}[1]{\delimitershortfall=0pt  \left.\begin{array}{ll} #1 \end{array}\hspace{-.3\baselineskip}\right\rbrace}
\newcommand{\caseslr}[1]{\delimitershortfall=0pt  \left\lbrace\hspace{-.3\baselineskip}\begin{array}{ll} #1 \end{array}\hspace{-.3\baselineskip}\right\rbrace}

\def\q#1uad{\ifnum#1=0\relax\else\quad\q{\the\numexpr#1-1\relax}uad\fi}
% e.g. \q1uad = \quad, \q2uad = \qquad etc.

\newcommand{\qqquad}{\q3uad}
\newcommand{\minusquad}{\hspace{-1em}}

%% Placeholder utils
% \§{#1}   Saves #1 as placeholder and prints it
% \.       Prints an \hphantom with the size of the recalled placeholder.
\def\indentstring{}
\def\§#1{\def\indentstring{#1}#1}
\def\.{{$\hphantom{\text{\indentstring}}$}}
%% Placeholder utils end

\newcommand{\impl}{\ifmmode\ensuremath{\mskip\thinmuskip\Rightarrow\mskip\thinmuskip}\else$\Rightarrow$\fi\xspace}
\newcommand{\Impl}{\ifmmode\implies\else$\Longrightarrow$\fi\xspace}

\newcommand{\derives}{\Rightarrow}

\newcommand{\gdw}{\ifmmode\mskip\thickmuskip\Leftrightarrow\mskip\thickmuskip\else$\Leftrightarrow$\fi\xspace}
\newcommand{\Gdw}{\ifmmode\iff\else$\Longleftrightarrow$\fi\xspace}

% Legacy code from the algo tutorial slides. Perhaps useful. Try with care.
\mycomment{
	\newcommand{\impl}{\ifmmode\ensuremath{\mskip\thinmuskip\Rightarrow\mskip\thinmuskip}\else$\Rightarrow$\xspace\fi}  
	\newcommand{\Impl}{\ifmmode\implies\else$\Longrightarrow$\xspace\fi}
	
	\newcommand{\gdw}{\ifmmode\mskip\thickmuskip\Leftrightarrow\mskip\thickmuskip\else$\Leftrightarrow$\xspace\fi}
	\newcommand{\Gdw}{\ifmmode\iff\else$\Longleftrightarrow$\xspace\fi}
}
	
\newcommand{\gdwdef}{\ifmmode\mskip\thickmuskip:\Leftrightarrow\mskip\thickmuskip\else:$\Leftrightarrow$\xspace\fi}
\newcommand{\Gdwdef}{\ifmmode\mskip\thickmuskip:\Longleftrightarrow\mskip\thickmuskip\else:$\Longleftrightarrow$\xspace\fi}

\newcommand{\symbitemnegoffset}{\hspace{-.5\baselineskip}}
\newcommand{\implitem}{\item[\impl\symbitemnegoffset]}
\newcommand{\Implitem}{\item[\Impl\symbitemnegoffset]}


\newcommand{\forcenewline}{\mbox{}\\}

\newcommand{\bfalert}[1]{\textbf{\alert{#1}}}
\let\elem\in   % I'm a Haskell freak. Don't judge me. :P


\def\|#1|{\text{\normalfont #1}}  % | steht für senkrecht (anstatt kursiv wie sonst im math mode)


% proper math typography
\newcommand{\functionto}{\longrightarrow}
\renewcommand{\geq}{\geqslant}
\renewcommand{\leq}{\leqslant}
\let\oldsubset\subset
\renewcommand{\subset}{\subseteq} % for all idiots out there using subset

\newenvironment{threealign}{%
	\[
	\begin{array}{r@{\ }c@{\ }l}
}{%
	\end{array}	
	\]
}

\newcommand{\concludes}{ \\ \hline  }
\newcommand{\deduction}[1]{
	\begin{varwidth}{.8\linewidth}
		\begin{tabular}{>{$}c<{$}}
			#1
		\end{tabular}
	\end{varwidth}	
}

\definecolor{hoareorange}{rgb}{1,.85,.6}
\newcommand{\hoareassert}[1]{\setlength{\fboxsep}{1pt}\setlength{\fboxrule}{-1.4pt}\fcolorbox{white}{hoareorange}{\ensuremath{\{\;#1\;\}}}\setlength\fboxrule{\defaultfboxrule}\setlength{\fboxsep}{3pt}}

\newcommand{\mailto}[1]{\href{mailto:#1}{{\textcolor{blue}{\underline{#1}}}}}
\newcommand{\urlnamed}[2]{\href{#2}{\textcolor{blue}{\underline{#1}}}}
\renewcommand{\url}[1]{\urlnamed{#1}{#1}}

\newcommand{\hanging}{\hangindent=0.7cm}
\newcommand{\indented}{\hanging}


% \hstretchto prints #2 left-aligned into a box of the width of #1
\def\hstretchto#1#2{%
	\mbox{}\vphantom{#2}\rlap{#2}\hphantom{#1}%
}

\def\vstretchto#1#2{%
	\mbox{}\hphantom{#2}\smash{#2}\vphantom{#1}%
}


%requires \thisyear to be defined (s. config.tex)!
\edef\nextyear{\the\numexpr\thisyear+1\relax}


% --- \frameheight constant ---
\newlength\fullframeheight
\newlength\framewithtitleheight
\setlength\fullframeheight{.92\textheight}
\setlength\framewithtitleheight{.86\textheight}

\newlength\frameheight
\setlength\frameheight{\fullframeheight}

\let\frametitleentry\relax
\let\oldframetitle\frametitle
\def\newframetitle#1{\global\def\frametitleentry{#1}\if\relax\frametitleentry\relax\else\setlength\frameheight{\framewithtitleheight}\fi\oldframetitle{#1}}
\let\frametitle\newframetitle

\def\newframetitleoff{\let\frametitle\oldframetitle}
\def\newframetitleon{\let\frametitle\newframetitle}
% --- \frameheight constant end ---

\newcommand{\fakeframetitle}[1]{%
	\vspace{-2.05\baselineskip}%
	{\Large \textbf{#1}} \\%
	\smallskip
}



\newenvironment{headframe}{\Huge THIS IS AN ERROR. PLEASE CONTACT THE ADMIN OF THIS TEX CODE. (headframe env def failed)}{}
\RenewEnviron{headframe}[1][]{
	\begin{frame}\frametitle{\ }
		\centering
		\Huge\textbf{\textsc{\BODY} \\
		}
		\Large {#1}
		\frametitle{\ }
	\end{frame}
}


\makeatletter
% Provides color if undefined.
\newcommand{\colorprovide}[2]{%
	\@ifundefinedcolor{#1}{\colorlet{#1}{#2}}{}}
\makeatother


\colorprovide{lightred}{red!30}
\colorprovide{lightgreen}{green!40}
\colorprovide{lightyellow}{yellow!50}
\colorprovide{lightblue}{blue!10}
\colorprovide{beamerlightred}{lightred}
\colorprovide{beamerlightgreen}{lightgreen}
\colorprovide{beamerlightyellow}{lightyellow}
\colorprovide{beamerlightblue}{lightblue}
\colorprovide{fullred}{red!60}
\colorprovide{fullgreen}{green}
\definecolor{darkred}{RGB}{115,48,38}
\definecolor{darkgreen}{RGB}{48,115,38}
\definecolor{darkyellow}{RGB}{100,100,0}

\only<handout:0>{\colorlet{adaptinglightred}{beamerlightred}}
\only<handout:0>{\colorlet{adaptinglightgreen}{beamerlightgreen}}
\only<handout:0>{\colorlet{adaptinglightyellow}{beamerlightyellow}}
\only<handout:0>{\colorlet{adaptinglightblue}{beamerlightblue}}
\only<beamer:0>{\colorlet{adaptinglightred}{lightred}}
\only<beamer:0>{\colorlet{adaptinglightgreen}{lightgreen}}
\only<beamer:0>{\colorlet{adaptinglightyellow}{lightyellow}}
\only<beamer:0>{\colorlet{adaptinglightblue}{lightblue}}
\only<handout:0>{\colorlet{adaptingred}{lightred}}
\only<beamer:0>{\colorlet{adaptingred}{fullred}}
\only<handout:0>{\colorlet{adaptinggreen}{lightgreen}}
\only<beamer:0>{\colorlet{adaptinggreen}{fullgreen}}



\newcommand{\TrueQuestion}[1]{
	\TrueQuestionE{#1}{}
}

\newcommand{\YesQuestion}[1]{
	\YesQuestionE{#1}{}
}

\newcommand{\FalseQuestion}[1]{
	\FalseQuestionE{#1}{}
}

\newcommand{\NoQuestion}[1]{
	\NoQuestionE{#1}{}
}

\newcommand{\DependsQuestion}[1]{
	\DependsQuestionE{#1}{}
}

\newcommand{\QuestionVspace}{\vspace{4pt}}
\newcommand{\QuestionParbox}[1]{\begin{varwidth}{.85\linewidth}#1\end{varwidth}}
\newcommand{\ExplanationParbox}[1]{\begin{varwidth}{.97\linewidth}#1\end{varwidth}}
\colorlet{questionlightgray}{gray!23}
\let\defaultfboxrule\fboxrule

% #1: bg color
% #2: fg color short answer
% #3: short answer text
% #4: question
% #5: explanation
\newcommand{\GenericQuestion}[5]{
	\setlength\fboxrule{2pt}
	\only<+|handout:0>{\hspace{-2pt}\fcolorbox{white}{questionlightgray}{\QuestionParbox{#4} \quad\textbf{?}}}
	\visible<+->{\hspace{-2pt}\fcolorbox{white}{#1}{\QuestionParbox{#4} \quad\textbf{\textcolor{#2}{#3}}} \if\relax#5\relax\else\ExplanationParbox{#5}\fi} \\
	\setlength\fboxrule{\defaultfboxrule}
}

% #1: Q text
% #2: Explanation
\newcommand{\TrueQuestionE}[2]{
	\GenericQuestion{adaptinglightgreen}{darkgreen}{Wahr.}{#1}{#2}
}

% #1: Q text
% #2: Explanation
\newcommand{\YesQuestionE}[2]{
	\GenericQuestion{adaptinglightgreen}{darkgreen}{Ja.}{#1}{#2}
}

% #1: Q text
% #2: Explanation
\newcommand{\FalseQuestionE}[2]{
	\GenericQuestion{adaptinglightred}{darkred}{Falsch.}{#1}{#2}
}

% #1: Q text
% #2: Explanation
\newcommand{\NoQuestionE}[2]{
	\GenericQuestion{adaptinglightred}{darkred}{Nein.}{#1}{#2}
}

% #1: Q text
% #2: Explanation
\newcommand{\DependsQuestionE}[2]{
	\GenericQuestion{adaptinglightyellow}{darkyellow}{Je nachdem!}{#1}{#2}
}

% #1: Q text
% #2: Answer
\newcommand{\ContentQuestion}[2]{
	\GenericQuestion{adaptinglightblue}{black}{\minusquad}{#1}{#2}
}

\ifnum\thisyear=2018 \else \errmessage{Old ILIAS link inside preamble. Please update.} \fi

\newcommand{\ILIAS}{\urlnamed{ILIAS}{https://ilias.studium.kit.edu/ilias.php?ref\_id=855240\&cmdClass=ilrepositorygui\&cmdNode=5r\&baseClass=ilrepositorygui}\xspace}

\newcommand{\Socrative}{\ifdefined\mysocrativeroom \only<handout:0>{socrative.com $\quad \~~> \quad $ Student login \\ Raumname:  \mysocrativeroom\\ \medskip}\else\fi}

\newcommand{\thasse}[1]{
	\ifdefined\ThassesTut #1\xspace \else\fi
}
\newcommand{\daniel}[1]{
	\ifdefined\DanielsTut #1\xspace \else\fi
}
\newcommand{\thassedaniel}[2]{\ifdefined\ThassesTut #1\else\ifdefined\DanielsTut #2\fi\fi\xspace}

\ifdefined\ThassesTut \ifdefined\DanielsTut \errmessage{ERROR: Both ThassesTut and DanielsTut flags are set. This is most likely an error. Please check your config.tex file.} \else \fi \else \ifdefined\DanielsTut \else \errmessage{ERROR: Neither ThassesTut  nor DanielsTut flags are set. This is most likely an error. Please check your config.tex file.} \fi\fi

%\newcommand{\sgn}{\text{sgn}}

%%%%%%%%%%%% INHALT %%%%%%%%%%%%%%%%

\newcommand{\lastframetitled}[6]{
	\frame{\frametitle{#6}
		\vspace{-#2\baselineskip}
		\begin{figure}[H]
			\centering
			\LARGE \textbf{\textsc{#5}} \\
			\vspace{.2\baselineskip}
			\includegraphics[#1]{#3}
			\vspace{-6pt}
			\begin{center}
				\small \url{#4} 
			\end{center}
		\end{figure} 
	}
}

% #1 number
% #2 title 
% #3 vspace (positive) without unit (\baselineskip)
\newcommand{\xkcdframe}[3]{
	\lastframetitled{width=.96\textwidth}{#3}{xkcd/#1}{http://xkcd.com/#1}{}{#2}
}

\newcommand{\xkcdframevert}[3]
{
	\lastframetitled{height=.96\frameheight}{#3}{xkcd/#1}{http://xkcd.com/#1}{}{#2}
}

% #1 number
% #2 title 
% #3 vspace (positive) without unit (\baselineskip)
% #4 \includegraphics[] optional parameters
\newcommand{\xkcdframecustom}[4]
{
	\lastframetitled{#4}{#3}{xkcd/#1}{http://xkcd.com/#1}{}{#2}
}

\morescalingdelimiters

\begin{document}
\starttut{4}



\framePrevEpisode

\begin{frame}{Rückblick}
	\begin{itemize}
		\item \textbf{Aussagen} sind Sätze, die wahr oder falsch sind
		\item Wir können Aussagen mit \textbf{Konnektiven} zusammenbauen: \\
		$\bund, \boder, \bnot, \bimp$
		\item \textbf{Aussagevariablen} helfen dabei, konkrete Inhalte zu ignorieren 
		\item \textbf{Interpretationen} liefern Wahrheitswerte zu Variablen
		\item $val_I(\*)$ liefert Wahrheitswert für ganze Formel (rekursiv)
	\end{itemize}
\end{frame}

\begin{frame}{Äquivalenz \& Co.}
	\begin{itemize}
		\item $P \leftrightarrow Q \quad\widehat{=}$ Wahr, wenn $P$ und $Q$ äquivalent sind. \\($P, Q$ sind aussagenlogische Variablen oder Formeln)
		\item $P \equiv Q \quad\widehat{=}$ Wir wissen: $P$ und $Q$ sind äquivalent für alle Interpretationen. \\($P, Q$ sind aussagenlogische Formeln)
		\item $P \Leftrightarrow Q \quad\widehat{=}$ $P$ gilt genau dann wenn $Q$ gilt \\($P, Q$ sind Aussagen, verwendet in Beweisen)
		\item $P = Q \quad\widehat{=}$ $P$ ist gleich $Q$ \\($P, Q$ sind Werte, z.B. Zahlen oder Ausdrücke, z.B. $f(x), a + b$)
		\item $P \widehat{=} Q \quad\widehat{=}$ $P$ entspricht $Q$ (informell)
	\end{itemize}
\end{frame}

\begin{frame}[t]{Wahr oder Falsch?}
 	% Socrative: https://b.socrative.com/teacher/#import-quiz/31489145
 	\Socrative
 	
 	\TrueQuestion{Dieser Satz ist eine Aussage.}
 	\FalseQuestionE{$\alA \boder \alB = \alB \boder \alA$}{Das sind syntaktisch verschiedene AL-Formeln!}
 	\TrueQuestion{$\alA \boder \alB \equiv \alB \boder \alA$}
 	% \TrueQuestion{Der AL-Kalkül ist vollständig und korrekt.}  % Wurde noch gar nicht erklärt...!?
 	%\TrueQuestion{Es gibt unendlich viele Axiome im Aussagenkalkül.}
 	%\FalseQuestion{$\bleftBr \word G \bimp \word H \brightBr \; \vdash \;  \bleftBr \word H \bimp \word G \brightBr$}
 	\FalseQuestion{Induktion kann man nur auf Zahlen anwenden.}
\end{frame}

\section{Formale Sprachen}

\begin{frame}
	\frametitle{Rückblick}
	Sei $\Sigma = \{A, B, ..., Z, a, b, ..., z\}$ ein Alphabet.\\
	\pause
	Dann enthält $\Sigma^*$ alle Wörter, die man mit Zeichen aus $\Sigma$ bilden kann. Aber nicht jedes dieser Wörter ist auch sinnvoll.\\[1em]
	
	\enquote{egnarts si efiL} ist kein sinnvolles Wort. \pause Oder? \\[1em]
	\pause
	Wie wir sehen, hängt es immer vom Kontext ab, welche Wörter wir als (syntaktisch) korrekt betrachten.\\
	
\end{frame}

\begin{frame}
	\frametitle{Formale Sprachen}
		\begin{Definition}
			Sei $\Sigma$ ein Alphabet.\\
			Eine formale Sprache $L$ ist eine Teilmenge von $\Sigma^*$.
		\end{Definition}
		\pause
		\vspace{10pt}
		Durch die formale Sprache geben wir an, welche der möglichen Wörter wir als  \emph{syntaktisch korrekt} ansehen.\\
		\pause
		Formale Sprachen werden häufig nicht direkt, sondern über Bildungsvorschriften angegeben.
		
		\pause
		\begin{Beispiel}
			$\Sigma = \{0, 1\}$ \\
			$L = \{ \omega \in \Sigma^* \mid \omega \text{ endet auf } 10 \}  = \{10, 010, 110, 0010, ...\}$
		\end{Beispiel}
\end{frame}

\begin{frame}
	\frametitle{Formale Sprachen}
	
	\begin{Beispiel}
		Formale Sprache $L$ aller Wörter über $A=\{\literal{a},\literal{b}\}$, in denen nirgends das Teilwort $\literal{ab}$ vorkommt.
		\begin{itemize}
			\pause
			\item $L=\{\literal{a},\literal{b}\}^*
			\setminus \{w_1 \cdot \literal{ab} \cdot w_2 \mid w_1,w_2\in
			\{\literal{a},\literal{b}\}^*\}$
			
			\pause
			\item Erst ein beliebiges Wort (evtl. $\varepsilon$) nur aus $\literal{b}$,\\
			danach ein beliebiges Wort (evtl. $\varepsilon$) nur aus $\literal{a}$.
			
			\pause
			\item $L=\{w_1w_2 \mid w_1\in 
			\{\literal{b}\}^*  \text{ und }  w_2\in \{\literal{a}\}^* \}$
		\end{itemize}
	\end{Beispiel}

	\begin{block}{Bemerkungen}
		\begin{itemize}
			\pause
			\item Die Beschreibung einer formalen Sprache ist nicht eindeutig (siehe oben).
			\pause
			\item Immer auf den Unterschied achten: $\{\literal{abc} \} \neq \literal{abc} $
		\end{itemize}
	\end{block}
\end{frame}


\begin{frame}
	\frametitle{Beispiele aus dem Leben}
	\begin{itemize}
		\item Sprache der korrekten IP4-Adressen \pause
		\begin{align*}
		192.168.178.1,\\
		76.147.112.6, ...
		\end{align*} 
		\pause Aber nicht: $000.999.123.666$ \pause
		\item Formale Sprache der Schlüsselwörter in Java $$L = \{ class, int , if, \dots \}$$ \pause
		\item Formale Sprache der legalen Zahlen vom Typ \textbf{int}: Mit $A = \{0...9\}$ 
		\pause $$A \cdot A^* = \{-19, 12849, 1001, 42, ...\}$$ 
		\pause Und minus? Also besser $ \{ -, \varepsilon \} \cdot A \cdot A^*$
	\end{itemize}
\end{frame}


\begin{frame}
	\frametitle{Produkt}
		\begin{Definition}
			Seien $L_1$ und $L_2$ zwei formale Sprachen. Dann bezeichnet
				$$L_1 \cdot L_2 = \{w_1 w_2 \mid w_1 \in L_1 \text{ und } w_2 \in L_2 \}$$
				das \textbf{Produkt} der Sprachen $L_1$ und $L_2$.
		\end{Definition}
		\pause
		In $L_1 \cdot L_2$ sind also alle Wörter enthalten, deren erster Teil aus $L_1$ und deren zweiter Teil aus $L_2$ ist.
	
\end{frame}

\begin{frame}
	\frametitle{Produkt}
	\begin{Beispiele}
		$\{a, b\} \cdot \{c, d\} = \{ac, ad, bc, bd\}$\\[0.3em]
		\pause
		$ L = \{w_1w_2 \mid w_1\in \{\literal{b}\}^* ,  w_2\in \{\literal{a}\}^* \} 
		= \{\literal{b}\}^* \cdot \{\literal{a}\}^* $\\[1em]
		\pause
		Für alle formalen Sprachen $L$ gilt:\\
		$$ L \cdot \{\varepsilon\} = L  \qquad L \cdot \emptyset = \emptyset$$
	\end{Beispiele}
\end{frame}

\begin{frame}
	\frametitle{Potenzen}
	\begin{Definition}
	Damit kann man induktiv die Potenz formaler Sprachen definieren:
	$$L^0 = \{\varepsilon \}$$
	$$L^{i+1} = L^i \cdot L$$
	\end{Definition} \pause
	$L^i$ enthält also alle Kombinationen von $i$ (nicht unbedingt verschiedenen) Wörtern aus $L$.
\end{frame}

\begin{frame}
	\frametitle{Konkatenationsabschluss}
	\begin{Definition}
		Der Konkatenationsabschluss einer formalen Sprache $L$ ist $$L^\ast = \bigcup \limits_{i=0}^\infty L^i$$ 
		\pause
		Der $\varepsilon$-freie Konkatenationsabschluss ist $$L^+ = \bigcup \limits_{i=1}^\infty L^i$$
	\end{Definition} \pause
	Achtung: Der $\varepsilon$-freie Konkatenationsabschluss muss nicht $\varepsilon$-frei sein! \pause
	
	$$ \{\}^* = \{\varepsilon\} $$
\end{frame}


\begin{frame}
	\frametitle{Aufgabe \stars{3}}
	Es sei $A = \{a, b\}$. Beschreiben Sie die folgenden formalen Sprachen mit den Symbolen $\{, \}, a, b,
\varepsilon, \cup, \ast,$ Komma,$ ), ($ und $+$:
	\begin{itemize}
		\item die Menge aller Wörter über $A$, die das Teilwort \code{ab} enthalten.
		\item die Menge aller Wörter über $A$, deren vorletztes Zeichen ein b ist.
		\item die Menge aller Wörter über $A$, in denen nirgends zwei b’s unmittelbar hintereinander vorkommen.
	\end{itemize}
\end{frame}

\begin{frame}
	\frametitle{Lösung}
	\begin{itemize}
		\item \textit{die Menge aller Wörter über $A$, die das Teilwort \code{ab} enthalten.}  \pause
			$$\{a,b\}^\ast \cdot \{ab\} \cdot \{a,b\}^\ast$$ \pause
		\item \textit{die Menge aller Wörter über $A$, deren vorletztes Zeichen ein b ist.}  \pause
			$$\{a,b\}^\ast \cdot \{b\} \cdot \{a,b\}$$ \pause
		\item \textit{die Menge aller Wörter über $A$, in denen nirgends zwei b’s unmittelbar hintereinander vorkommen.}  \pause
			$$\{a, ba\}^\ast \cdot \{b, \varepsilon \}$$
	\end{itemize}
\end{frame}



% TODO: Im letzten Jahr hat die Zeit nicht gereicht,
% daher diesen Inhalt hier eher streichen.
%\def\mycircle{\raisebox{1pt}{\Circle}}

\morescalingdelimiters

\begin{frame}{Zum Aufwärmen: Sprache gesucht!} 
	Haben Alphabet $A = \{ \mword \triangle, \mword \square, \mword \mycircle \}$.\\
\end{frame}

%\begin{frame}
%	\frametitle{Zum Aufwärmen: Sprache gesucht!}
%	
%	Ihr erhaltet eine Karte mit einer Beschreibung einer formalen Sprache über dem Alphabet $\Sigma = \{ \triangle, \square, \circ \}$.\\
%	Ziel ist es, jeweils alle Beschreibungen von gleichen Sprachen zu sammeln.\\
%	Dazu überlegt ihr euch zunächst in 4er-Gruppen, ob ihr Beschreibungen von gleichen Sprachen habt oder was andere Beschreibungen wären. \\
%	Dann tauschen sich jeweils 3 4er-Gruppen aus und sammeln die Beschreibungen.
%\end{frame}

\begin{frame}{Zum Aufwärmen: Sprache gesucht!}
	
	Die Sprache der Wörter, die mit einem Kreis beginnen und danach keinen Kreis mehr enthalten.
	\bigskip
	\pause
	
	$$ \{\mword \mycircle\} \cdot \{\mword \triangle, \mword \square\}^* $$
	\bigskip
	\pause
	
	$$ \set{w \in A^* \mid w = \mword \Circle \cdot v, v \in \{\mword \triangle, \mword \square\}^* } $$

\end{frame}

\begin{frame}{Zum Aufwärmen: Sprache gesucht!}
	
	Die Sprache der Wörter, deren vorletztes Zeichen ein Dreieck ist.
	\bigskip
	\pause
	
	$$ \{\mword \triangle, \mword \square, \mword \mycircle\}^* \cdot \set{\mword \triangle} \cdot \{\mword \triangle, \mword \square, \mword \mycircle\} $$
	\bigskip
	\pause

	$$ \{w \in A^* \mid w = v \cdot \mword \triangle \cdot z, v \in A^*, z \in A \} $$

	
\end{frame}

\begin{frame}{Zum Aufwärmen: Sprache gesucht!}
	Die Sprache der Wörter, in denen nirgends eine ungerade Anzahl an Dreiecken nebeneinander steht.
	\bigskip
	\pause
	$$ \left(\{\mword \square, \mword \mycircle\}^* \cdot \{\mword \triangle\mword \triangle\}^* \right)^* $$
\end{frame}

\begin{frame}{Zum Aufwärmen: Sprache gesucht!}
	
	
	Die Sprache der Wörter, in denen eine gerade Anzahl an Vierecken vorkommt.
	\bigskip
	\pause
	$$ \{\mword \triangle, \mword \mycircle\}^* \cdot \left( \{\mword \square\} \cdot \{\mword \triangle, \mword \mycircle\}^* \cdot \{\mword \square\} \cdot \{\mword \triangle, \mword \mycircle\}^* \right)^* $$
	

\end{frame}

\begin{frame}{Zum Aufwärmen: Sprache gesucht!}
	Die Sprache der Wörter, in denen nirgends zwei Kreise aufeinander folgen.
	\bigskip
	\pause
	$$ \{\mword \square, \mword \triangle\}^* \cdot \left( \{\mword \mycircle\} \cdot \{\mword \square, \mword \triangle\}^+ \right)^* \cdot \{\mword \mycircle, \varepsilon\} $$
	

	
\end{frame}

\section{Formale Sprachen}

\begin{frame}
	\frametitle{Beispiel}
	Sei $A = \{a,b\}$ ein Alphabet. Mit $L$ wollen wir alle Wörter beschreiben, die genau ein $b$ enthalten. \\ \pause
	$$ L = \{w_1 b w_2 \mid w_1, w_2 \in \{a\}^\ast \}  \qquad  L = \{a\}^\ast \cdot \{b \} \cdot \{a\}^\ast$$
	
	Was ist $L^3$? Was enthält $L^i$? \pause
	Zum Beispiel ist $$aaababaaaabaa = aaaba \ baa \ aabaa \in L_3$$ \pause
	$L^i$ enthält alle Wörter, die genau $i$-mal ein $b$ enthalten! \\[1em]
	
	Was enthält $$L^i \setminus \{b\}^\ast$$ \pause
	Alle Wörter, die aus $i$ $b$'s bestehen, aber auch noch mindestens ein $a$ enthalten. \\
\end{frame}

\begin{frame}
	\frametitle{Aufgabe}
	Welche Eigenschaft muss eine formale Sprache $L$ über einem Alphabet
	$A$ erfüllen, damit gilt: $$ L^0 \subseteq L^1 \subseteq L^2 \subseteq L^3 \subseteq ... $$
	
	\pause
	\begin{block}{Lösung}
		Das gilt, wenn $$ \varepsilon \in L $$
	\end{block}
	
\end{frame}

\subsubsection{A1}
\begin{frame}
	\frametitle{Aufgabe (Klausur) \stars{4}}
		\begin{itemize}
			\item Wiederlegen Sie: Für alle formalen Sprachen $L_1 , L_2$ gilt: 
			$$L_1^\ast \cup L_2^\ast = (L_1 \cup L_2 )^\ast$$
			
			\item Zeigen Sie: Für alle formalen Sprachen $L$ gilt: 
				$$L^\ast \cdot L = L^+ $$ 
	\end{itemize}

	Tipp zu 2) (nicht in der Klausur gegeben): Hier handelt es sich um eine Mengengleichheit, also argumentieren wir mit \enquote{$\subseteq$} und \enquote{$\supseteq$}
\end{frame}

\begin{frame}
	\frametitle{Lösung}
	\textit{Für alle formalen Sprachen $L_1 , L_2$ gilt: 
		$$L_1^\ast \cup L_2^\ast = (L_1 \cup L_2 )^\ast$$ } \\[2em] \pause
	Diese Aussage ist falsch: Sei $L_1 = \{a\}$ und $L_2 = \{b\}$. Dann liegt \code{ab} in $(L_1 \cup L_2 )^\ast = \{a, b\}^\ast$ aber nicht in $L_1^\ast \cup L_2^\ast = \{a\}^\ast \cup \{b\}^\ast$.
\end{frame}

\begin{frame}
	\frametitle{Lösung}
	\textit{Für alle formalen Sprachen $L$ gilt: 
		$$L^\ast \cdot L = L^+ $$ } \\[1em] \pause
	Diese Aussage ist wahr! 
	\begin{block}{1. Schritt: $L^\ast \cdot L \subseteq L^+$:} \pause
	Wenn $w \in L^\ast \cdot L$ liegt, dann lässt es sich in Teilwörter auftrennen $$ w = w_1 \cdot w_2$$ mit $w_1 \in L^\ast$ und $w_2 \in L$. Für $w_1$ existiert ein $i \in \N_0$ mit $w_1 \in L^i$. Also $$w = w_1 w_2 \in L^i \cdot L = L^{i+1} \subset L^+$$
	\end{block}
\end{frame}

\begin{frame}
	\frametitle{Lösung}
	\textit{Für alle formalen Sprachen $L$ gilt: 
		$$L^\ast \cdot L = L^+ $$ } \\[1em] 
	Diese Aussage ist wahr! 
	\begin{block}{2. Schritt: $L^\ast \cdot L \supseteq L^+$:} \pause
	Wähle nun $w \in L^+$. Dann existiert ein $i \in \N_+$ mit $w \in L^i$. Da $i > 0$ lässt es sich schreiben als $i = j + 1$ für ein $j \in \N_0$. Also ist $$w \in L^{j+1} = L^j \cdot L \subset L^\ast \cdot L$$
	\end{block}
\end{frame}

\subsubsection{A2}
\begin{frame}
	\frametitle{Aufgabe (WS 2008) \stars{3}}
	Es sei $A = \{a, b\}$. Die Sprache $L \subset A^\ast$ sei definiert durch $$L = (\{a\}^\ast \cdot \{b\} \cdot \{a\}^\ast)^\ast$$
	Zeigen Sie, dass jedes Wort $w$ aus $\{a, b\}^\ast$, das mindestens einmal das Zeichen
	$b$ enthält, in $L$ liegt. (Hinweis: Führen Sie eine Induktion über die Anzahl der
	Vorkommen des Zeichens $b$ in $w$ durch.)
\end{frame}

\begin{frame}
	\frametitle{Lösung}
	$$L = (\{a\}^\ast \cdot \{b\} \cdot \{a\}^\ast)^\ast$$
	Sei $k$ die Anzahl der Vorkommen von $b$ in einem Wort $w \in \{a, b\}^\ast$.
	\begin{block}{Induktionsanfang}  \pause
		Für $k = 1$: In diesem Fall lässt sich das Wort $w$ aufteilen in $$w = w_1 \cdot b \cdot w_2$$ wobei $w_1$ und $w_2$ keine $b$ enthalten und somit in $\{a\}^\ast$ liegen. Damit gilt $w \in \{a\}^\ast \cdot \{b\} \cdot \{a\}^\ast$ und somit auch $$w \in (\{a\}^\ast \cdot \{b\} \cdot \{a\}^\ast)^\ast = L$$
	\end{block}
\end{frame}

\begin{frame}
	\frametitle{Lösung}
	\begin{block}{Induktionsannahme}  \pause
		Für ein festes $k \in \N$ gilt, dass alle Wörter über $\{a, b\}^\ast$, die genau $k$-mal das Zeichen $b$ enthalten, in $L$ liegen.
	\end{block} \pause
	\begin{block}{Induktionsschritt}  \pause
		Wir betrachten ein Wort $w$, das genau $k + 1$ mal das Zeichen $b$ enthält. Dann kann man $w$ zerlegen in $w$ = $w_1 \cdot w_2$, wobei $w_1$ genau einmal das Zeichen $b$ enthält und $w_2$ genau $k$-mal das Zeichen $b$. \pause Nach Induktionsanfang liegt $w_1$ in $\{a\}^\ast \{b\}\{a\}^\ast$. Nach Induktionsvoraussetzung liegt $w_2$ in $(\{a\}^\ast \{b\}\{a\}^\ast )^\ast$, was bedeutet, dass $w = w_1 \cdot w_2$ in $$(\{a\}^\ast \{b\}\{a\}^\ast )(\{a\}^\ast \{b\}\{a\}^\ast )^\ast \subseteq (\{a\}^\ast \{b\}\{a\}^\ast )^\ast = L$$ liegt und die Behauptung ist gezeigt.
	\end{block}
\end{frame}

%\subsubsection{A3}
%\begin{frame}
%	\frametitle{Noch mehr Aufgaben}
%	Begründen oder widerlegen Sie:
%	\begin{itemize}
%		\item Für alle formalen Sprachen $L$ gilt: 
%		$$(L_1^\ast \cdot L_2^\ast)^\ast = (L_1 \cdot L_2)^\ast$$ 
%		
%		\item Für alle formalen Sprachen $L_1 , L_2$ gilt: 
%		$$(L_1^\ast \cup L_2^\ast)^\ast = (L_1 \cup L_2 )^\ast$$
%	\end{itemize}
%\end{frame}
%
%\begin{frame}
%	\frametitle{Lösung}
%	\textit{Für alle formalen Sprachen $L_1 , L_2$ gilt: 
%		$$(L_1^\ast \cdot L_2^\ast )^\ast = (L_1 \cdot L_2 )^\ast$$ } \\[2em] \pause
%	Diese Aussage ist falsch: Sei $L_1 = \{a\}$ und $L_2 = \{b\}$. Dann liegt $$\mathbf{aa} = \mathbf{aa} \cdot \varepsilon$$ in $(L_1^\ast \cdot L_2^\ast ) = (L_1^\ast \cdot L_2^\ast )^1 \subset (L_1^\ast \cdot L_2^\ast )^\ast$, aber nicht in $(L_1 \cdot L_2 )^\ast = \{ab\}^\ast$.
%	
%\end{frame}
%
%\begin{frame}
%	\frametitle{Lösung}
%	\textit{Für alle formalen Sprachen $L_1 , L_2$ gilt: 
%		$$(L_1^\ast \cup L_2^\ast )^\ast = (L_1 \cup L_2 )^\ast$$ } \\[2em] \pause
%	Die Aussage ist korrekt: Sei $w$ ein Wort aus $(L_1^\ast \cup L_2^\ast )^\ast$. Dieses lässt sich in Teilwörter $w_1 , \cdots , w_k$ unterteilen, so dass für $1 \leq i \leq k$ gilt: 
%	$$w_i \in (L_1^\ast \cup L_2^\ast ) \implies w_i \in L_1^\ast \text{ oder } w_i \in L_2^\ast$$
%	Diese Teilwörter $w_i$ lassen sich wieder in Teilwörter $w_{i_1}, \dots w_{i_s}$ zerlegen, die entweder aus $L_1$ kommen, wenn $w_i \in L_1^\ast$ liegt, oder in $L_2$ liegen, wenn $w_i \in L_2^\ast$ liegt. Damit lässt sich $w$ in Teilwörter $w_{i_j}$ aus $L_1 \cup L_2$ unterteilen und es folgt $w \in (L_1 \cup L_2 )^\ast$. 
%\end{frame}
%
%\begin{frame}
%	\frametitle{Lösung}
%	\textit{Für alle formalen Sprachen $L_1 , L_2$ gilt: 
%		$$(L_1^\ast \cup L_2^\ast )^\ast = (L_1 \cup L_2 )^\ast$$ } \\[2em]
%	Sei umgekehrt ein Wort $w$ aus $(L_1 \cup L_2 )^\ast$. Dieses lässt sich dann in Teilwörter $w_1, \dots, w_k$ unterteilen, so dass für $1 \leq i \leq k$ gilt 
%	$$w_i \in L_1 \cup L_2 \implies w_i \in L_1 \subset L_1^\ast \text{ oder } w_i \in L_2 \subset L_2^\ast$$
%	Somit lässt sich $w$ in Teilwörter aus $L_1^\ast \cup L_2^\ast$ unterteilen, und es folgt $w \in (L_1^\ast \cup L_2^\ast )^\ast$.
%\end{frame}

%\section{Von der Darstellung zur Zahl}

\subsection{Definitionen}
\begin{frame}{Numerischer Wert einer Ziffernfolge}
	\begin{Definition}
		Zu einer Zahlenbasis $b$ definiere 
		\begin{align*}
			\fnum_b &\from Z_b \functionto \Z \\
			\fnum_b(x) &:= x \text{ (als Zahl)} \quad \text{für einzelne Ziffern $x \in Z_b$} \\
			& \\
			\fNum_b &\from Z_b^* \functionto \Z \\
			\fNum_b(\eps) &:= 0 \\
			\fNum_b(wx) &:= b\cdot \fNum_b(w) + \fnum_b(x) \quad \text{ für alle } w\in Z_b^\ast, x\in Z_b. 
		\end{align*}
	\end{Definition}

	\pause
	\textbf{Hinweis}: $\fNum_b$ ist eine Abbildung, die einem Wort (Zahlendarstellung) eine Zahl (Wert) zuordnet. \\
	Wir schreiben für die Zahl aber wieder eine Darstellung hin (nämlich im Dezimalsystem).
\end{frame}
\begin{frame}{Aufgabe}
	Berechnet die Zahlenwerte von $ \word{11}_2, \word{321}_4, \word{B2}_{16}$.
	\begin{align*} 
	\fNum_2(\word{11}) &= \visible<2->{2\cdot \fNum_2(\word 1) + \fnum_2(\word 1) \\
	&= 2\cdot 1 + 1 \\
	&= 3  \\}
	\visible<3->{\fNum_4(\word{321}) &=} \visible<4->{ 4\cdot \fNum_4(\word{32}) + \fnum_4(\word 1) \\
	&= 4\cdot \left( 4\cdot \fNum_4(\word 3) + \fnum_4(\word 2) \right) + \fnum_4(\word 1) \\
	&= 4^2\cdot \fnum_4(\word 3) + 4 \cdot \fnum_4(\word 2) + \fnum_4(\word 1) \\
	&= 57 \\}
	\visible<5->{\fNum_{16}(\word{B2}) &=} \visible<6->{ 16 \cdot \fNum_{16}(\word B) + \fnum_{16}(\word 2) \\
	&= 16\cdot 11 + 2 \\
	&= 178}
	\end{align*}

\end{frame}

\mycomment{  % man sieht hier nicht viel...
	\begin{frame}{Wohldefiniertheit}
		\emph{Behauptung}: Die Definition 
			$$ \fNum_b(\eps) = 0  $$  
			$$ \fNum_b(wx) = b\cdot \fNum_b(w) + \fnum_b(x) \text{ für alle } w\in Z_b^\ast, x\in Z_b $$ 
			ist wohldefiniert und weist jedem Wort eine eindeutige Bedeutung zu, die dem Zahlenwert entspricht.
	\end{frame}
	\begin{frame}{Beweis}
		\begin{block}{Beweis durch vollständige Induktion über $n=\vert w \vert $}
		\begin{itemize}
			\only<1-2>{\item<1->[{IA.:}] $n = 0 = \vert w \vert \implies w = \eps $. \\
			Für $w = \eps $ ist $\fNum_b$ wohldefiniert und sinnvoll (nämlich $\fNum_b(\eps) = 0$).
			\item<2->[{IV.:}] Für ein beliebig aber festes $n\in\N_0$ sei $\fNum_b(w)$ für alle $w$ mit $\setsize{w} = n$ wohldefiniert und entspreche dem Zahlenwert. }
			\only<3->{\item<3->[{IS.:}] Wähle $w'$ mit $\vert w' \vert = n+1 $, dann gibt es ein $w\in Z_b^n, x\in Z_b$, so dass $ w' = wx $ \\
			Mit der Definition gilt nun $$ \fNum_b(w') = b\cdot {\underbrace{\fNum_b(w)}_{IV}} + \fnum_b(x) $$
			Die Summe ist laut $IV$ wohldefiniert. Auch ist laut $IV$ $\fNum_b(w)$ der Zahlenwert von $w$ und damit auch $\fNum_b(w')$.}
		\end{itemize}
		\end{block}
	
	\end{frame}
}

%\subsection{Aufgabe}
%\begin{frame}{Aufgabe. WS 2010 }
%Es bezeichne $\Z$ die Menge der ganzen Zahlen. Gegeben sei eine Ziffernmenge $Z_{-2} = \{N, E\}$ mit der Festlegung $num_2 (N) = 0$ und $num_2 (E) = 1$. Wir definieren eine Abbildung $\fNum_{-2} : Z_{-2}^\ast \functionto \Z$ wie folgt:
%	$$\fNum_{-2} (\eps) = 0$$
%	$$\forall \ w \in Z_{-2}^\ast \ \forall \ x \in Z_{-2} : \fNum_{-2} (wx) = -2 \cdot \fNum_{-2} (w) + num_2 ( x )$$
%
%	\begin{itemize}	
%		\item Geben Sie für $w \in \{E, EN, EE, ENE, EEN, EEE\}$ jeweils $\fNum_{-2} (w)$ an.
%		\item Für welche Zahlen $x \in \Z$ gibt es ein $w \in Z_{-2}^\ast$ mit $\fNum_{-2} (w) = x$?
%	\end{itemize}
%\end{frame}
%
%\begin{frame}{Lösung}
%\textit{Geben Sie für $w \in \{E, EN, EE, ENE, EEN, EEE\}$ jeweils $\fNum_{-2} (w)$ an.} \pause
%	\begin{table}[h!]	
%		\begin{tabular}{>{$}l<{$}>{$}l<{$}}
%			\fNum_{-2} (E)\pause & = 1 \\ \pause 
%			\fNum_{-2} (EN)\pause & = -2 \\ \pause
%			\fNum_{-2} (EE)\pause & = -1 \\ \pause
%			\fNum_{-2} (ENE)\pause & = 5 \\ \pause
%			\fNum_{-2} (EEN)\pause & = 2 \\ \pause
%			\fNum_{-2} (EEE)\pause & = 3
%	\end{tabular}
%	\end{table}
%	\pause
%	\textit{Für welche Zahlen $x \in \Z$ gibt es ein $w \in Z_{-2}^\ast$ mit $\fNum_{-2} (w) = x$?} \\[1em]\pause
%	Für alle!
%\end{frame}

\section{Von der Zahl zur Darstellung}
\begin{frame}{Division und Modulo}
	\begin{block}{Definition}
		$ x \div y$ ist die ganzzahlige Division von x durch y.\\
		$ x \mod y$ liefert den Rest dieser Division.
	\end{block} 
	\pause
	
	\begin{block}{Beobachtung}
		$ x\div y \in \N_0, \qquad x\mod y \in \{0,\dots, y-1\} $
	\end{block}
	\pause
	
	\begin{Beispiel}
		\begin{tabular}{c|cccc|cccc}
			$y$ & \multicolumn{4}{c|}{2} & \multicolumn{4}{c}{3} \\
			$x$ & 1 & 2 & 5 & 8 & 1 & 2 & 5 & 8 \\ \pause
			$x\div y$ & 0 & 1 & 2 & 4 & 0 & 0 & 1 & 2 \\
			$x\mod y$ & 1 & 0 & 1 & 0 & 1 & 2 & 2 & 2 \\
		\end{tabular}
	\end{Beispiel}
	
\end{frame}

\begin{frame}{Division und Modulo}

	\begin{block}{Lemma}
		$$ x = y \cdot (x \div y ) + \left( x \mod y \right)$$ 
	\end{block}

	\begin{Beispiel}
		\begin{table}[h!]
			\centering
			\begin{tabular}{c|cccccccccccc}
				$x$ & 0 & 1 & 2 & 3 & 4 & 5 & 6 & 7 & 8 & 9 & 10 & 11 \\ \hline
				$x\div 4 $ & \only<2->{0 & 0 & 0 & 0 & 1 & 1 & 1 & 1 & 2 & 2 & 2 & 2 } \only<1|handout:0>{&&&&&&&&&&&} \\
				$x\mod 4$ & \only<3->{0&1&2&3&0&1&2&3&0&1&2&3} \only<1-2|handout:0>{&} \\
				$4 \· \left( x\div 4\right) $ & \only<4->{0&0&0&0&4&4&4&4&8&8&8&8} \only<1-3|handout:0>{&&&&&&&&&&}  \\ \hline
				$4 \· \left( x\div 4\right) + x \mod 4 $ & \only<5->{0 & 1 & 2 & 3 & 4 & 5 & 6 & 7 & 8 & 9 & 10 & 11} \only<1-4|handout:0>{&&&&&&&&&&}
			\end{tabular}
		\end{table}
	\end{Beispiel}

	\visible<5-> {$ \Impl$ \enquote{Beweis durch Beispiel} :P \qed}

\end{frame}

\begin{frame}{Repräsentation von Zahlen}
	Wir definieren
	\begin{threealign}
	\fRepr_k \from \; \N_0 &\functionto& Z_k,  \\
	n &\mapsto& \begin{cases} \frepr_k(n), & n<k \\ \fRepr_k\left( n\div k \right) \cdot \frepr_k\left( n \mod k \right), & n\geq k 
	\end{cases}.
	\end{threealign}
	\pause
	\begin{block}{Es gilt}
		$\fRepr_k(n)$ ist das kürzeste Wort $w\in Z_k^\ast$ mit $\fNum_k(w)=n$, also 
		$$ \fNum_k\left( \fRepr_k(n)\right) = n. $$ 
	\end{block}
	\pause
	\emph{Anmerkung}:
	Im Allgemeinen ist $$ \fRepr_k\left(\fNum_k(w)\right) \neq w, $$ da überflüssige Nullen in $w$ wegfallen können. 
\end{frame}

\begin{frame}{Repräsentation von Zahlen}
	\begin{threealign}
	\fRepr_k : \; \N_0 &\functionto& Z_k,  \\
	n &\mapsto& \begin{cases} \frepr_k(n), & n<k \\ \fRepr_k\left( n\div k \right) \cdot \frepr_k\left( n \mod k \right), & n\geq k 
	\end{cases}
	\end{threealign}
	
	\begin{block}{Aufgabe}
		Berechnet folgende Darstellungen:\\
		$\fRepr_2(42) = \only<2->{\word{101010}}$ \\
		$\fRepr_4(42) = \only<3->{\word{222}}$ \\
		$\fRepr_8(42) = \only<4->{\word{52}}$ \\
		$\fRepr_{16}(42) = \only<5->{\word{2A}}$
	\end{block}
\end{frame}

\begin{frame}{Beispiel: ausführl. Lösung}
	\begin{align*}
		\fRepr_8(42) &= \fRepr_8(42 \div 8) \cdot \frepr_8(42 \mod 8) \\
		&= \fRepr_8(5) \cdot \frepr_8(2)\\
		&= \frepr_8(5) \cdot \word 2\\
		&= \word 5 \cdot \word 2\\
		&= \word{52}_8\thassedaniel{}{.}
	\end{align*}
	
\end{frame}


\begin{frame}	
	\begin{block}{Was ihr nun wissen solltet}
		\begin{itemize}
			\item Was eine formale Sprache ist und warum das Konzept wichtig ist
			\item Wie man einfache formale Sprachen formal angeben kann
			\item Einfache Operationen auf formalen Sprachen
			\item Wie man formale Sprachen angeben kann
			\item Wie man Beweise mit formalen Sprachen führt
		\end{itemize}
	\end{block}
	
	\begin{block}{Was nächstes Mal kommt}
		\begin{itemize}
			\item Wie man von Zahlendarstellungen zu Zahlen kommt...
			\item[] ... und wieder zurück
			\item Nicht immer so positiv: Negative Zahlen
			%\item Komprimierung: Huffmann-Codierungen
		\end{itemize}
	\end{block}
\end{frame}

% TODO 
\thassedaniel{
	\xkcdframevert{953}{Danke für eure Aufmerksamkeit! \smiley}{2.5}
}{
	\xkcdframe{1516}{Win by Induction}{2}
}

\end{document}
