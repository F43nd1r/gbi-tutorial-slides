%beamer

% TODO:
% - VL-Folien einarbeiten
% - Weniger f_* etc. Theorie, mehr Aufgaben zwischendrin und zu Akzeptoren
% - Bessere Hinleitung zu Akzeptoren

% Comment/uncomment this line to toggle handout mode
%\newcommand{\handout}{}

\input{../framework/PraeambelTut.tex}

\morescalingdelimiters

\begin{document}
\starttut{12}


\framePrevEpisode

\begin{frame}{Rückblick: Laufzeitabschätzungen}
	\begin{itemize}[<+->]
		\item Asymptotisches Wachstum
		\item $O, \Theta, \Omega$
		\item Beweisverfahren
		\item Rechenregeln im O-Kalkül
	\end{itemize}

	\pause
	\begin{block}{Typische Laufzeiten}
		Lineare Suche: $\Th{n}$ \\
		Binäre Suche:  $\Th{\log n}$ \\
		Potenzmenge berechnen: $\Th{2^n}$
	\end{block}
\end{frame}

\begin{frame}{Rückblick}
	\centering
	\includegraphics[scale=0.5]{laufzeit/polyVsExp}
\end{frame}

\begin{frame}[t]{Wahr oder Falsch?}
	\TrueQuestionE{$x^4 \in \Oh{{(x^3)}^3}$}{ $ {(x^3)}^3 = x^9$}
	\FalseQuestionE{$\sqrt{n} \in \Om{2^n}$}{}
	\TrueQuestionE{$log_{5000} n \in \Th{\log_2{n^4}}$}{ $ \log_2{n^4} = 4 \cdot \log_2{n}$}
	%
	% ACHTUNG: Sehr aufwendig zu erklären. Besser weglassen, um Zeit zu sparen!
	\FalseQuestionE{$O(f_1) + f_2 = O(f_1 + f_2)$}{Z.~B.: \\ $\Oh{n} + 4 = \set{f(n) + 4 \Mid f(n) \in \Oh{n}} \neq \set{f(n) \Mid f(n) \in \Oh{n}} = \Oh{n+4}$. \\ \textbf{Aber} es gilt: $O(f_1) + O(f_2) = O(f_1 + f_2)$}
\end{frame}

\begin{frame}[t]{Wahr oder Falsch?}
	\FalseQuestion{Für zwei Funktionen $f, g$ gilt immer $f \preceq g$ oder $f \succeq g$.}
	\medskip
	
	\visible<2>{
		Es gibt unvergleichbare Funktionen! Beispiel:
		\begin{align*}
		f(n) &=
		\begin{cases}
		1, & \text{ falls $n$ gerade} \\
		n, & \text{ falls $n$ ungerade} \\
		\end{cases} \\
		g(n) &=
		\begin{cases}
		n, & \text{ falls $n$ gerade} \\
		1, & \text{ falls $n$ ungerade} \\
		\end{cases} \\
		\end{align*}
		Es gilt \textbf{nicht} $g\preceq f$, es gilt \textbf{nicht} $f\preceq
		g$ und es gilt \textbf{erst recht nicht} $f\asymp g$.
	}
\end{frame}



\input{../Bloecke/Automaten}

\input{../Bloecke/Akzeptoren}

% ---------

% WiSe 10/11 Aufgabe 6 c 
\begin{frame}{Übung: Akzeptoren}
	Die Sprache $L\subseteq \{\word a,\word b\}^\ast $ sei definiert als die Menge aller Wörter $w$, die folgende Bedingungen erfüllen:
	\begin{align*}
	N_{\word b}(w) &> N_{\word a}(w)\\ 
	\forall v_1,v_2 \in \{\word a,\word b\}^\ast : \qquad w &\neq v_1 \word{bb} v_2 
	\end{align*}
	
	Gebt einen endlichen Akzeptor an, der $L$ erkennt. \\
	
	\bigskip
	\pause
	\begin{block}{Tipps}
		\begin{itemize}[<+->]
			\item Die zweite Bedingung bedeutet: Das Wort darf nirgends zwei \word b hintereinander enthalten.
			\item Was passiert, wenn das Wort mit einem \word a beginnt?\\
			Kann das Wort noch akzeptiert werden?
		\end{itemize}
	\end{block}
\end{frame}

\begin{frame}{Übung: Akzeptoren: Lösung}
	\begin{figure}
		\centering
		\includegraphics[width=0.7\linewidth]{automaten/Loesung2.pdf}
	\end{figure}
\end{frame}

\input{../Bloecke/Turing}

\begin{frame}	
	\begin{block}{Was ihr nun wissen solltet}
		\begin{itemize}
			\item Endlich: Automaten!
			\item ... und wie man damit Wörter akzeptiert/ablehnt
			\item Turingmaschinen
		\end{itemize}
	\end{block}
	
	\begin{block}{Was die nächsten Male kommt}
		\begin{itemize}
			\item Nicht immer so vulgär: Reguläre Ausdrücke
			\item Rechtslineare Grammatiken
			%\item Turingmaschinen -- mächtiger wird es nicht mehr!
		\end{itemize}
	\end{block}
\end{frame}


\xkcdframevert{1319}{Danke für eure Aufmerksamkeit! \smiley}{2.5}
%\lastframe{0.6}{30}{xkcd/automation_1319.png}{http://www.xkcd.com/1319}
%\xkcdframe{0.5}{30}{xkcd/houston_1438.png}{http://www.xkcd.com/1438}{Oh, hi mom. No, nothing important, just work.}



%\slideThanks

\end{document}