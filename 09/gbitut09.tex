%beamer

% Comment/uncomment this line to toggle handout mode
% \newcommand{\handout}{}

%% Beamer-Klasse im korrekten Modus
\ifdefined \handout
\documentclass[handout]{beamer} % Handout mode
\else
\documentclass{beamer}
\fi

%% UTF-8-Encoding
\usepackage[utf8]{inputenc}

% % \bigtimes abgeschrieben von http://tex.stackexchange.com/questions/14386/importing-a-single-symbol-from-a-different-font
% \DeclareFontFamily{U}{mathx}{\hyphenchar\font45}
% \DeclareFontShape{U}{mathx}{m}{n}{
%       <5> <6> <7> <8> <9> <10> gen * mathx
%       <10.95> mathx10 <12> <14.4> <17.28> <20.74> <24.88> mathx12
%       }{}
% \DeclareSymbolFont{mathx}{U}{mathx}{m}{n}
% \DeclareMathSymbol{\bigtimes}{\mathop}{mathx}{161}

\RequirePackage{xcolor}

\def\9{\square}
%\def\9{\blank}

% f"ur Aussagenlogik
\colorlet{alcolor}{blue}
\RequirePackage{tikz}
\usetikzlibrary{arrows.meta}
\newcommand{\alimpl}{\mathrel{\tikz[x={(0.1ex,0ex)},y={(0ex,0.1ex)},>={Classical TikZ Rightarrow[]}]{\draw[alcolor,->,line width=0.7pt,line cap=round] (0,0) -- (15,0);\path (0,-6);}}}
\newcommand{\aleqv}{\mathrel{\tikz[x={(0.1ex,0ex)},y={(0ex,0.1ex)},>={Classical TikZ Rightarrow[]}]{\draw[alcolor,<->,line width=0.7pt,line cap=round] (0,0) -- (18,0);\path (0,-6);}}}
\newcommand{\aland}{\mathbin{\raisebox{-0.6pt}{\rotatebox{90}{\texttt{\color{alcolor}\char62}}}}}
\newcommand{\alor}{\mathbin{\raisebox{-0.8pt}{\rotatebox{90}{\texttt{\color{alcolor}\char60}}}}}
%\newcommand{\ali}[1]{_{\mathtt{\color{alcolor}#1}}}
\newcommand{\alv}[1]{\mathtt{\color{alcolor}#1}}
\newcommand{\alnot}{\mathop{\tikz[x={(0.1ex,0ex)},y={(0ex,0.1ex)}]{\draw[alcolor,line width=0.7pt,line cap=round,line join=round] (0,0) -- (10,0) -- (10,-4);\path (0,-8) ;}}}
\newcommand{\alP}{\alv{P}} %ali{#1}}
%\newcommand{\alka}{\negthinspace\hbox{\texttt{\color{alcolor}(}}}
\newcommand{\alka}{\negthinspace\text{\texttt{\color{alcolor}(}}}
%\newcommand{\alkz}{\texttt{\color{alcolor})}}\negthinspace}
\newcommand{\alkz}{\text{\texttt{\color{alcolor})}}\negthinspace}
\newcommand{\AAL}{A_{AL}}
\newcommand{\LAL}{\hbox{\textit{For}}_{AL}}
\newcommand{\AxAL}{\hbox{\textit{Ax}}_{AL}}
\newcommand{\AxEq}{\hbox{\textit{Ax}}_{Eq}}
\newcommand{\AxPL}{\hbox{\textit{Ax}}_{PL}}
\newcommand{\AALV}{\hbox{\textit{Var}}_{AL}}
\newcommand{\MP}{\hbox{\textit{MP}}}
\newcommand{\GEN}{\hbox{\textit{GEN}}}
\newcommand{\W}{\ensuremath{\hbox{\textbf{w}}}\xspace}
\newcommand{\F}{\ensuremath{\hbox{\textbf{f}}}\xspace}
\newcommand{\WF}{\ensuremath{\{\W,\F\}}\xspace}
\newcommand{\val}{\hbox{\textit{val}}}
\newcommand{\valDIb}{\val_{D,I,\beta}}

\newcommand*{\from}{\colon}

% die nachfolgenden Sachen angepasst an cmtt
\newlength{\ttquantwd}
\setlength{\ttquantwd}{1ex}
\newlength{\ttquantht}
\setlength{\ttquantht}{6.75pt}
\def\plall{%
  \tikz[line width=0.67pt,line cap=round,line join=round,baseline=(B),alcolor] {
    \draw (-0.5\ttquantwd,\ttquantht) -- node[coordinate,pos=0.4] (lll){} (-0.25pt,-0.0pt) -- (0.25pt,-0.0pt) -- node[coordinate,pos=0.6] (rrr){} (0.5\ttquantwd,\ttquantht);
    \draw (lll) -- (rrr);
    \coordinate (B) at (0,-0.35pt);
  }%
}
\def\plexist{%
  \tikz[line width=0.67pt,line cap=round,line join=round,baseline=(B),alcolor] {
    \draw (-0.9\ttquantwd,\ttquantht) -- (0,\ttquantht) -- node[coordinate,pos=0.5] (mmm){} (0,0) --  (-0.9\ttquantwd,0);
    \draw (mmm) -- ++(-0.75\ttquantwd,0);
    \coordinate (B) at (0,-0.35pt);
  }\ensuremath{\,}%
}
\let\plexists=\plexist
\newcommand{\NT}[1]{\ensuremath{\langle\mathrm{#1} \rangle}}

\newcommand{\CPL}{\text{\itshape Const}_{PL}}
\newcommand{\FPL}{\text{\itshape Fun}_{PL}}
\newcommand{\RPL}{\text{\itshape Rel}_{PL}}
\newcommand{\VPL}{\text{\itshape Var}_{PL}}
\newcommand{\ATer}{A_{\text{\itshape Ter}}}
\newcommand{\ARel}{A_{\text{\itshape Rel}}}
\newcommand{\AFor}{A_{\text{\itshape For}}}
\newcommand{\LTer}{L_{\text{\itshape Ter}}}
\newcommand{\LRel}{L_{\text{\itshape Rel}}}
\newcommand{\LFor}{L_{\text{\itshape For}}}
\newcommand{\NTer}{N_{\text{\itshape Ter}}}
\newcommand{\NRel}{N_{\text{\itshape Rel}}}
\newcommand{\NFor}{N_{\text{\itshape For}}}
\newcommand{\PTer}{P_{\text{\itshape Ter}}}
\newcommand{\PRel}{P_{\text{\itshape Rel}}}
\newcommand{\PFor}{P_{\text{\itshape For}}}

\newcommand{\plka}{\alka}
\newcommand{\plkz}{\alkz}
%\newcommand{\plka}{\plfoo{(}}
%\newcommand{\plkz}{\plfoo{)}}
\newcommand{\plcomma}{\hbox{\texttt{\color{alcolor},}}}
\newcommand{\pleq}{{\color{alcolor}\,\dot=\,}}

% MODIFIED (DJ)
% previously: \newcommand{\plfoo}[1]{\mathtt{\color{alcolor}#1}}
\newcommand{\plfoo}[1]{\texttt{\color{alcolor}#1}}

\newcommand{\plc}{\plfoo{c}}
\newcommand{\pld}{\plfoo{d}}
\newcommand{\plf}{\plfoo{f}}
\newcommand{\plg}{\plfoo{g}}
\newcommand{\plh}{\plfoo{h}}
\newcommand{\plx}{\plfoo{x}}
\newcommand{\ply}{\plfoo{y}}
\newcommand{\plz}{\plfoo{z}}
\newcommand{\plR}{\plfoo{R}}
\newcommand{\plS}{\plfoo{S}}

\newcommand{\bv}{\mathrm{bv}}
\newcommand{\fv}{\mathrm{fv}}

%\newcommand{\AxAL}{\hbox{\textit{Ax}}_{AL}}
%\newcommand{\AALV}{\hbox{\textit{Var}}_{AL}}

%\renewcommand{\#}[1]{\literal{#1}}
\newcommand{\A}{\mathcal{A}}
\newcommand{\Adr}{\text{Adr}}
\newcommand{\ar}{\mathrm{ar}}
\newcommand{\ascii}[1]{\literal{\char#1}}
%\newcommand{\assert}[1]{\text{/\!\!/\ } #1}
\newcommand{\assert}[1]{\colorbox{black!7!white}{\ensuremath{\{\;#1\;\}}}}
\newcommand{\Assert}[1]{$\langle$\textit{#1}$\rangle$}
\newcommand{\B}{\mathcal{B}}
\newcommand{\bfmod}{\mathbin{\kw{ mod }}}
\newcommand{\bb}{{\text{bb}}}
\def\bottom{\hbox{\small$\pmb{\bot}$}}
\newcommand{\card}[1]{|#1|}
%\newcommand{\cod}{\mathop{\text{cod}}}  % ist in thwmathabbrevs
\newcommand{\Conf}{\mathcal{C}}
\newcommand{\define}[1]{\emph{#1}}
%\renewcommand{\dh}{d.\,h.\@\xspace}
%\newcommand{\Dh}{D.\,h.\@\xspace}
%\newcommand{\engl}[1]{engl.\xspace\emph{#1}}
\newcommand{\eps}{\varepsilon}
%\newcommand{\evtl}{evtl.\@\xspace}
\newcommand{\fbin}{\text{bin}}
\newcommand{\finv}{\text{inv}}
\newcommand{\fnum}{\text{num}}
\newcommand{\fNum}{{\text{Num}}}
\newcommand{\frepr}{\text{repr}}
\newcommand{\fRepr}{\text{Repr}}
\newcommand{\fZkpl}{\text{Zkpl}}
\newcommand{\fLen}{\text{Len}}
\newcommand{\fsem}{\text{sem}}
\providecommand{\fspace}{\mathord{\text{space}}}
\providecommand{\fSpace}{\mathord{\text{Space}}}
\providecommand{\ftime}{\mathord{\text{time}}}
\providecommand{\fTime}{\mathord{\text{Time}}}
\newcommand{\fTrans}{\text{Trans}}
\newcommand{\fVal}{\text{Val}}

% MODIFIED (DJ)
\newcommand{\Val}{\text{Val}}

%\def\G{\mathbb{Z}}
\newcommand{\HT}[1]{\normalfont\textsc{HT-#1}}
\newcommand{\htr}[3]{\{#1\}\;#2\; \{#3\}}
\newcommand{\Id}{\text{I}}
%\newcommand{\ie}{i.\,e.\@\xspace}
\newcommand{\instr}[2]{\texttt{#1}\ \textit{#2}}
\newcommand{\Instr}[2]{\texttt{#1}\ \textrm{#2}}
\newcommand{\instrr}[3]{\texttt{#1}\ \textit{#2}\texttt{(#3)}}
\newcommand{\Instrr}[3]{\texttt{#1}\ \textrm{#2}\texttt{(#3)}}

% MODIFIED (DJ)
% previously:  \newcommand{\io}{\!\mid\!}
\newcommand{\io}{\ensuremath{\!\mid\!}}

\usepackage{KITcolors}
\newcommand{\literal}[1]{\hbox{\textcolor{blue!95!white}{\textup{\texttt{\scalebox{1.11}{#1}}}}}}
%\newcommand{\literal}[1]{\hbox{\textcolor{KITblue!80!black}{\textup{\texttt{#1}}}}}
\def\kasten#1{\leavevmode\literal{\setlength{\fboxsep}{1pt}\fbox{\vrule  width 0pt height 1.5ex depth 0.5ex #1}}}
\newcommand{\kw}[1]{\ensuremath{\mathbf{#1}}}
\newcommand{\lang}[1]{\ensuremath{\langle#1\rangle}}
%\newcommand{\maw}{m.\,a.\,w.\@\xspace}
%\newcommand{\MaW}{M.\,a.\,w.\@\xspace}
\newcommand{\mdefine}[2][FOOBAR]{\define{#2}\def\foobar{FOOBAR}\def\optarg{#1}\ifx\foobar\optarg\def\optarg{#2}\fi\graffito{\optarg}}
\newcommand{\meins}{\rotatebox[origin=c]{180}{1}}
\newcommand{\Mem}{\text{Mem}}
\newcommand{\memread}{\text{memread}}
\newcommand{\memwrite}{\text{memwrite}}
\providecommand{\meta}[1]{\ensuremath{\langle}\textit{#1}\ensuremath{\rangle}}
%\newcommand{\N}{\mathbb{N}}
\newcommand{\NP}{\mathbf{NP}}
\newcommand{\Nadd}{N_{\text{add}}}
\newcommand{\Nmult}{N_{\text{mult}}}
% MODIFIED (DJ): added \!, mathcal{O}
\newcommand{\Oh}[1]{\mathcal{O}\!\left(#1\right)}
\newcommand{\Om}[1]{\Omega\!\left(#1\right)}
\newcommand{\personname}[1]{\textsc{#1}}
\newcommand{\regname}[1]{\texttt{#1}}
\newcommand{\mima}{\textsc{Mima}\xspace}
\newcommand{\mimax}{\textsc{Mima-X}\xspace}

\def\Pclass{\text{\bfseries P}}
\def\PSPACE{\text{\bfseries PSPACE}}

\newcommand{\SPush}{\text{push}}
\newcommand{\SPop}{\text{pop}}
\newcommand{\SPeek}{\text{peek}}
\newcommand{\STop}{\text{top}}
\newcommand{\STos}{\text{\itshape tos}}
\newcommand{\SBos}{\text{\itshape bos}}

%\newcommand{\R}{\mathbb{R}}
\newcommand{\Rnullplus}{\R_0^{+}}
\newcommand{\Rplus}{\R_{+}}
\newcommand{\resp}{resp.\@\xspace}
\newcommand{\Sem}{\text{Sem}}
\newcommand{\sgn}{\mathop{\text{sgn}}}
\newcommand{\sqbox}{\mathop{\raisebox{-6.2pt}{\hbox{\hbox to 0pt{$^{^{\sqcap}}$\hss}$^{^{\sqcup}}$}}}}
\newcommand{\sqleq}{\sqsubseteq}
\newcommand{\sqgeq}{\sqsupseteq}
% MODIFIED (DJ): added \!
\newcommand{\Th}[1]{\Theta\!\left(#1\right)}
%\newcommand{\usw}{usw.\@\xspace}
\newcommand{\V}[1]{\hbox{\textit{#1}}}
\newcommand{\x}{\times}
\newcommand{\ZK}{\mathbb{K}}
%\newcommand{\Z}{\mathbb{Z}}
\newcommand{\zB}{z.\,B.\@\xspace}
\newcommand{\ZB}{Z.\,B.\@\xspace}
% \newcommand{\bb}{{\text{bb}}}
% \def\##1{\hbox{\textcolor{darkblue}{\texttt{#1}}}}
% \def\A{\mathcal{A}}
% \newcommand{\0}{\#0}
% \newcommand{\1}{\#1}
% \newcommand{\Obj}{\text{Obj}}
% \newcommand{\start}{\mathop{\text{start}}}
% \newcommand{\compactlist}{\addtolength{\itemsep}{-\parskip}}
% \newcommand{\fval}{\text{val}}
% \newcommand{\lang}[1]{\ensuremath{\langle#1\rangle}}
% \newcommand{\io}{\!\mid\!}
% \def\sqbox{\mathop{\raisebox{-6.2pt}{\hbox{\hbox to 0pt{$^{^{\sqcap}}$\hss}$^{^{\sqcup}}$}}}}
% \def\sqleq{\sqsubseteq}
% \def\sqgeq{\sqsupseteq}
\def\Td{T_{\overline{d}}}
% \newcommand{\csym}[1]{\ensuremath{\#{c}_{\#{\hbox{\scriptsize #1}}}}}
% \newcommand{\F}{\ensuremath{\mathcal{F}}}
% \newcommand{\fsym}[2]{\ensuremath{\#{f}^{\#{\hbox{\scriptsize #1}}}_{\#{\hbox{\scriptsize #2}}}}}
% \newcommand{\rsym}[2]{\ensuremath{\#{R}^{\#{\hbox{\scriptsize #1}}}_{\#{\hbox{\scriptsize #2}}}}}
% \newcommand{\xsym}[1]{\ensuremath{\#{x}_{\#{\hbox{\scriptsize #1}}}}}
% \newcommand{\I}{\mathcal{I}}
% ********************************************************************

\usepackage[blue]{../framework/thwregex}
\usepackage{environ}
\usepackage{bm}
\usepackage{calc}
\usepackage{varwidth}
\usepackage{wasysym}
\usepackage{mathtools}

%%%%%%%%%%%%%%%%%%%%%%%%%%%%%%%%%%%% Copied from Style_Tut.tex


% Das ist der KIT-Stil
%\usepackage{../TutTexbib/beamerthemekit}
\usepackage[deutsch,titlepage0]{../framework/KIT/beamerthemeKITmod}
\TitleImage[width=\titleimagewd]{../figures/titlepage.jpg}
%\usetheme[deutsch,titlepage0]{KIT}

% Include PDFs
\usepackage{pdfpages}

% Libertine font (Original GBI font)
\usepackage{libertine}
%\renewcommand*\familydefault{\sfdefault}  %% Only if the base font of the document is to be sans serif

% Nicer math symbols
\usepackage{eulervm}
%\usepackage{mathpazo}
\renewcommand\ttdefault{cmtt} % Computer Modern typewriter font, see lecture slides.

\usepackage{csquotes}

%%%%%%

%% Schönere Schriften
\usepackage[TS1,T1]{fontenc}

%% Bibliothek für Graphiken
\usepackage{graphicx}

%% der wird sowieso in jeder Datei gesetzt
%%\graphicspath{{../figures/}}

%% Anzeigetiefe für Inhaltsverzeichnis: 1 Stufe
\setcounter{tocdepth}{1}

%% Hyperlinks
\usepackage{hyperref}
% I don't know why, but this works and only includes sections and NOT subsections in the pdf-bookmarks.
\hypersetup{bookmarksdepth=subsection} 

%\usepackage{lmodern}
\usepackage{colortbl}
\usepackage[absolute,overlay]{textpos}
\usepackage{listings}
\usepackage{forloop}
%\usepackage{algorithmic} % PseudoCode package 

\usepackage{tikz}
\usetikzlibrary{matrix}
\usetikzlibrary{arrows.meta}
\usetikzlibrary{automata}
\usetikzlibrary{tikzmark}

% Needed for gbi-macros
\usepackage{xspace}

%%%%%%

%% Verbatim
\usepackage{moreverb}

%%%%%%%%%%%%%%%%%%%%%%%%%%%%%%%%%%%% Copy end

%% Tabellen
\usepackage{array}
\usepackage{multicol}

%% Bibliotheken für viele mathematische Symbole
\usepackage{amsmath, amsfonts, amssymb}

%% Deutsche Silbentrennung und Beschriftungen
\usepackage[ngerman]{babel}

\usepackage{kbordermatrix}

% kbordermatrix settings
\renewcommand{\kbldelim}{(} % Left delimiter
\renewcommand{\kbrdelim}{)} % Right delimiter


% This is a configuration file with personal tutor information.
% It is therefore excluded from the git repository, so changes in this file will not conflict in git commits.

% Copy this template, rename to config.tex and add your information below.

\newcommand{\myname}{Lukas Morawietz}
\newcommand{\mymail}{lukas.morawietz@gmail.com} % Consider using your named student mail address to keep your u**** account private.
\newcommand{\mytutnumber}{31}

% Don't forget to update ILIAS url. WARNING: Underscores '_' and Ampersands '&' have to be escaped with backslashes '\'. Blame TeX, not me.
\newcommand{\myILIASurl}{https://ilias.studium.kit.edu/ilias.php?ref\_id=855240\&cmdClass=ilrepositorygui\&cmdNode=5r\&baseClass=ilrepositorygui}

% Uncommenting this will print Socrative info with here defined roomname whenever \Socrative is called.
% (Otherwise, \Socrative will remain silent.)
% \newcommand{\mysocrativeroom}{???}

%\def\ThassesTut{}
\def\DanielsTut{}

\newcommand{\aboutMeFrame}{
	\begin{frame}{Über mich}
		\myname \\
		Informatik, 9. Fachsemester (Bachelor)
		% Lebensgeschichte...
		% Stammbaum...
		% Aufarbeitung der eigenen Todesser-Vergangenheit...
	\end{frame}
}

\def\thisyear{2019}

% Update date of exam
\def\myKlausurtermin{18.~März~2020, 14:00–16:00~Uhr}

\def\mydate#1{
		  \ifnum#1=1\relax	  23. Oktober \thisyear \
	\else \ifnum#1=2\relax	  30. Oktober \thisyear \
	\else \ifnum#1=3\relax    06. November \thisyear \
	\else \ifnum#1=4\relax    13. November \thisyear \
	\else \ifnum#1=5\relax    20. November \thisyear \
	\else \ifnum#1=6\relax    27. November \thisyear \
	\else \ifnum#1=7\relax    04. Dezember \thisyear \
	\else \ifnum#1=8\relax    11. Dezember \thisyear \
	\else \ifnum#1=9\relax    18. Dezember \thisyear \
	\else \ifnum#1=10\relax   08. Januar \nextyear \
	\else \ifnum#1=11\relax   15. Januar \nextyear \
	\else \ifnum#1=12\relax   22. Januar \nextyear \
	\else \ifnum#1=13\relax   29. Januar \nextyear \
	\else \ifnum#1=14\relax   05. Februar \nextyear \
	\else \textbf{Datum undefiniert!} 
	\fi\fi\fi\fi\fi\fi\fi\fi\fi\fi\fi\fi\fi\fi
}

\def\mylasttimestext{Was letztes Mal geschah...}

\colorlet{beamerlightred}{red!40}
\colorlet{beamerlightgreen}{green!50}
\colorlet{beamerlightyellow}{yellow!50}
\colorlet{lightred}{red!30}
\colorlet{lightgreen}{green!40}
\colorlet{lightyellow}{yellow!50}
\colorlet{fullred}{red!60}
\colorlet{fullgreen}{green}

\definecolor{myalertcolor}{rgb}{1,0.33,0.24}
\setbeamercolor{alerted text}{fg=myalertcolor}

% Flag to toggle display of KIT Logo.
% If you want to conform to the official logo guidelines, 
% you are not allowed to use the logo and should disable it
% using the following flag. Just saying.
% (But it's too beautiful, so best leave this commented. :P)
%\newcommand{\noKITLogo}{}

% Toggle handout mode by including the following line before including PraeambelTut
% and removing the % at the start (but do NOT remove the % char here, otherwise handout mode will always be on!)
% Please keep handout mode off in all commits!

% \newcommand{\handout}{}



% define custom \handout command flag if handout mode is toggled  #DirtyAsHellButWell...
\only<beamer:0>{\def\handout{}} %beamer:0 == handout mode

\newcommand{\R}{\mathbb{R}}
\newcommand{\N}{\mathbb{N}}
\newcommand{\Z}{\mathbb{Z}}
\newcommand{\Q}{\mathbb{Q}}
\newcommand{\BB}{\mathbb{B}}
\newcommand{\C}{\mathbb{C}}
\newcommand{\K}{\mathbb{K}}
\newcommand{\G}{\mathbb{G}}
\newcommand{\nullel}{\mathcal{O}}
\newcommand{\einsel}{\mathds{1}}
\newcommand{\Pot}{\mathcal{P}}
\renewcommand{\O}{\text{O}}

\def\word#1{\hbox{\textcolor{blue}{\texttt{#1}}}}
\let\literal\word
\def\mword#1{\hbox{\textcolor{blue}{$\mathtt{#1}$}}}  % math word
\def\sp{\scalebox{1}[.5]{\textvisiblespace}}
\def\wordsp{\word{\sp}}

%\newcommand{\literal}[1]{\textcolor{blue}{\texttt{#1}}}
\newcommand{\realTilde}{\textasciitilde \ }
\newcommand{\setsize}[1]{\ensuremath{\left\lvert #1 \right\rvert}}
\newcommand{\size}[1]{\setsize{#1}}  % Shame on you, TeXStudio...
\newcommand{\set}[1]{\left\{#1\right\}}
\newcommand{\tuple}[1]{\left(#1\right)}
\newcommand{\normalvar}[1]{\text{$#1$}}

% Modified by DJ
\let\oldemptyset\emptyset
\let\emptyset\varnothing % proper emptyset

\newcommand{\boder}{\ensuremath{\mathbin{\textcolor{blue}{\vee}}}\xspace}
\newcommand{\bund}{\ensuremath{\mathbin{\textcolor{blue}{\wedge}}}\xspace}
\newcommand{\bimp}{\ensuremath{\mathrel{\textcolor{blue}{\to}}}\xspace}
\newcommand{\bgdw}{\ensuremath{\mathrel{\textcolor{blue}{\leftrightarrow}}}\xspace}
\newcommand{\bnot}{\ensuremath{\textcolor{blue}{\neg}}\xspace}
\newcommand{\bone}{\ensuremath{\textcolor{blue}{1}}\text{}}
\newcommand{\bzero}{\ensuremath{\textcolor{blue}{0}}\text{}}
\newcommand{\bleftBr}{\ensuremath{\textcolor{blue}{\texttt{(}}}\text{}}
\newcommand{\brightBr}{\ensuremath{\textcolor{blue}{\texttt{)}}}\text{}}

% Fix of \b... commands:

\renewcommand{\boder}{\alor}
\renewcommand{\bund}{\aland}
\renewcommand{\bimp}{\alimpl}
\renewcommand{\bgdw}{\aleqv}
\renewcommand{\bnot}{\alnot}
\renewcommand{\bleftBr}{\alka}
\renewcommand{\brightBr}{\alkz}
\newcommand{\alA}{\word A}
\newcommand{\alB}{\word B}
\newcommand{\alC}{\word C}

\newcommand{\plB}{\plfoo{B}}
\newcommand{\plE}{\plfoo{E}}

\newcommand{\summe}[2]{\sum\limits_{#1}^{#2}}
\newcommand{\limes}[1]{\lim\limits_{#1}}

%\newcommand{\numpp}{\advance \value{weeknum} by -2 \theweeknum \advance \value{weeknum} by 2}
%\newcommand{\nump}{\advance \value{weeknum} by -1 \theweeknum \advance \value{weeknum} by 1}

\newcommand{\mycomment}[1]{}
\newcommand{\Comment}[1]{}

%% DISCLAIMER START 
% It is INSANELY IMPORTANT NOT TO DO THIS OUTSIDE BEAMER CLASS! IN ARTCILE DOCUMENTS, THIS IS VERY LIKELY TO BUG AROUND!
\makeatletter%
\@ifclassloaded{beamer}%
{
	% TODO 
	% no time...
	% redefine section to ignore multiple \section calls with the same title
}%
{
	\errmessage{ERROR: section command redefinition outside of beamer class document! Please contact the author of this code.}
}%
\makeatother%
%% DISCLAIMER END

\newcounter{abc}
\newenvironment{alist}{
  \begin{list}{(\alph{abc})}{
      \usecounter{abc}\setlength{\leftmargin}{8mm}\setlength{\labelsep}{2mm}
    }
}{\end{list}}


\newcommand{\stdarraystretch}{1.20}
\renewcommand{\arraystretch}{\stdarraystretch}  % for proper row spacing in tables

\newcommand{\morescalingdelimiters}{   % for proper \left( \right) typography
	\delimitershortfall=-1pt  
	\delimiterfactor=1
}

\newcommand{\centered}[1]{\vspace{-\baselineskip}\begin{center}#1\end{center}\vspace{-\baselineskip}}

% for \implitem and \item[bla] stuff to look right:
\setbeamercolor*{itemize item}{fg=black}
\setbeamercolor*{itemize subitem}{fg=black}
\setbeamercolor*{itemize subsubitem}{fg=black}

\setbeamercolor*{description item}{fg=black}
\setbeamercolor*{description subitem}{fg=black}
\setbeamercolor*{description subsubitem}{fg=black}

\renewcommand{\qedsymbol}{\textcolor{black}{\openbox}}

\renewcommand{\mod}{\mathop{\textbf{mod}}}
\renewcommand{\div}{\mathop{\textbf{div}}}

\newcommand{\ceil}[1]{\left\lceil#1\right\rceil}
\newcommand{\floor}[1]{\left\lfloor#1\right\rfloor}
\newcommand{\abs}[1]{\left\lvert #1 \right\rvert}
\newcommand{\Matrix}[1]{\begin{pmatrix} #1 \end{pmatrix}}
\newcommand{\braced}[1]{\left\lbrace #1 \right\rbrace}

% "something" placeholder. Useful for repairing spacing of operator sections, like `\sth = 42`.
\def\sth{\vphantom{.}}

\def\fract#1/#2 {\frac{#1}{#2}} % ! Trailing space is crucial!
\def\dfract#1/#2 {\dfrac{#1}{#2}} % ! Trailing space is crucial!

\newcommand{\Mid}{\;\middle|\;}

\let\after\circ



\def\·{\cdot}
\def\*{\cdot}
\def\?>{\ensuremath{\rightsquigarrow}}  % Fuck you, Latex
\def\~~>{\ensuremath{\rightsquigarrow}}  

\newcommand{\tight}[1]{{\renewcommand{\arraystretch}{0.76} #1}}
\newcommand{\stackedtight}[1]{\renewcommand{\arraystretch}{0.76} \begin{matrix} #1 \end{matrix} }
\newcommand{\stacked}[1]{\begin{matrix} #1 \end{matrix} }
\newcommand{\casesl}[1]{\delimitershortfall=0pt  \left\lbrace\hspace{-.3\baselineskip}\begin{array}{ll} #1 \end{array}\right.}
\newcommand{\casesr}[1]{\delimitershortfall=0pt  \left.\begin{array}{ll} #1 \end{array}\hspace{-.3\baselineskip}\right\rbrace}
\newcommand{\caseslr}[1]{\delimitershortfall=0pt  \left\lbrace\hspace{-.3\baselineskip}\begin{array}{ll} #1 \end{array}\hspace{-.3\baselineskip}\right\rbrace}

\def\q#1uad{\ifnum#1=0\relax\else\quad\q{\the\numexpr#1-1\relax}uad\fi}
% e.g. \q1uad = \quad, \q2uad = \qquad etc.

\newcommand{\qqquad}{\q3uad}
\newcommand{\minusquad}{\hspace{-1em}}

%% Placeholder utils
% \§{#1}   Saves #1 as placeholder and prints it
% \.       Prints an \hphantom with the size of the recalled placeholder.
\def\indentstring{}
\def\§#1{\def\indentstring{#1}#1}
\def\.{{$\hphantom{\text{\indentstring}}$}}
%% Placeholder utils end

\newcommand{\impl}{\ifmmode\ensuremath{\mskip\thinmuskip\Rightarrow\mskip\thinmuskip}\else$\Rightarrow$\fi\xspace}
\newcommand{\Impl}{\ifmmode\implies\else$\Longrightarrow$\fi\xspace}

\newcommand{\derives}{\Rightarrow}

\newcommand{\gdw}{\ifmmode\mskip\thickmuskip\Leftrightarrow\mskip\thickmuskip\else$\Leftrightarrow$\fi\xspace}
\newcommand{\Gdw}{\ifmmode\iff\else$\Longleftrightarrow$\fi\xspace}

% Legacy code from the algo tutorial slides. Perhaps useful. Try with care.
\mycomment{
	\newcommand{\impl}{\ifmmode\ensuremath{\mskip\thinmuskip\Rightarrow\mskip\thinmuskip}\else$\Rightarrow$\xspace\fi}  
	\newcommand{\Impl}{\ifmmode\implies\else$\Longrightarrow$\xspace\fi}
	
	\newcommand{\gdw}{\ifmmode\mskip\thickmuskip\Leftrightarrow\mskip\thickmuskip\else$\Leftrightarrow$\xspace\fi}
	\newcommand{\Gdw}{\ifmmode\iff\else$\Longleftrightarrow$\xspace\fi}
}
	
\newcommand{\gdwdef}{\ifmmode\mskip\thickmuskip:\Leftrightarrow\mskip\thickmuskip\else:$\Leftrightarrow$\xspace\fi}
\newcommand{\Gdwdef}{\ifmmode\mskip\thickmuskip:\Longleftrightarrow\mskip\thickmuskip\else:$\Longleftrightarrow$\xspace\fi}

\newcommand{\symbitemnegoffset}{\hspace{-.5\baselineskip}}
\newcommand{\implitem}{\item[\impl\symbitemnegoffset]}
\newcommand{\Implitem}{\item[\Impl\symbitemnegoffset]}


\newcommand{\forcenewline}{\mbox{}\\}

\newcommand{\bfalert}[1]{\textbf{\alert{#1}}}
\let\elem\in   % I'm a Haskell freak. Don't judge me. :P


\def\|#1|{\text{\normalfont #1}}  % | steht für senkrecht (anstatt kursiv wie sonst im math mode)


% proper math typography
\newcommand{\functionto}{\longrightarrow}
\renewcommand{\geq}{\geqslant}
\renewcommand{\leq}{\leqslant}
\let\oldsubset\subset
\renewcommand{\subset}{\subseteq} % for all idiots out there using subset

\newenvironment{threealign}{%
	\[
	\begin{array}{r@{\ }c@{\ }l}
}{%
	\end{array}	
	\]
}

\newcommand{\concludes}{ \\ \hline  }
\newcommand{\deduction}[1]{
	\begin{varwidth}{.8\linewidth}
		\begin{tabular}{>{$}c<{$}}
			#1
		\end{tabular}
	\end{varwidth}	
}

\definecolor{hoareorange}{rgb}{1,.85,.6}
\newcommand{\hoareassert}[1]{\setlength{\fboxsep}{1pt}\setlength{\fboxrule}{-1.4pt}\fcolorbox{white}{hoareorange}{\ensuremath{\{\;#1\;\}}}\setlength\fboxrule{\defaultfboxrule}\setlength{\fboxsep}{3pt}}

\newcommand{\mailto}[1]{\href{mailto:#1}{{\textcolor{blue}{\underline{#1}}}}}
\newcommand{\urlnamed}[2]{\href{#2}{\textcolor{blue}{\underline{#1}}}}
\renewcommand{\url}[1]{\urlnamed{#1}{#1}}

\newcommand{\hanging}{\hangindent=0.7cm}
\newcommand{\indented}{\hanging}


% \hstretchto prints #2 left-aligned into a box of the width of #1
\def\hstretchto#1#2{%
	\mbox{}\vphantom{#2}\rlap{#2}\hphantom{#1}%
}

\def\vstretchto#1#2{%
	\mbox{}\hphantom{#2}\smash{#2}\vphantom{#1}%
}


%requires \thisyear to be defined (s. config.tex)!
\edef\nextyear{\the\numexpr\thisyear+1\relax}


% --- \frameheight constant ---
\newlength\fullframeheight
\newlength\framewithtitleheight
\setlength\fullframeheight{.92\textheight}
\setlength\framewithtitleheight{.86\textheight}

\newlength\frameheight
\setlength\frameheight{\fullframeheight}

\let\frametitleentry\relax
\let\oldframetitle\frametitle
\def\newframetitle#1{\global\def\frametitleentry{#1}\if\relax\frametitleentry\relax\else\setlength\frameheight{\framewithtitleheight}\fi\oldframetitle{#1}}
\let\frametitle\newframetitle

\def\newframetitleoff{\let\frametitle\oldframetitle}
\def\newframetitleon{\let\frametitle\newframetitle}
% --- \frameheight constant end ---

\newcommand{\fakeframetitle}[1]{%
	\vspace{-2.05\baselineskip}%
	{\Large \textbf{#1}} \\%
	\smallskip
}



\newenvironment{headframe}{\Huge THIS IS AN ERROR. PLEASE CONTACT THE ADMIN OF THIS TEX CODE. (headframe env def failed)}{}
\RenewEnviron{headframe}[1][]{
	\begin{frame}\frametitle{\ }
		\centering
		\Huge\textbf{\textsc{\BODY} \\
		}
		\Large {#1}
		\frametitle{\ }
	\end{frame}
}


\makeatletter
% Provides color if undefined.
\newcommand{\colorprovide}[2]{%
	\@ifundefinedcolor{#1}{\colorlet{#1}{#2}}{}}
\makeatother


\colorprovide{lightred}{red!30}
\colorprovide{lightgreen}{green!40}
\colorprovide{lightyellow}{yellow!50}
\colorprovide{lightblue}{blue!10}
\colorprovide{beamerlightred}{lightred}
\colorprovide{beamerlightgreen}{lightgreen}
\colorprovide{beamerlightyellow}{lightyellow}
\colorprovide{beamerlightblue}{lightblue}
\colorprovide{fullred}{red!60}
\colorprovide{fullgreen}{green}
\definecolor{darkred}{RGB}{115,48,38}
\definecolor{darkgreen}{RGB}{48,115,38}
\definecolor{darkyellow}{RGB}{100,100,0}

\only<handout:0>{\colorlet{adaptinglightred}{beamerlightred}}
\only<handout:0>{\colorlet{adaptinglightgreen}{beamerlightgreen}}
\only<handout:0>{\colorlet{adaptinglightyellow}{beamerlightyellow}}
\only<handout:0>{\colorlet{adaptinglightblue}{beamerlightblue}}
\only<beamer:0>{\colorlet{adaptinglightred}{lightred}}
\only<beamer:0>{\colorlet{adaptinglightgreen}{lightgreen}}
\only<beamer:0>{\colorlet{adaptinglightyellow}{lightyellow}}
\only<beamer:0>{\colorlet{adaptinglightblue}{lightblue}}
\only<handout:0>{\colorlet{adaptingred}{lightred}}
\only<beamer:0>{\colorlet{adaptingred}{fullred}}
\only<handout:0>{\colorlet{adaptinggreen}{lightgreen}}
\only<beamer:0>{\colorlet{adaptinggreen}{fullgreen}}



\newcommand{\TrueQuestion}[1]{
	\TrueQuestionE{#1}{}
}

\newcommand{\YesQuestion}[1]{
	\YesQuestionE{#1}{}
}

\newcommand{\FalseQuestion}[1]{
	\FalseQuestionE{#1}{}
}

\newcommand{\NoQuestion}[1]{
	\NoQuestionE{#1}{}
}

\newcommand{\DependsQuestion}[1]{
	\DependsQuestionE{#1}{}
}

\newcommand{\QuestionVspace}{\vspace{4pt}}
\newcommand{\QuestionParbox}[1]{\begin{varwidth}{.85\linewidth}#1\end{varwidth}}
\newcommand{\ExplanationParbox}[1]{\begin{varwidth}{.97\linewidth}#1\end{varwidth}}
\colorlet{questionlightgray}{gray!23}
\let\defaultfboxrule\fboxrule

% #1: bg color
% #2: fg color short answer
% #3: short answer text
% #4: question
% #5: explanation
\newcommand{\GenericQuestion}[5]{
	\setlength\fboxrule{2pt}
	\only<+|handout:0>{\hspace{-2pt}\fcolorbox{white}{questionlightgray}{\QuestionParbox{#4} \quad\textbf{?}}}
	\visible<+->{\hspace{-2pt}\fcolorbox{white}{#1}{\QuestionParbox{#4} \quad\textbf{\textcolor{#2}{#3}}} \if\relax#5\relax\else\ExplanationParbox{#5}\fi} \\
	\setlength\fboxrule{\defaultfboxrule}
}

% #1: Q text
% #2: Explanation
\newcommand{\TrueQuestionE}[2]{
	\GenericQuestion{adaptinglightgreen}{darkgreen}{Wahr.}{#1}{#2}
}

% #1: Q text
% #2: Explanation
\newcommand{\YesQuestionE}[2]{
	\GenericQuestion{adaptinglightgreen}{darkgreen}{Ja.}{#1}{#2}
}

% #1: Q text
% #2: Explanation
\newcommand{\FalseQuestionE}[2]{
	\GenericQuestion{adaptinglightred}{darkred}{Falsch.}{#1}{#2}
}

% #1: Q text
% #2: Explanation
\newcommand{\NoQuestionE}[2]{
	\GenericQuestion{adaptinglightred}{darkred}{Nein.}{#1}{#2}
}

% #1: Q text
% #2: Explanation
\newcommand{\DependsQuestionE}[2]{
	\GenericQuestion{adaptinglightyellow}{darkyellow}{Je nachdem!}{#1}{#2}
}

% #1: Q text
% #2: Answer
\newcommand{\ContentQuestion}[2]{
	\GenericQuestion{adaptinglightblue}{black}{\minusquad}{#1}{#2}
}

\ifnum\thisyear=2018 \else \errmessage{Old ILIAS link inside preamble. Please update.} \fi

\newcommand{\ILIAS}{\urlnamed{ILIAS}{https://ilias.studium.kit.edu/ilias.php?ref\_id=855240\&cmdClass=ilrepositorygui\&cmdNode=5r\&baseClass=ilrepositorygui}\xspace}

\newcommand{\Socrative}{\ifdefined\mysocrativeroom \only<handout:0>{socrative.com $\quad \~~> \quad $ Student login \\ Raumname:  \mysocrativeroom\\ \medskip}\else\fi}

\newcommand{\thasse}[1]{
	\ifdefined\ThassesTut #1\xspace \else\fi
}
\newcommand{\daniel}[1]{
	\ifdefined\DanielsTut #1\xspace \else\fi
}
\newcommand{\thassedaniel}[2]{\ifdefined\ThassesTut #1\else\ifdefined\DanielsTut #2\fi\fi\xspace}

\ifdefined\ThassesTut \ifdefined\DanielsTut \errmessage{ERROR: Both ThassesTut and DanielsTut flags are set. This is most likely an error. Please check your config.tex file.} \else \fi \else \ifdefined\DanielsTut \else \errmessage{ERROR: Neither ThassesTut  nor DanielsTut flags are set. This is most likely an error. Please check your config.tex file.} \fi\fi

%\newcommand{\sgn}{\text{sgn}}

%%%%%%%%%%%% INHALT %%%%%%%%%%%%%%%%

\newcommand{\lastframetitled}[6]{
	\frame{\frametitle{#6}
		\vspace{-#2\baselineskip}
		\begin{figure}[H]
			\centering
			\LARGE \textbf{\textsc{#5}} \\
			\vspace{.2\baselineskip}
			\includegraphics[#1]{#3}
			\vspace{-6pt}
			\begin{center}
				\small \url{#4} 
			\end{center}
		\end{figure} 
	}
}

% #1 number
% #2 title 
% #3 vspace (positive) without unit (\baselineskip)
\newcommand{\xkcdframe}[3]{
	\lastframetitled{width=.96\textwidth}{#3}{xkcd/#1}{http://xkcd.com/#1}{}{#2}
}

\newcommand{\xkcdframevert}[3]
{
	\lastframetitled{height=.96\frameheight}{#3}{xkcd/#1}{http://xkcd.com/#1}{}{#2}
}

% #1 number
% #2 title 
% #3 vspace (positive) without unit (\baselineskip)
% #4 \includegraphics[] optional parameters
\newcommand{\xkcdframecustom}[4]
{
	\lastframetitled{#4}{#3}{xkcd/#1}{http://xkcd.com/#1}{}{#2}
}

\morescalingdelimiters

\begin{document}
\starttut{9}

\begin{frame}{Zu Blatt \#3}
	Durchschnitt: \quad etwa \thassedaniel{58}{53}~\% der Punkte \\
	\begin{itemize}
		\item \textbf{Induktionen}: Schreibt mir bitte die Aussage hin, über die ihr die Induktion macht. Wenn IV falsch, hab ich sonst keinen Plan, was ihr zeigen wollt.
		\item \textbf{A3.1}: Schaut euch Injektivität/Surjektivität nochmal an...
		\item \textbf{A3.2}: Huffman-Bäume: Es werden IMMER die zwei KLEINSTEN Knoten verbunden. Auch „über Kreuz“. 
		\item \textbf{A3.3}: Terme mit zu vielen Pünktchen („...“) sind keine Definition. \\ Alles, was nicht rekursiv ist, muss falsch sein wegen Klammerausdrücken.
		\item \textbf{A3.6}: Induktion über $n = \size{w_1} = \size{w_2}$, also beide Wortlängen gleichzeitig, geht NICHT! (Gibt nämlich auch Wörter, wo beide Längen nicht gleich sind... :P)
		
	\end{itemize}
\end{frame}

\framePrevEpisode

\daniel{
	
\thasse{
	\begin{frame}{}
		\begin{block} {Aussagenlogik}
			\begin{itemize}
				\item \enquote{Es regnet und alle Vögel sind grau.}
				\item atomar: \enquote{Es regnet.}, \enquote{ Alle Vögel sind grau.}
				\item Diese beiden Aussagen lassen sich ihrerseits nicht in weitere Teilaussagen zerlegen!
			\end{itemize}
		\end{block}
		
		\pause
		\begin{block} {Prädikatenlogik}	
			\begin{itemize}
				\item In der Prädikatenlogik werden atomare Aussagen hinsichtlich ihrer inneren Struktur untersucht.
				\item \enquote{Alle Vögel sind grau}
				\item lässt sich in : \enquote{Alle Vögel}, \enquote{sind grau} zerlegen.
			\end{itemize}
			
		\end{block}
	\end{frame}
}

\section{Prädikatenlogik: Syntax}

\begin{frame}{Syntax}
	\begin{block}{Aufbau von prädikatenlogischen Formeln}
	\begin{itemize}[<+->]
		\item \textbf{Terme}: Liefern \enquote{Werte} (Zahlen, Wörter, whatever...); \\ 
		Aus Konstanten, Variablen und Funktionssymbolen zusammengesetzt.
		\item \textbf{Atomare Formeln}: Liefern Wahrheitswerte $\in \BB$; \\
		 Aus Termen und Relationssymbolen zusammengesetzt.
		\item \textbf{Prädikatenlogische Formeln}: aus atomaren Formeln und AL-Konnektiven sowie Quantoren ($\forall, \exists$) zusammengesetzt. 
	\end{itemize}
	\end{block}
\end{frame}

\begin{frame}{Terme}
	Bestehen aus... \\
	\medskip
	
	\textbf{Variablen} \, (endlich viele) \quad Alphabet $\VPL$ \\
	$\word{x}_i$ \quad $\word x, \word y, \word z$ \\
	Können in einer Formel beliebig verschiedene Werte annehmen \\
	\medskip \pause
	
	\textbf{Konstanten} \, (endlich viele) \quad Alphabet $\CPL$ \\
	$\word{c}_i$ \quad $\word c, \word d$ \\
	Konstanter Wert in der gesamten Formel \\
	\medskip \pause
	
	\textbf{Funktionen} \, (endlich viele) \quad Alphabet $\FPL$ \\
	$\word{f}_i$ \quad $\word f, \word g, \word h$ \\
	Liefern Werte für irgendwelche Eingaben (wie gewohnt) \\
	Jedes $\word{f}_i$ hat Stelligkeit $\ar(\word{f}_i) \in \N_+$ \quad „Wieviel Argumente $\word{f}_i$ nimmt“ {\small (Arität)} \\
	\smallskip \pause
	
	Beispiel: 
	\quad Funktionen $\word f$ mit $\ar(\word f) = 2$, \quad $\word g$ mit $\ar(\word g) = 1$. \\ \pause
	\quad OK: \qquad \word{f(x,y)} \qquad \word{g(c)} \qquad \word{g(f(x,c))} \\
	\quad Kaputt: \qquad \word{f(c)} \qquad \word{g(x,y,z)} \qquad \word{f(x,y} \qquad \word{f(x,,}
\end{frame}

\begin{frame}{Atomare Formeln}
	Zusammengesetzt aus... \\
	\medskip
	
	\textbf{Termen} (von eben) als Argumente in \\
	\medskip
	
	\textbf{Relationen} \, (endlich viele) \quad Alphabet $\RPL$  \\
	$\pleq$ und beliebige $\word R_i$ \quad $\word R, \word S$ \\
	Liefern Wahrheitswerte in $\BB = \set{\W, \F}$ \\
	Haben auch Stelligkeit $\ar(\word R_i) \in \N_+$ \\
	\medskip \pause
	
	Beispiele: \quad Relationen \word R mit $\ar(\word R) = 1$, \quad \word S mit $\ar(\word S) = 2$ \\
	\quad $\word x \pleq \word c$ \qquad $\word{g(x)} \pleq \word{g(y)}$ \qquad $\word{R(g(c))}$ \qquad $\word{S(x,f(x,y))}$
	\medskip \pause
	
	Relationen und Funktionen können \textbf{nur mit Termen} aufgerufen werden! \\
	\impl \quad \word{R(S(x,y))} \qquad \word{g(S(x))} \qquad $\word{R(x)} \pleq \word{S(x,y)}$ \qquad $\word x \pleq \word{(}\word y \pleq \word z\word{)}$ \\
	Alles \textbf{falsch}!
\end{frame}

\begin{frame}{Prädikatenlogische Formeln}
	Bestehen aus... \\
	\medskip
	
	\textbf{Atomaren Formeln} (von eben) \\
	\medskip
	
	\textbf{AL-Konnektiven} \\
	$\alnot, \aland, \alor, \alimpl$ und Klammern {\small (einsparbar)} \\
	Verknüpfen PL-Formeln \\
	\medskip \pause
	
	\textbf{Quantoren} \\
	$\plall$ All-Quantor \quad $\plexist$ Existenz-Quantor \\
	Verwendung: $\bleftBr\plall \word x_i \text{...Formel...}\brightBr$ \quad $\bleftBr\plexist \word x_i \text{...Formel...}\brightBr$ \\
	\impl NUR \textbf{Variablen} können quantifiziert werden! \\
	\medskip 
	
	Klammerregeln: Wie bei AL-Formeln; Quantoren binden stärker als alles andere.
	\bigskip \pause
	
	\textbf{Falsch}: \\
	$\bleftBr\plexist \word c \text{...Formel...}\brightBr$ mit $\word c \in \CPL$ \quad \word c keine Variable! \\
	$\bleftBr\plall \word x \word{:} \text{...Formel...}\brightBr$ \quad KEINE Doppelpunkte! \\
\end{frame}

\begin{frame}{Aufgabe: Syntaktische Fehler}
	Findet jeweils den syntaktischen Fehler in den folgenden prädikatenlogischen Formeln:
	\begin{itemize}
		\item $\plexists \word z \: \plka \word{c(z)} \plkz$\\
		\visible<2-|handout:2>{Konstanten (\word c) sind keine Relationen!}
		\item $\word{S(c)} \aland \word{T(d)} \alimpl \word{S(R(c,d))}$\\
		\visible<3-|handout:2>{Nur Terme als Argumente, keine Relationen (\word{R(c,d)}).}
		\item $\plall \plx \plall \ply \: \word{R(f(x,f(y)))}$\\
		\visible<4-|handout:2>{$\ar(\word f) = 1$ oder $\ar(\word f) = 2$, aber nicht beides gleichzeitig.}
		\item $\plall \plx \: \plka \word{S(x)} \aland \word{T(x)} \plkz \alimpl \plexists \word R \: \plka \plall \plx \: \word{R(x,x)} \plkz $\\
		\visible<5-|handout:2>{Relationen können nicht quantifiziert werden ($\plexists \word R$).}
		\item $\plall \word f \: \word{S(f)} \aland \plall \plx \: \word{g(R(x))}$\\
		\visible<6-|handout:2>{Nur Terme als Argumente, keine Relationen (\word{R(x)}).\\
		Hinweis: $\plall \word f \: \word{S(f)}$ ist \emph{erlaubt}. Hier ist $\word f$ nur eine Variable, \word f muss nicht immer eine Funktion sein. Hier wird nirgends \word f „aufgerufen“ (\word{f(}...\word{)}).}
	\end{itemize}
\end{frame}

\begin{frame}{Formeln aufstellen}
	\begin{Beispiel}
		Alle Menschen lieben Weihnachten.\\ % Aber Gänse eher nicht...
		\medskip
		
		\pause
		$\plall \plx \; {\plka \word{Mensch(x)} \alimpl \word{liebt(x,Weihnachten)} \plkz}$
	\end{Beispiel}
% TODO: Zweites Beispiel (evtl. aus Übung klauen)
\end{frame}

\begin{frame}{Aufgabe 2 (WS 15/16, Blatt 7)}
	\begin{block}{Aufgabe}
		Formuliert die folgenden Aussagen als Formeln in Prädikatenlogik:
		\begin{enumerate}
			\item Nicht alle Vögel können fliegen.
			\item Wenn es irgendjemand kann, dann kann es Donald Ervin Knuth.
			\item John liebt jeden, der sich nicht selbst liebt.
		\end{enumerate}
	\end{block}
	
	\visible<2-|handout:2>{
		\begin{block}{Lösung}
			\begin{enumerate}
				\item \qqquad $\plexist \plx {\plka \plfoo{Vogel}{\plka \plx \plkz} \aland \alnot \plfoo{flugfähig}{\plka \plx \plkz} \plkz}$
				\visible<3-|handout:2>{
					\item \qqquad $
					\plexist \plx {\plka \plfoo{kann\_es}{\plka \plx \plkz} \plkz}
					\alimpl
					\plfoo{kann\_es}{\plka \plfoo{knuth} \plkz}
					$
				}
				\visible<4-|handout:2>{
					\item \qqquad $
					\plall \plx {\plka \alnot \plfoo{liebt}{\plka \plx \plcomma \plx \plkz} \alimpl \plfoo{liebt}{\plka \plfoo{John} \plcomma \plx \plkz} \plkz}
					$
				}
			\end{enumerate}
		\end{block}
	}
\end{frame}

\begin{frame}{Freie und gebundene Variablenvorkommen}
	\textbf{Vorkommen} einer Variable in einer PL-Formel: \\
	$G = \plall \word{x\,R(\underline{x})}$ \Impl \word x kommt in $G$ vor. \\
	„Direkt hinter Quantoren“ zählt nicht! \\
	Bsp.: $F = \plall \word{x\,R(c)}$ \Impl \word x kommt \textbf{nicht} in $F$ vor! \\
	\medskip \pause
	
	\textbf{Freie Variablenvorkommen} \quad $\fv(F)$ \\
	Alle, die \emph{irgendwo} in $F$ vorkommen, ohne dass sie quantifiziert sind {\small ($=$ ein Quantor sie einführt)}. \\
	\smallskip \pause
	Beispiel: $\fv\left(\word{R(\underline{x},\underline{y},c)} \aland \plexist \word x\, \plka \word x \pleq \underline{\word y} \plkz \right) = \set{\word x, \word y}$ \\
	\medskip \pause
	
	\textbf{Gebundene Variablenvorkommen} \quad $\bv(F)$ \\
	Alle, die \emph{irgendwo} in $F$ vorkommen und dabei quantifziert sind. \\
	\smallskip \pause
	Beispiel: $\bv\left(\word{R(x,y,c)} \aland \plexist \word x\, \plka \underline{\word x} \pleq \word y \plkz \right) = \set{\word x}$ \\
	\medskip \pause
	
	Formel $F$ heißt \textbf{geschlossen} $:\!\!\Gdw \fv(F) = \emptyset$\\
	\medskip 

\end{frame}

\begin{frame}{Freie und gebundene Variablenvorkommen}
	\begin{equation*}
	F = \plexist \plx \, \plE{\plka \plx \plcomma \ply \plkz}
		\, \alor \,
		\plall \plz \, \plall \plx \, \plall \ply \, {\plka
			\plE{\plka \plx \plcomma \plz \plkz} \aland \plE{\plka \ply \plcomma \plz \plkz}
		\plkz}
	\end{equation*}
	
	\begin{block}{Aufgabe}
		Welche Variablenvorkommen sind frei ($\fv$) und welche gebunden ($\bv$)?\\
		Ist die Formel geschlossen?
	\end{block}

	\pause
	\begin{block}{Lösung}
		Nur die Variable $\fv(F) = \{\ply\}$ kommt frei in $F$ vor.\\
		Genau die Variablen $\bv(F) = \{\plx, \ply, \plz\}$ kommen gebunden in $F$ vor.\\
		Da $\fv(F) \neq \emptyset$, ist $F$ nicht geschlossen.
	\end{block}
	
\end{frame}

\begin{frame}{Substitutionen}
	\delimitershortfall=0pt
	Wollen \textbf{Variablen} (!) durch andere Terme ersetzen. \\
	\impl Eine Substitutionsabbildung $$\sigma_S \from \LFor \functionto \LFor, \quad \text{längliche Definition s. VL}$$ wendet Ersetzungen aus $S$ auf eine Formel an. \\
	\medskip \pause
	
	Beispiel: \\
	$\sigma_{\set{\word x/\word y}}\left(\word x \pleq \word c\right) = \word y \pleq \word c$. \\
	$\sigma_{\set{\word x/\word{f(c)}}}\left(\word{R(x,y)} \aland \plexists  \word y\,\plka\word y \pleq \word x \plkz\right) = \word{R(\word{f(c)},y)} \aland \plexists  \word y\,\plka\word y \pleq \word{f(c)\plkz}$. \\
	\medskip \pause
	
	Mehrere auf einmal: \\
	$\sigma_{\set{\word x/\word y, \, \word y/\word x}}\left(\word x \pleq \word y\right) = \word y \pleq \word x$.
\end{frame}

\begin{frame}{Substitutionen}
	\delimitershortfall=0pt
	Ersetzt werden nur \textbf{freie Variablenvorkommen}!\\
	Gebundene Vorkommen, also Variablen im Wirkungsbereich eines (eigenen) Quantors, werden \textbf{nicht} ersetzt. \\
	\medskip \pause
	
	Beispiel: \\
	$\sigma_{\set{\word y/\word{f(c)}}}\left(\word{R(x,y)} \aland \plexists \word y\, \plka\word y \pleq \word x \plkz\right) = \word{R(x,\word{f(c)})} \aland \plexists \word y\,\plka\word y \pleq \word x \plkz$
	
\end{frame}

\begin{frame}{Substitutionen: Kollisionsfreiheit}
	Bei einer \textbf{kollisionsfreien} Substitution werden keine Variablen durch Ersetzung \enquote{aus Versehen} gebunden.  \\
	\medskip
	Ersetzen wir eine freie Variable $\word x$ durch einen Term, in dem die Variable $\word y$ frei vorkommt, so darf sich $\word x$ nicht im Wirkungsbereich eines Quantors über $\word y$ befinden. \\
	(Sonst wird \word y nämlich unfreiwillig quantifiziert und die Bedeutung ändert sich!)
	
	\pause
	\begin{Beispiel}
		$F = \plall \plx \plka\plx \pleq \ply\plkz$\\
		Kollisionsfrei: $\sigma_{\{\ply/\plz\}}(\plall \plx \plka\plx \pleq \ply\plkz) = \plall \plx \plka\plx \pleq \underline{\plz}\plkz$\\
		Nicht kollisionsfrei: $\sigma_{\{\ply/\plx\}}(\plall \plx \plka\plx \pleq \ply\plkz) = \plall \plx \plka\plx \pleq \underline{\plx}\plkz$ 
	\end{Beispiel}
\end{frame}

\begin{frame}{Substitutionen: Aufgabe}
	$F = \alnot \plexist \plx {\plka \plE{\plka \plx \plcomma \ply \plkz}\plkz}$\\
	$G = \plall \plx \plall \ply {\plka 
			\plE{\plka \plx \plcomma \plz \plkz} \aland 
			\plE{\plka \ply \plcomma \plz \plkz} \plkz}$
	\medskip
	
	Gebt jeweils eine Substitution $\sigma$ an, die \emph{nicht} kollisionsfrei für $F$ bzw. $G$ ist.\\[1em] \pause
		
	\impl Die Substitution $\sigma_{\{\ply/\plx\}}$ macht $F$ kaputt.\\ \pause
	\impl Die Substitutionen $\sigma_{\{\plz/\plx\}}$ oder $\sigma_{\{\plz/\ply\}}$ machen $G$ kaputt.
	
\end{frame}


}


\section{Prädikatenlogik: Semantik}

\begin{frame}[t]{Interpretationen}
	\begin{block}{Interpretation $(D,I)$}
		\begin{itemize}
			\item nichtleere Menge $D$ \quad „Universum“
			\item $I(\word c_i) \in D$
			\item $I(\word f_i) \from D^k \functionto D$ \quad ($k$: Stelligkeit von $\word f_i$)
			\item $I(\word R_i) \subseteq D^k$ \; Relation auf D \quad ($k$: Stelligkeit von $\word R_i$)
			\item \textbf{KEINE} Werte für Variablen!
		\end{itemize}
		Beispiel: \\
		\only<all:1>{
			\begin{itemize}
				\item $D = \N_0$
				\item $I(\word c) = 42$
				\item $\ar(\word f) = 1$, \\ $I(\word f) \from \N_0 \functionto \N_0, \, I(\word f)(n) = n+2$
				\item $\ar(\word R) = 2$, \; $I(\word R) = \set{(n, m) \Mid 2n \geq m }$ 
			\end{itemize}
		}
		\visible<all:2>{
			\begin{itemize}
				\item $D = \text{Alle Menschen} \cup \text{Alle Namen}$
				\item $I(\word{knuth}) = \text{Donald E. Knuth}$
				\item $\ar(\word{vorname}) = 1$, \\ $I(\word {vorname}) \from D \functionto D, \, I(\word {vorname})(\text{Donald E. Knuth}) = \word{Donald}$, ....
				\item $\ar(\word {bekannt}) = 1$, \; $I(\word {bekannt}) = \set{\text{Donald E. Knuth, Donald Trump, ...}}$
			\end{itemize}
		}
	\end{block}
\end{frame}

\begin{frame}{Variablenbelegungen, Auswertung}
	\begin{block}{Variablenbelegung $\beta \from \VPL \functionto D$}
		Mehrere verschiedene für eine feste Interpretation $(D,I)$ möglich! \\
		Beispiel: \\
		\begin{itemize}
			\item $\beta_1(\word x) = 42$ und $\beta_1(\word y) = 1337$ \quad oder
			\item $\beta_2(\word x) = 2$ und $\beta_2(\word y) = 10000$ 
		\end{itemize}
	\end{block}
	\pause
	\begin{block}{Auswertung}
		$\valDIb$ \quad „Auswertungsfunktion“ \\
		\begin{itemize}
			\item rechnet Werte aus für Terme \\
			$\valDIb{}_{{}_1}(\word x) = 42$, \; $\valDIb{}_{{}_1}(\word{f(y)}) = 1337 + 2 = 1339$
			\item rechnet Wahrheitswerte aus für ganze PL-Formeln \\
			$\valDIb(\plexists \word n \, \plka\word{Prim(n)\plkz}) = \W$, \; $\valDIb(\plall \word m \, \plka \word{vorname(m)}\pleq \word{Donald} \plkz) = \F$
			\item „wie wir's kennen“
		\end{itemize}
	\end{block}
\end{frame}

\begin{frame}{Aufgabe 1}
	\[ F := \word{R(42)} \aland \plexists \plx {\plka \plx \pleq \ply \plkz}  \]
	Gebt eine Interpretation $(D_1 = \N_0, I_1)$ und eine Variablenbelegung $\beta_1$ so an, dass $val_{D_1, I_1, \beta_1}(F) = \W$ gilt.\\
	\smallskip
	$I_1(\word R) = \visible<2-|handout:2>{\{42\},}$ \qquad
	$\beta_1 \colon \VPL \functionto D_1, \, $ \visible<2-|handout:2>{$\beta_1(\ply) = 1, \beta_1(\plx) \text{ beliebig}$}
	
	\medskip
	
	\[ G := \plexists \plx {\plka \word{R(f(x))} \aland \plx \pleq \ply \plkz}  \]
	Gebt eine Interpretation $(D_2 = \N_0, I_2)$ und eine Variablenbelegung $\beta_2$ so an, dass $val_{D_2, I_2, \beta_2}(G) = \W$ gilt.\\
	\smallskip
	\visible<3-|handout:2>{$I_2(\word R) = \N_0, \qquad I_2(\word f)(x) = x$, \qquad
	$\beta_2 \colon \VPL \functionto D_2, \, \beta_2(\ply) =  \beta_1(\plx) = 1$}
\end{frame}

\begin{frame}{Aufgabe 2}
	\[ F := \plexists \plx {\plka \word{f(x)} \pleq \plx \aland \plx \pleq \ply \plkz}  \]
	
	Gebt eine Interpretation $(D_1, I_1)$, $D_1 \nsubseteq \R$, und eine Variablenbelegung $\beta_1$ so an, dass $val_{D_1, I_1, \beta_1}(F) = \W$ gilt.\\
	\smallskip
	\visible<2-|handout:2>{Die Interpretation $(D_1, I_1)$ mit $ D_1 = \{ \word a, \word b \}^*, \, I_1(\word f)(w) = \word a$ und die Variablenbelegung $\beta_1 \colon \VPL \functionto D_1$, $v \mapsto \word a$, tun es.} \\
	\medskip
	
	\[ G := \alnot\word{S(z)} \aland \plexists \plx {\plka \word{S(x)} \aland \plall \ply \word{R(x,y)} \plkz}  \]
	
	Gebt eine Interpretation $(D_2, I_2)$, $D_2 \subseteq \N_0$, und eine Variablenbelegung $\beta_2$ so an, dass $val_{D_2, I_2, \beta_2}(G) = \W$ gilt.\\
	\smallskip
	\visible<3-|handout:2>{Die Interpretation $(D_2, I_2)$ mit $ D_2 = \set{0, ..., 5}, \, I_2(\word S) = \set{n \Mid \text{$n$ ist prim}}, I_2(\word R) = {\geq} $ und die Variablenbelegung $\beta_2 \colon \VPL \functionto D_2$, $\plz \mapsto 4, \, \text{...Rest beliebig...}$, tun es.}
	
\end{frame}

\thasse{
	%	\begin{frame}{Rückblick: Prädikatenlogik}
	%		\begin{itemize}[<+->]
	%			\item Deutlich komplizierterer Aufbau als Aussagenlogik
	%			\item Auswertung mit Interpretation und Variablenbelegung
	%			\item Quantoren erlauben allgemeine Aussagen
	%		\end{itemize}
	%	\end{frame}
	
	\begin{frame}{Wahr oder Falsch?}
	Sei $\RPL = \{R,S\}$ mit $\ar(R) = 2$ und $\ar(S) = 1$, \\
	$\FPL = \{f,g\}$ mit $\ar(f) = 1$ und $\ar(g) = 2$ 
	\TrueQuestionE{$\word{R(y,g(x,y))}$ ist präd.log. syntaktisch korrekt.}{}
	\FalseQuestionE{$\word{f(S(x))}$ ist präd.log. syntaktisch korrekt.}{Eine Relation kann nicht innerhalb einer Funktion auftauchen, da sie keinen Term, sondern eine atomare Formel darstellt.}
	\FalseQuestionE{\enquote{Nicht alle Kinder spielen nicht.} $\equiv \plall \word{x\,(child(x)} \alimpl \word{play(x)}) $}{Es wird nur ausgesagt, dass es mindestens ein Kind gibt, das spielt.}
	%TODO: Eine W/F-Frage zu freien/gebundenen Variablen
	%TODO: Eine W/F-Frage zu PL Substitution
\end{frame}

}

\section{Eine Weihnachtsgeschichte}
%% Zum Aufwärmen

\begin{frame}{Dr. Meta am Nordpol}
	Der ebenso geniale wie hochwissenschaftliche Superbösewicht Dr. Meta ist entsetzt.
	Seit Jahren sinken seine Bekanntheitswerte, mittlerweile kennt nur noch eine kleine Gruppe von Studenten an einer einzigen Universität seinen Namen. (Vielleicht hätte er seine Twitter-Kampagne aktiver verfolgen sollen – nein, die Übungsblätter in GBI zu infiltrieren war wichtiger!)\\
	\bigskip \pause
	Damit alle Welt von seinen Schurkentaten erfährt, verfolgt er einen ehrgeizigen Plan: Er möchte seine Memoiren an sämtliche Haushalte verteilen. Dies von Hand zu erledigen ist eines Superschurkens seiner Größe jedoch nicht würdig. Daher möchte er das effizienteste und flächendeckendste Logistiksystem der Welt verwenden:\\ 
	Den Weihnachtsmann! Statt Geschenken soll es dieses Jahr Angst und Schrecken unterm Tannenbaum geben.
\end{frame}

\begin{frame}{Dr. Meta am Nordpol}
	\only<1|handout:1>{
	Am Nordpol angekommen muss Dr. Meta leider feststellen, dass das Weihnachtsdorf nicht so leicht zu erreichen ist wie erhofft.\\
	Nach Tagen des Umherirrens gerät er an eine Hütte, von der zwei Wege fortführen. Vor der Hütte steht ein Schild:\\
	\bigskip
	}
	\only<1-3|handout:2>{
	%„Willkommen im Servicezentrum für unangemeldete Eindringlinge!\\
	„Gib acht, Eindringling. \\
	\smallskip
	}
	\only<2-3|handout:2>{
	Einer dieser beiden Wege führt direkt zum Weihnachtsdorf, der andere jedoch...“\\
	Nur der Wächterwichtel weiß, welcher Weg der richtige ist.\\
	\textbf{Streng abwechselnd sagt er an einem Tag die Wahrheit und am darauf folgenden Tag lügt er}.\\ 
	\smallskip
	Keiner außer ihm weiß, ob der heute die Wahrheit sagt oder aber heute lügt.\\
	}
	\visible<3|handout:2>{
	Ihr dürft dem Wichtel genau eine Frage stellen, danach müsst ihr euch für einen der Wege entscheiden.\\
	\bigskip
	Helft Dr. Meta, das Weihnachtsdorf zu erreichen, indem ihr die richtige Frage stellt.\\
	}
\end{frame}

\begin{frame}{Dr. Meta am Nordpol\textsuperscript{\thefootnote}}
	\stepcounter{footnote}\footnotetext{Rätselidee frei übernommen nach: Der Philosoph am Scheideweg} % Fuck you, LaTeX...
		Helft Dr. Meta, das Weihnachtsdorf zu erreichen, indem ihr die richtige Frage stellt. %Wie lautet die richtige Frage?
	\\[1em]
	\impl „Was würdest du sagen, wenn ich dich morgen fragen würde, ob der linke Weg zum Weihnachtsdorf führt?“\\
	\bigskip
	
	%Und wenn der Philosoph jeden Morgen würfelt, ob er die Wahrheit sagt oder lügt?\\ 
	%\only<2|handout:2>{
	%Was würdest du sagen, wenn ich dich \textbf{heute} fragen würde, ob der linke Weg der richtige ist?\\ }
\end{frame}

\begin{frame}	
	\begin{block}{Was ihr nun wissen solltet}
		\begin{itemize}
			\daniel{
				\item Wie prädikatenlogische Formeln aufgebaut sind
				\item Wie man damit präzise Aussagen trifft
				\item Syntaxkram :P
			}
			\item Wie prädikatenlogische Formeln ausgewertet werden
		\end{itemize}
	\end{block}
	
	\begin{block}{Was nächstes Mal kommt}
		\begin{itemize}
			\item Alles korrekt? – Beweise mit dem Hoare-Kalkül
		\end{itemize}
	\end{block}
\end{frame}

\xkcdframevert{835}{Frohe Weihnachten und bis nächstes Jahr! \smiley}{2.2}
%\lastframetitled{0.47}{0}{xkcd/christmastree.png}{https://www.xkcd.com/835}{\vspace{-1.66\baselineskip}\\Frohe Weihnachten und einen \\ guten Start ins neue Jahr! \smiley}
\slideThanks

\end{document}